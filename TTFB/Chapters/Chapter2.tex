\chapter{Introducción específica} % Main chapter title

\label{Chapter2}

%----------------------------------------------------------------------------------------
%	SECTION 1
%----------------------------------------------------------------------------------------

En este capítulo se abordan los conceptos fundamentales que sustentan el desarrollo del trabajo. 
En primer lugar, se presenta una introducción al aprendizaje supervisado y se describen algoritmos 
como la regresión lineal y los modelos basados en árboles. Luego, se desarrolla el apartado 
de teledetección, donde se detallan las plataformas utilizadas, las colecciones de datos y las 
características de las bandas del satélite Sentinel-2. Finalmente, se incluye una sección dedicada 
al procesamiento de imágenes.

\section{Aprendizaje supervisado}
En el aprendizaje automático supervisado, los datos anotados conforman el conjunto de entrenamiento, mientras 
que los datos no anotados integran el conjunto de prueba. Cuando las anotaciones corresponden a valores 
discretos, se denominan etiquetas de clase, mientras que, si son valores numéricos continuos, se las conoce 
como valores objetivo continuos. Los problemas de aprendizaje supervisado se dividen en dos grandes categorías,
clasificación y regresión. En el caso de la regresión, el modelo se entrena a partir de un conjunto de datos 
etiquetados para aprender las relaciones existentes entre las variables y, posteriormente, predecir 
resultados de valor continuo sobre nuevos datos no observados \citep{mitchell2015}.

\subsection{Regresión lineal}

La regresión lineal \citep{shetty2022} es uno de los métodos más simples y ampliamente utilizados dentro de los 
modelos de regresión. Se trata de un enfoque lineal que busca modelar la relación existente entre
una variable dependiente (respuesta escalar) y una o más variables independientes (predictoras). 
Este método se emplea con frecuencia en tareas como la predicción de precios, la estimación de 
calificaciones o cualquier otro problema que implique valores continuos. En términos generales, la 
regresión lineal es un proceso estadístico utilizado con fines 
predictivos dentro del campo del aprendizaje automático. El modelo asume que la variable de salida 
$y$ y puede expresarse como una combinación lineal de las variables de entrada $x$. 


Por ejemplo, puede considerarse una ecuación lineal que combina un conjunto de variables de entrada $x$ y produce un valor de salida previsto 
$y$ correspondiente a esas entradas. En este contexto, tanto las variables de entrada como la salida son de naturaleza numérica.
Cada variable de entrada está asociada a un coeficiente, que actúa como un factor de escala y determina el grado 
de influencia de dicha variable sobre la respuesta. Además, se incorpora un término independiente, 
conocido como intersección o ordenada al origen, que representa el valor de salida cuando todas las variables de entrada son nulas.
El proceso de entrenamiento del modelo de regresión lineal consiste, en estimar los valores 
óptimos de estos coeficientes de manera que la ecuación resultante describa adecuadamente la relación 
entre los datos de entrada y los valores observados.

La figura \ref{fig:lr} muestra el ajuste de la línea de regresión lineal sobre los 
puntos de datos observados, lo que evidencia la correspondencia entre el modelo y el comportamiento de las variables.

\begin{figure}[h]
	\centering
	\includegraphics[width=0.6\textwidth]{./Figures/lr.png}
	\caption{Ilustración del ajuste de un modelo de regresión lineal sobre un conjunto de datos\protect\footnotemark.}
	\label{fig:lr}
\end{figure}
\footnotetext{Imagen adaptada de \citep{shetty2022}}

\subsection{Modelos basados en árboles}

Para las tareas de regresión se utilizan diversos algoritmos, entre ellos, los modelos basados en árboles que son algoritmos de aprendizaje supervisado que usan estructuras tipo árbol para dividir
datos y hacer predicciones. Los árboles simples siguen reglas de sí/no
hasta formar grupos homogéneos, ofrecen una interpretación sencilla y admiten distintos tipos de variables.
Se pueden categorizar en árboles de decisión individuales, métodos de ensamble que combinan múltiples árboles, 
y otras variantes basadas en \textit{boosting}. A continuación, se detallan los utilizados en este trabajo: 

\begin{itemize}
	\item \textit{Decision Tree Regressor} \citep{dtr}: es un modelo que predice un valor continuo divide recursivamente el conjunto
	de datos. En cada división, busca la característica que mejor reduce la varianza de la variable objetivo. 
	\item \textit{Random Forest Regressor} \citep{rfr}: introduce aleatoriedad al considerar solo un subconjunto de características 
	para cada división. Construye un gran número de árboles de decisión individuales, entrenando cada uno en una 
	submuestra del conjunto de datos seleccionada con reemplazo.
	\item \textit{LGBM Regressor} \citep{lgbm}: utiliza la técnica de crecimiento de árboles \textit{leaf-wise} (por hoja), enfocándose en hojas con la mayor pérdida a optimizar. Tiene alta 
	eficiencia y velocidad, especialmente en grandes conjuntos de datos.
	\item \textit{Gradient Boosting Regressor} \citep{gbr}: construye un modelo aditivo en una moda gradual \textit{stage-wise}, cada árbol intenta corregir 
	secuencialmente los errores residuales del conjunto de árboles predecesores para mejorar la predicción general.
	\item \textit{XGB Regressor} \citep{xgbr}: este modelo incluye métodos de regularización para controlar el sobreajuste, el uso de una aproximación de 
	segundo orden (Taylor) de la función de pérdida (Hessiano) para una optimización más precisa y el soporte para el procesamiento paralelo. 
	Es conocido por ofrecer resultados de vanguardia en tareas de regresión.
\end{itemize}

Las bibliotecas \texttt{XGBoost} \citep{xgboost} y \texttt{scikit-learn} \citep{scikitlearn} implementan los algoritmos mencionados.

\section{Teledetección}

La teledetección ha sido reconocida como una importante fuente de información durante las
últimas décadas en una amplia gama de aplicaciones de observación de la Tierra. La
periodicidad se considera el requisito más importante, ya que su cumplimiento permite 
abordar los problemas identificados en tiempo real. Durante la última década, estos
requisitos se han cumplido cada vez más gracias a una flota de satélites de reconocimiento
con capacidades avanzadas que permiten aplicaciones agrícolas rentables. Además, varias 
agencias espaciales y proveedores de productos satelitales han adoptado una política 
de acceso libre y sin restricciones a los datos; por ejemplo, la Agencia Espacial 
Europea (ESA) para el programa Copernicus, incluidos los satélites Sentinel-1 y Sentinel-2,
aplica una política de datos libre, completa y abierta. 

Una aplicación en la que la observación de la tierra puede ser una importante fuente de 
información es la agricultura a pequeña escala. La alta resolución espacial necesaria 
para detectar los campos agrícolas de los pequeños agricultores, la resolución radiométrica
para discriminar entre los tipos de plantas en este entorno heterogéneo y la 
resolución temporal necesaria para supervisar los acontecimientos y la evolución 
(por ejemplo, las prácticas agrícolas y el crecimiento de los cultivos) durante la
temporada de cultivo se han podido obtener últimamente gracias a los sensores satelitales.
La exploración de cultivos requiere mucho tiempo y mano de obra, y por lo tanto es 
costosa, la observación de la tierra presenta una alternativa viable \citep{Stratoulias2017}. En la figura 
\ref{fig:remotesensing} se presenta un esquema de proceso de teledetección.

\begin{figure}[h]
	\centering
	\includegraphics[width=0.6\textwidth]{./Figures/esquema_teledeteccion.png}
	\caption{Esquema de proceso de teledetección. (a) Fuente de iluminación. (b) 
	Sensor. (c) Antena en tierra. (d) Software. (e) Usuarios\protect\footnotemark.}
	\label{fig:remotesensing}
\end{figure}
\footnotetext{Imagen adaptada de \citep{pntign}}

\subsection{Plataformas de teledetección}
\label{subsec:plataformas}
En la tabla \ref{tab:plataformas} se presentan plataformas de teledetección disponibles y requisitos sugeridos 
(*marginal, **óptimo) para supervisar la fenología de la floración forestal \citep{Dixon2023}.
		
\begin{table}[h]
	\centering
	\caption{Plataformas de teledetección disponibles.}
		\begin{tabular}{l c c c c}    
			\toprule
			\textbf{Sensor} & \textbf{Resolución} & \textbf{Frecuencia} & \textbf{Extensión} & \textbf{Coste} \\
			\midrule
			PlanetScope & 3m*  & Diariamente**  & Regional** & Moderado* \\		
			Sentinel-2	 & 10 - 20 m  & 5 días  & Continental**   & Bajo**  \\
			Landsat	 & 30 m  & 16 días  & Continental** & Bajo** \\
			\bottomrule
		\end{tabular}
		\label{tab:plataformas}
\end{table}

\subsection{Colecciones de datos}

El programa Copernicus \citep{copernicus}, una iniciativa conjunta de la Comisión Europea y la Agencia Espacial Europea (ESA), se implementa a través 
de la constelación de satélites Sentinel, desarrollados por la ESA. Estos satélites ofrecen una amplia gama de datos, que incluye:
imágenes de radar (Sentinel-1A/1B), imágenes ópticas de alta resolución (Sentinel-2A/2B), datos de monitoreo ambiental y 
climático para océanos y tierras (Sentinel-3), y datos sobre la calidad del aire (Sentinel-5P).

\subsubsection{Características de bandas de Sentinel-2}
\label{subsec:bandas}
\begin{table}[!htpb]
  \centering
  \caption{Bandas espectrales de Sentinel-2 \citep{Sentinel2}.}
  \begin{tabular}{l c c c}    
    \toprule
     \textbf{Nombre} & \textbf{Longitud de onda ($\mu$m)} & \textbf{Resolución (m)}\\
    \midrule
	B1 - Aerosol & 0,43 - 0,45 & 60 \\
    B2 - Azul & 0,45 - 0,52 & 10  \\		
    B3 - Verde & 0,54 - 0,57  & 10  \\
    B4 - Rojo & 0,65 - 0,68  & 10  \\
    B5 - Borde rojo 1 & 0,69 - 0,71 & 20  \\
    B6 - Borde rojo 2 & 0,73 - 0,74 & 20  \\
	B7 - Borde rojo 3 & 0,77 - 0,79 & 20  \\
    B8 - NIR & 0,78 - 0,90  & 10  \\
	B8A - Borde rojo 4 & 0,85 - 0,87  & 20 \\
	B9 - Vapor de agua &  0,93 - 0,95 & 60 \\
	B10 - Cirro & 1,36 - 1,39 & 60 \\
    B11 - SWIR 1 & 1,56 - 1,65 & 20  \\
	B12 - SWIR 2 & 2,10 - 2,28  & 20  \\
    \bottomrule
  \end{tabular}
  \label{tab:tab-bandas}
\end{table}

\section{Procesamiento de Imágenes}
El procesamiento de imágenes satelitales se ha convertido en una herramienta esencial para diversos
campos, como la agricultura, la gestión de recursos naturales, la monitorización ambiental y 
la planificación urbana. Actualmente, existen varios métodos y técnicas avanzadas utilizados para
analizar y extraer información útil de estas imágenes.
	
\subsection{\textit{TIFF (Tagged Image File Format)}}
El formato TIFF es un estándar basado en etiquetas que se utiliza para almacenar e intercambiar 
imágenes ráster. La especificación GeoTIFF amplía este formato al incorporar un conjunto de 
etiquetas adicionales que permiten describir la información cartográfica asociada a las imágenes.
De esta manera, GeoTIFF facilita el uso de imágenes provenientes  sistemas de imágenes por 
satélite, fotografías aéreas escaneadas, mapas escaneados o como resultado de análisis 
geográficos. Su objetivo es establecer un vínculo directo entre la imagen ráster y un sistema de 
coordenadas \cite{Mahammad2003}.

\subsection{Índices de vegetación}

La composición espectral del flujo radiante que emana de la superficie terrestre 
proporciona información sobre las propiedades físicas del suelo, el agua y la vegetación 
en entornos terrestres. Las técnicas, modelos e índices de teledetección están diseñados 
para convertir esta información espectral en una forma fácilmente interpretable \citep{Bannari1995}. La construcción de un índice de 
vegetación implica la combinación de diferentes bandas sensibles para eliminar el impacto de los efectos
ambientales de fondo (por ejemplo, el suelo sin vegetación, las masas de agua).

A continuación, se presentan índices de vegetación relevantes para este trabajo:

\begin{itemize}
	\item \textit{Normalized Difference Vegetation Index} (NDVI) \citep{Xue2017}: se calcula como 
	   	relación normalizada entre las bandas roja e infrarroja cercana. Tiene una reacción sensible a la vegetación verde,
		 incluso en zonas con cobertura vegetal escasa, es sensible a los efectos del brillo y color del suelo,
		la atmósfera, las nubes y la sombra de las nubes, y la sombra del dosel foliar.
	\item \textit{Enhanced Vegetation Index} (EVI) \citep{Xue2017}: corrige los efectos del suelo y de la atmósfera. Su formulación 
		incluye los valores de NIR, R y B. El término L representa un parámetro de ajuste asociado
		  al suelo, cuyo valor se establece en 1. Además, se incorporan dos parámetros constantes con valores de 6 y 
		  7.5, respectivamente.
	\item \textit{Atmospherically Resistant Vegetation Index} (ARVI) \citep{Xue2017}: se utiliza para eliminar los efectos de los
		aerosoles atmosféricos. Se basa en el supuesto de que la atmósfera afecta significativamente a R en comparación con el 
		NIR y reduce la dependencia de este índice de vegetación de los efectos atmosféricos. 
	\item \textit{Renormalized Difference Vegetation Index} (RDVI) \citep{vescovo2012}: su objetivo es linealizar las relaciones entre 
		el índice, los parámetros biofísicos y reducir el efecto de saturación en coberturas altas. Utiliza las bandas infrarrojo cercano
		 y roja.
	\item \textit{Green Normalized Difference Vegetation Index} (GNDVI) \citep{gndvi}: es una variante del NDVI que sustituye 
		la banda roja por la banda verde, esta modificación permite una mayor sensibilidad al contenido de clorofila y a la 
		absorción del nitrógeno en el follaje.
	\item \textit{Structure Insensitive Pigment Index} (SIPI) \citep{vi}: se usa para monitorear la salud de las plantas en 
		regiones con alta variabilidad en la estructura del dosel o el índice de área foliar, 
		para la detección temprana de enfermedades de las plantas u otras causas de estrés.
	\item \textit{Normalized Difference Red Edge Index} (NDRE) \citep{ndre}: es un método para medir la cantidad de clorofila 
	en las plantas. Su aplicación resulta eficaz hacia la mitad o el final del ciclo de cultivo, cuando las plantas 
	alcanzan su madurez y se encuentran próximas a la cosecha.
	\item \textit{Leaf Chlorophyll Index} (LCI) \citep{nata2021}: es un índice de vegetación que involucra la banda lateral roja,
	 que utiliza las características de reflectancia espectral de la banda lateral roja
	 y la banda del infrarrojo cercano para mostrar las diferencias en el contenido de clorofila.  
\end{itemize}

