% Chapter 1

\chapter{Introducción general} % Main chapter title

\label{Chapter1} % For referencing the chapter elsewhere, use \ref{Chapter1} 
\label{IntroGeneral}

En este capítulo se introduce la problemática y se interpreta la importancia que
implica el trabajo propuesto. Luego, se realiza un análisis del estado de arte sobre la gestión 
eficiente de salud de cultivos y predicción de floración en durazneros. Finalmente, se 
puntualizan los objetivos y el alcance del trabajo.



%----------------------------------------------------------------------------------------

% Define some commands to keep the formatting separated from the content 
\newcommand{\keyword}[1]{\textbf{#1}}
\newcommand{\tabhead}[1]{\textbf{#1}}
\newcommand{\code}[1]{\texttt{#1}}
\newcommand{\file}[1]{\texttt{\bfseries#1}}
\newcommand{\option}[1]{\texttt{\itshape#1}}
\newcommand{\grados}{$^{\circ}$}

%----------------------------------------------------------------------------------------

%\section{Introducción}

%----------------------------------------------------------------------------------------
\section{Contexto del trabajo}

En la EEA San Pedro de INTA se busca intensificar de manera sustentable la producción
de frutales, hortalizas y viveros. Actualmente, la inspección y evaluación
dependen de visitas de campo y observaciones manuales, procesos costosos, laboriosos
 y con alcance limitado. Además, el cambio climático y la variabilidad
meteorológica añaden capas adicionales de complejidad a la gestión agronómica.

El Laboratorio de Biotecnología cuenta con datos históricos sobre las etapas de
floración y cosecha de frutos en durazneros. En este contexto, las imágenes
satelitales se presentan como una herramienta poderosa, ya que ofrecen datos
consistentes y de amplia cobertura, y permiten evaluar la vegetación de forma
precisa y objetiva mediante índices especializados.

El objetivo de este trabajo es automatizar la gestión de montes frutales y la predicción de la fecha 
de floración a partir de datos disponibles e índices derivados de imágenes satelitales. Para ello, 
se emplean métodos de aprendizaje automático junto con técnicas de preprocesamiento de imágenes. Se 
propone el desarrollo de una herramienta accesible para los usuarios, que permita optimizar los 
resultados y reducir el tiempo y esfuerzo requeridos en el proceso.

En la figura \ref{fig:flujotesis} se puede observar el flujo de información que integra
las etapas de procesamiento de imágenes y entrenamiento de modelos predictivos planteadas para este trabajo.

\pagebreak

\begin{figure}[!htbp]
	\centering
	\includegraphics[width=\textwidth]{./Figures/flujo_tesis_tffb.png}
	\caption{Diagrama de preproceso de datos y entrenamiento predictivo.}
	\label{fig:flujotesis}
\end{figure}

\pagebreak


\section{Estado del arte}

En los últimos años, la recopilación masiva de datos agrícolas ha transformado la 
toma de decisiones en la gestión de cultivos, lo que ha permitido optimizar su salud
y rendimiento. Las etapas fenológicas de la vegetación actúan como
indicadores importantes en el seguimiento del crecimiento de la vegetación y la evaluación de cómo 
el cambio climático puede afectar a la vegetación. 

La disponibilidad de datos obtenidos mediante teledetección con altas resoluciones espaciales y espectrales permite 
utilizar modelos de aprendizaje automático y aprendizaje profundo para apoyar la toma de decisiones
en la agricultura \citep{YEM2023}. La fenología precisa de las plantas es crucial para la producción y las
operaciones agrícolas porque guía cronogramas específicos para la fertilización, el riego, el control de plagas, la 
cosecha y la reproducción \citep{ML2022}.


\subsection{Importancia de la teledetección en la agricultura}
La teledetección cumple un papel fundamental en la agricultura, proporciona información valiosa sobre
el estado de los cultivos, las condiciones ambientales y las prácticas de manejo de suelo. Su 
relevancia radica en su capacidad para recopilar datos sobre grandes extensiones de forma rápida, no invasiva 
y frecuente, lo que permite a los agricultores, investigadores y responsables políticos tomar decisiones informadas 
y optimizar la productividad agrícola. A continuación, se describen algunos aspectos que destacan la importancia 
de la teledetección en este ámbito:

\begin{itemize}
  \item Monitoreo y gestión de cultivos: la teledetección permite el monitoreo sistemático de la salud y el crecimiento de
        los cultivos a lo largo de la temporada de cultivo. Mediante el análisis de imágenes multiespectrales, 
        los agricultores pueden detectar signos tempranos de estrés, deficiencias nutricionales, infestaciones de plagas y enfermedades, 
        lo que permite una intervención oportuna para mitigar posibles pérdidas de rendimiento. Este enfoque proactivo de la gestión de 
        los cultivos ayuda a optimizar la asignación de recursos, como el agua, los fertilizantes y los pesticidas, lo que se traduce en 
        un aumento del rendimiento y una mayor eficiencia en el uso de los recursos \citep{Kumar}.
  \item Predicción y evaluación del rendimiento: los datos de teledetección, combinados con técnicas analíticas avanzadas,
        pueden utilizarse para predecir el rendimiento de los cultivos y evaluar la variabilidad del rendimiento en los campos.
        Mediante el análisis de índices de vegetación y modelos de estimación de biomasa derivados de imágenes satelitales 
        o de drones, los agricultores pueden anticipar las fluctuaciones del rendimiento, optimizar los calendarios de cosecha
        y tomar decisiones informadas sobre la comercialización \citep{Kumar}.
\end{itemize}


\subsection{Monitoreo de la salud de cultivos}
A continuación, se presentan estudios enfocados en durazneros realizados en la región pampeana y patagónica de Argentina:
\begin{itemize}
  \item El trabajo titulado La variación temporal del índice NDVI predice los cambios temporales de la cobertura vegetal en las tierras secas de la Patagonia argentina
        demuestra que las series temporales de índices de vegetación derivados de sensores satelitales y productos de mediana resolución, combinadas con mediciones de campo 
        y modelos estadísticos, son herramientas útiles para monitorear el estado vegetativo de huertos
        de duraznero, detectar variaciones intraestacionales y apoyar la identificación de etapas fenológicas relevantes para la
        floración \citep{Patagonia2021}.
  \item El estudio \textit{Phenology and reproductive traits of peaches and nectarines in Central-East Argentina} describe los patrones fenológicos, la densidad de 
        floración y los rasgos reproductivos de distintas variedades de duraznero y nectarina en la región pampeana. Los autores analizan la variabilidad entre 
        cultivares y años, y discuten cómo los factores ambientales y las prácticas de manejo influyen en la ocurrencia y sincronía de la 
        floración. Los resultados aportan series temporales y relaciones entre cultivares y clima que resultan útiles para la parametrización
        de modelos fenológicos, por ejemplo, para estimar las fechas de plena floración a partir de acumulados térmicos o del cumplimiento de 
        los requerimientos de frío, lo que convierte a este trabajo en una fuente local altamente pertinente para la predicción de la floración del duraznero \citep{Gariglio2006}.
  \item El trabajo \textit{Comparison of methods for estimation of chilling and heat requirements of nectarine and peach genotypes for flowering} compara distintos modelos 
        y métodos para estimar los requerimientos de frío y de acumulación térmica necesarios para la apertura de yemas florales en genotipos de nectarina y
        genotipos de duraznero cultivadas en Argentina. Al evaluar y contrastar fórmulas para contabilizar horas/fríos efectivos y distintas curvas de 
        acumulación térmica, el estudio proporciona parámetros y recomendaciones prácticas para ajustar modelos fenológicos locales que predicen la fecha de floración según las 
        condiciones climáticas de la región pampeana argentina \citep{Maulion2014}.
      \end{itemize}

\section{Alcance y objetivos}

El objetivo principal de este trabajo fue desarrollar un algoritmo que permita descargar y analizar imágenes satelitales con el fin
de vincularlas a características fenológicas de los durazneros. Se buscó determinar a partir de los 
datos disponibles, el avance de las etapas de floración y maduración de los frutos, así como explorar información satelital 
relevante para la predicción de la fecha de floración.

A continuación, se detallan las actividades incluidas en este trabajo:

\begin{itemize}
  \item Evaluación de las diferentes fuentes de datos disponibles:
  \begin{itemize}
    \item Datos tabulares, que corresponden a mediciones de campo
    de las etapas de floración y cosecha de frutos, obtenidos durante 5
    años.
    \end{itemize}
  \item La descarga de imágenes satelitales, correspondientes a parcelas de durazneros monitoreados. 
  \item La determinación del progreso de las etapas de floración y maduración de
  los frutos en el duraznero.
  \item La exploración de información satelital en etapa de crecimiento para anticipar la fecha de floración.
  \item El desarrollo de una herramienta de fácil acceso para el equipo del cliente.
  \item La elaboración de un informe que detalle el procedimiento realizado y los resultados obtenidos.  
  \end{itemize}

  Los siguientes elementos quedan fuera del alcance del presente trabajo:
  \begin{itemize}
    \item El desarrollo de una interfaz web para el sistema.
    \item El despliegue del desarrollo en un entorno de producción.
    \end{itemize}

