\chapter{Diseño e implementación} % Main chapter title

\label{Chapter3} % Change X to a consecutive number; for referencing this chapter elsewhere, use \ref{ChapterX}
Este capítulo aborda en detalle el proceso de preparación, análisis e integración de los datos utilizados 
en el estudio. En primer lugar, se describen las etapas de preprocesamiento, que incluyen el flujo de procesamiento
 de datos crudos, con la gestión y extracción de información a partir de archivos satelitales, y el flujo 
 de procesamiento de datos tabulares, donde se trabaja sobre la estructura, depuración y reorganización de 
 los registros de floración y cosecha. Estas fases permiten obtener un conjunto de datos consolidado, apto
  para el análisis y modelado posterior. Posteriormente, se presenta la caracterización de los montes frutales 
  a partir de índices de vegetación, así como las limitaciones encontradas 
  y las soluciones propuestas. Además, se detalla la fusión de las 
  distintas fuentes de datos y se introduce la arquitectura de inteligencia
   artificial, que integra los componentes analíticos y predictivos del sistema. En conjunto, 
   estas secciones conforman la base metodológica sobre la cual se desarrolla la predicción de la floración en 
   los durazneros bajo estudio.

\definecolor{mygreen}{rgb}{0,0.6,0}
\definecolor{mygray}{rgb}{0.5,0.5,0.5}
\definecolor{mymauve}{rgb}{0.58,0,0.82}

%%%%%%%%%%%%%%%%%%%%%%%%%%%%%%%%%%%%%%%%%%%%%%%%%%%%%%%%%%%%%%%%%%%%%%%%%%%%%
% parámetros para configurar el formato del código en los entornos lstlisting
%%%%%%%%%%%%%%%%%%%%%%%%%%%%%%%%%%%%%%%%%%%%%%%%%%%%%%%%%%%%%%%%%%%%%%%%%%%%%
\lstset{ %
  backgroundcolor=\color{white},   % choose the background color; you must add \usepackage{color} or \usepackage{xcolor}
  basicstyle=\footnotesize,        % the size of the fonts that are used for the code
  breakatwhitespace=false,         % sets if automatic breaks should only happen at whitespace
  breaklines=true,                 % sets automatic line breaking
  captionpos=b,                    % sets the caption-position to bottom
  commentstyle=\color{mygreen},    % comment style
  deletekeywords={...},            % if you want to delete keywords from the given language
  %escapeinside={\%*}{*)},          % if you want to add LaTeX within your code
  %extendedchars=true,              % lets you use non-ASCII characters; for 8-bits encodings only, does not work with UTF-8
  %frame=single,	                % adds a frame around the code
  keepspaces=true,                 % keeps spaces in text, useful for keeping indentation of code (possibly needs columns=flexible)
  keywordstyle=\color{blue},       % keyword style
  language=[ANSI]C,                % the language of the code
  %otherkeywords={*,...},           % if you want to add more keywords to the set
  numbers=left,                    % where to put the line-numbers; possible values are (none, left, right)
  numbersep=5pt,                   % how far the line-numbers are from the code
  numberstyle=\tiny\color{mygray}, % the style that is used for the line-numbers
  rulecolor=\color{black},         % if not set, the frame-color may be changed on line-breaks within not-black text (e.g. comments (green here))
  showspaces=false,                % show spaces everywhere adding particular underscores; it overrides 'showstringspaces'
  showstringspaces=false,          % underline spaces within strings only
  showtabs=false,                  % show tabs within strings adding particular underscores
  stepnumber=1,                    % the step between two line-numbers. If it's 1, each line will be numbered
  stringstyle=\color{mymauve},     % string literal style
  tabsize=2,	                   % sets default tabsize to 2 spaces
  title=\lstname,                  % show the filename of files included with \lstinputlisting; also try caption instead of title
  morecomment=[s]{/*}{*/}
}


%----------------------------------------------------------------------------------------
%	SECTION 1
%----------------------------------------------------------------------------------------
\section{Enfoque para resolver el problema}

El Laboratorio de Biotecnología recopiló, durante un período de cinco años, datos sobre las 
fechas de floración y cosecha de durazneros, junto con la ubicación geográfica de cada parcela. 
Con esta información se evaluó la posibilidad de descargar imágenes satelitales correspondientes 
a dichas parcelas, tanto en los días previos como durante la floración. El objetivo fue calcular 
índices de vegetación y a partir de ello, analizar información satelital relevante que permita anticipar el momento
de la floración.

El proceso del trabajo se compone en: 

\begin{enumerate}
  \item Adquisición y preprocesamiento de datos
      \begin{itemize}
        \item Integración y limpieza de datos crudos: fechas de floración, cosecha y ubicaciones de parcelas.
      \end{itemize}
  \item Investigación de colecciones de datos de imágenes satelitales
      \begin{itemize}
        \item Análisis detallado de los catálogos de datos proporcionados por Earth Engine 
              y la constelación PlanetScope de Planet.
      \end{itemize}
  \item Obtención y procesamiento de imágenes satelitales
      \begin{itemize}
        \item Desarrollo de un script en Python para interactuar con la API de Google Earth Engine.
        \item Descarga de imágenes satelitales de las parcelas en fechas previas y durante la floración.
        \item Lectura y procesamiento de los archivos para extraer índices de vegetación y analizar
              su evolución en fechas previas a la floración.
      \end{itemize}
  \item Construcción del \textit{dataset}
      \begin{itemize}
        \item Confección de un nuevo conjunto de datos enriquecido con índices de vegetación.
        \item Aplicación de técnicas de ingeniería de características.
        \item Estructuración y limpieza del dataset final.
      \end{itemize}
  \item  Modelado y entrenamiento
      \begin{itemize}
        \item Preprocesamiento de datos.
        \item Entrenamiento de modelos.
      \end{itemize}
  \item Evaluación y ajuste del modelo
      \begin{itemize}
        \item Cálculo de métricas de desempeño del modelo.
        \item Validación de resultados. Retroalimentación para mejora y reentrenamiento de modelo.
      \end{itemize}
	\end{enumerate}

\section{Proceso de adquisición de imágenes de satélite}

El proceso completo de recopilación de imágenes aéreas para los cultivos de durazno puede desglosarse 
en tres etapas principales. La primera, abarca el análisis y las 
consideraciones fundamentales para la selección de la fuente de adquisición de datos. La segunda,
se centra en la problemática detectada respecto a la proximidad entre parcelas y la estrategia adoptada
para su resolución. Finalmente, la tercera etapa está vinculada con la definición de las fechas de 
descarga. 

\subsection{Consideraciones}
En la sección \ref{subsec:plataformas} se analizan las distintas plataformas de teledetección 
disponibles. A continuación, se detallan los criterios considerados para 
seleccionar la más adecuada:

\begin{itemize}
  \item Intervalo de revisión: se refiere a la periodicidad con la que el satélite captura
    nuevas imágenes. Dado que intervalos largos pueden generar variaciones significativas 
    en la vegetación, se priorizó un satélite con mayor frecuencia de captura.
  \item Disponibilidad temporal de los datos: se evaluó el rango de fechas en que los datos
    están disponibles. Como el análisis requiere información desde 2017, se seleccionó el 
    conjunto de datos que cubre dicho período.
  \item Cantidad de bandas espectrales: las bandas espectrales representan distintos rangos 
    del espectro electromagnético y permiten analizar diversas características de la superficie
    terrestre. Se optó por la plataforma que ofrece el mayor número de bandas, ya que son 
    fundamentales para el cálculo de los índices de vegetación.
  \item Accesibilidad: se consideró si el acceso a los datos es gratuito o pago. En este caso, 
    se priorizó el uso de fuentes de acceso libre.
  \item Resolución espacial: hace referencia al tamaño del píxel en cada banda espectral, lo 
    que influye en el nivel de detalle de la imagen. Aunque un menor tamaño de píxel proporciona
     mayor precisión, este criterio no fue determinante en la selección.
\end{itemize}

Tras analizar las distintas plataformas, se optó por Sentinel-2. Esta opción resultó más adecuada 
para el trabajo, ya que reúne las condiciones necesarias en cuanto a resolución temporal, acceso a 
datos históricos, riqueza espectral y disponibilidad gratuita.

\subsection{Proximidad de parcelas}

La figura \ref{fig:parcelasSP} muestra una imagen aérea de parcelas de árboles de durazno. Se observó 
que las parcelas están demasiado próximas entre sí. Esto dificultaba la correcta
extracción de los índices de vegetación, ya que la resolución espacial del satélite superaba el tamaño
individual de cada parcela. Para abordar esta limitación, se llevó a cabo una selección de 
parcelas que presentaran una distancia mínima adecuada entre sí, los recortes por debajo la figura principal
ilustran el conjunto final de parcelas seleccionadas tras este proceso.


\begin{figure}[h]
	\centering
	\includegraphics[width=0.9\textwidth]{./Figures/recorte_parcelas.png}
	\caption{Imagen satelital de parcelas. (a) Imagen con parcelas completas. (b) Parcelas seleccionadas\protect\footnotemark.}
	\label{fig:parcelasSP}
\end{figure}
\footnotetext{Imagen de elaboración propia a partir del proveedor Sentinel-2.}

\clearpage

\subsection{Calendario de obtención de imágenes}

En la etapa inicial del proyecto, la recolección de imágenes de satélite se delimitó a los 
periodos específicos de la floración de los árboles en estudio. Este lapso temporal comprendía, 
de forma aproximada, los meses de agosto, septiembre, octubre y noviembre, abarcando la ventana 
fenológica completa desde el inicio de la floración del primer ejemplar hasta su culminación 
en el último árbol. Esta definición inicial respondía a la necesidad de focalizar el análisis 
en la fase de máxima actividad biológica.

Posteriormente, a partir de un análisis y una revisión conjunta con el equipo del Laboratorio de 
Biotecnología, se determinó la conveniencia de expandir el marco temporal de la investigación. 
Esta ampliación presenta la importancia de incorporar la información satelital correspondiente a 
los meses inmediatamente previos a la floración. Se decidió incluir imágenes capturadas durante mayo, 
junio y julio.

La integración de estos meses pre-floración permite identificar tendencias previas a la floración y 
mejorar la precisión de los modelos predictivos que se verán en detalle más adelante.

\clearpage

\section{Área de estudio}  

En la figura \ref{fig:sanpedro} se observa el Partido de San Pedro, situado en la región del norte de la Provincia 
de Buenos Aires, Argentina. Esta zona es reconocida como un núcleo agroproductivo de 
significativa relevancia para el país. 

La estación INTA San Pedro, trabaja principalmente en cuatro grandes cadenas: las frutas, las hortalizas,
los viveros y las ornamentales. Además, por las características del territorio, también aborda a 
través de sus Agencias de Extensión los cultivos extensivos. Su trabajo está orientado hacia la 
producción integrada y de calidad, cuyos objetivos son proteger el medio ambiente, bregar por la 
salud de los operarios y consumidores y mejorar la competitividad de las empresas agropecuarias del sector.

\begin{figure}[h]
	\centering
	\includegraphics[width=0.9\textwidth]{./Figures/san-pedro-mapa.png}
	\caption{Imagen de Partido de San Pedro.\protect\footnotemark.}
	\label{fig:sanpedro}
\end{figure}
\footnotetext{Imagen de fuente.}

\clearpage

\subsection{Unidades experimentales}

La totalidad de las parcelas bajo estudio en este proyecto se encuentran ubicadas dentro de INTA 
San Pedro. Las parcelas seleccionadas forman parte de los ensayos productivos y experimentales que 
se llevan adelante en la institución.

En la figura \ref{fig:parcelas-enestudio} se presentan imágenes aéreas de los lotes de cultivo 
analizados, obtenidas y procesadas mediante la plataforma Google Earth Engine \citep{ee}, donde se identifican
las áreas correspondientes a cada parcela y su distribución espacial dentro del establecimiento.


\begin{figure}[!htbp]
	\centering
	\includegraphics[width=0.9\textwidth]{./Figures/total_parcelas.png}
	\caption{Parcelas de durazneros bajo estudio \protect\footnotemark.}
	\label{fig:parcelas-enestudio}
\end{figure}
\footnotetext{Imagen de elaboración propia a partir del proveedor Sentinel-2.}

\clearpage


\section{Preprocesamiento de datos}
El flujo general del procesamiento de datos consta de dos etapas principales: en la primera se gestionan y 
procesan las imágenes satelitales, y en la segunda se trabajan los datos tabulares vinculados con la floración 
y la cosecha.

\subsection{Flujo de procesamiento de datos crudos} 
En este apartado se describe el procedimiento aplicado para la gestión y procesamiento de las imágenes satelitales en 
su estado inicial. El flujo incluye la descarga de los archivos originales, su organización, y el procedimiento de 
generación del \textit{dataset} con información satelital.

\subsubsection{Gestión de archivos}
Se desarrolló un script que utiliza la librería earthengine-api \citep{eeapi} que realiza la descarga automática de imágenes satelitales a partir de 
los límites geográficos de cada parcela. Como salida, genera un conjunto de archivos 
en formato tiff, uno por parcela y obtenidos a intervalos de cinco días. Este procedimiento se aplicó para un período 
de cinco años de datos. 

A continuación, se ilustra la estructura final del directorio correspondiente a los datos recolectados:

\texttt{%
tiff/\\
|- 2017/\\
|  |- 2017-05-25\_2017-05-30\_clv2.tif\\
|  |- 2017-05-25\_2017-05-30\_clv3.tif\\
|  |- 2017-05-25\_2017-05-30\_clv5.tif\\
|  |- \dots\\
|- 2018/\\
|- 2019/\\
|- 2020/\\
|- 2021/\\
|- 2022/\\
|- 2023/%
}

En la tabla \ref{tab:tab-cantidadtifs} se resume la cantidad total de imágenes adquiridas por año, 
junto con sus respectivos tamaños y rangos temporales.

\begin{table}[h]
  \centering
  \caption{Total de archivos y periodo temporal por año.}
  \begin{tabular}{l c c c}    
    \toprule
     \textbf{Año} & \textbf{Total de archivos} & \textbf{Rango de tamaños} & \textbf{Rango de fechas}\\
    \midrule
    2017 & 405 & 8 - 39 KB & 2017-05-25 - 2017-10-15 \\		
    2018	 & 858  & 8 - 40 KB & 2018-05-15 - 2018-09-20\\
    2019	& 881  & 8 - 39 KB & 2019-05-10 - 2019-10-07\\
    2020	 & 909 & 8 - 39 KB & 2020-05-01 - 2020-09-25\\
    2021	 & 1076 & 8 - 40 KB & 2021-05-01 - 2021-09-29\\
    2022	 & 1077 & 8 - 39 KB & 2022-05-01 - 2022-09-17\\
    2023	 & 1542 & 8 - 39 KB & 2023-05-01 - 2023-11-10\\
    \bottomrule
  \end{tabular}
  \label{tab:tab-cantidadtifs}
\end{table}


\subsubsection{Extracción de datos}
Una vez finalizado el proceso de descarga de los archivos, se llevó a cabo la etapa de preprocesamiento de las 
imágenes satelitales. En esta fase, se utilizaron las bandas espectrales B2, B3, B4, B5, B6, B8 y B11 vistas en 
la seccion \ref{subsec:bandas} para calcular los índices
de vegetación RDVI, GNDVI, LCI, EVI, ARVI, SIPI y NDRE. Estos índices se emplearon como complemento del 
NDVI, con el objetivo de obtener una caracterización más detallada y robusta del estado de la vegetación, 
considerando distintos aspectos de la respuesta espectral de las plantas. De este modo, fue posible incorporar 
información relacionada con el vigor, la estructura del dosel y el contenido de clorofila, aspectos fundamentales
para el monitoreo y análisis de la dinámica vegetativa en el área de estudio.

La figura \ref{fig:flujo-extraccion} muestra el flujo de generación del \textit{dataset}. Comienza con la lectura de las bandas espectrales  de un archivo de entrada 
en formato tiff, que representa una parcela. Estas bandas conforman un cubo de datos 
tridimensional (n x m x z), donde n y m son las dimensiones espaciales y z representa la cantidad de bandas 
espectrales disponibles. Seguidamente se realiza el proceso del cálculo de índices por píxel: por cada archivo de parcela, 
se procede a calcular todos los índices de vegetación, utilizando las reflectancias 
registradas en las distintas bandas para cada uno de los píxeles. Esto genera una matriz de valores numéricos de 
índices. Finalmente, se realiza el cálculo de un estadístico de resumen para toda la parcela. Este valor de resumen es la información que 
se utiliza para rellenar el archivo de salida en formato csv, donde cada fila corresponderá a una parcela y 
contendrá los estadísticos de los índices de vegetación, de este modo, se completa la generación del conjunto de datos.

\begin{figure}[!htbp]
	\centering
	\includegraphics[width=0.8\textwidth]{./Figures/flujo-extracciondatos.png}
	\caption{Flujo de generación de \textit{dataset} con información de satélite\protect\footnotemark.}
	\label{fig:flujo-extraccion}
\end{figure}
\footnotetext{Imagen de elaboración propia.}

\subsubsection{\textit{Dataset} generado}

Aunque el proceso de generación del conjunto de datos genera un archivo en formato csv, la tabla \ref{tab:datasetsatelite} muestra
una representación parcial de la primera versión de los datos en formato tabular para una mejor comprensión de su estructura.

	\begin{table}[h]
		\centering
		\caption{Primera versión de conjunto de datos con información satelital.}
    \footnotesize
		\begin{tabular}{l c c c c c c}    
			\toprule
			\textbf{\#} & \textbf{ID} & \textbf{start-date} & \textbf{end-date} & \textbf{evi-median} & \textbf{\dots} & \textbf{gndvi-median} \\
			\midrule
			0 & clv2 & 2017-05-25 & 2017-05-30 & 0,2 & \dots & 0,2 \\		
			1 & clv3 & 2017-05-25 & 2017-05-30 & 0,5  & \dots & 0,1 \\
			2 & clv5 & 2017-05-25 & 2017-05-30 & 0,4  & \dots & 0,4 \\
      \dots & \dots & \dots & \dots & \dots  & \dots & \dots \\
      6749 & Flam Art & 2023-11-05 & 2023-11-10 & 0,8  & \dots & 0,5 \\
			\bottomrule
		\end{tabular}
		\label{tab:datasetsatelite}
	\end{table}

\subsection{Flujo de procesamiento de datos tabulares} 
En esta sección se describen en detalle los procedimientos aplicados a los datos de floración y cosecha. Se 
aborda la estructura inicial del conjunto de datos, el tratamiento de los valores nulos y el proceso de 
reestructuración de la información para su posterior análisis.

\subsubsection{Estructura de datos}
Inicialmente, se consultaron los datos de floración correspondientes a todas las 
parcelas para un período de estudio de cinco años. Una representación parcial de la 
estructura de dichos datos se observa en la tabla \ref{tab:firstdataset}.

	\begin{table}[h]
		\centering
		\caption{Estructura de datos de floración.}
    \footnotesize
		\begin{tabular}{l c c c c c}    
			\toprule
			\textbf{\#} & \textbf{ID} & \textbf{Dias-floracion-17} & \textbf{Dias-floracion-18} & \textbf{\dots} & \textbf{Coordenadas} \\
			\midrule
			0 & clv2 & 67 & 60 & \dots & [-59.7928,-33.7391][\dots] \\		
			1 & clv3 & NaN & 30 & \dots & [-59.7927,-33.7390][\dots] \\
			2 & clv5 & 73 & 65 & \dots & [-59.7926,-33.7389][\dots] \\
      \dots & \dots & \dots & \dots & \dots & \dots \\
      199 & Flam Art & 67 & NaN & \dots & [-59.7578,-33.7417][\dots]\\
			\bottomrule
		\end{tabular}
		\label{tab:firstdataset}
	\end{table}

Este conjunto de datos corresponde a parcelas agrícolas, donde cada una es identificada por un ID único.  A continuación, se listan aspectos a tener en cuenta:
  \begin{itemize}
    \item La columna de coordenadas contiene los datos necesarios para formar un polígono que define la ubicación y forma de cada parcela.
    \item Cada parcela contiene de uno a tres árboles de la misma variedad. En total, se registraron 200 variedades distintas de durazneros.
    \item El formato de los datos de cosecha sigue la misma estructura que la presentada en la tabla \ref{tab:firstdataset}.
    \item Las fechas de floración y cosecha se registran en formato juliano, indicando el número de días transcurridos desde el 1 de julio de la temporada de cultivo hasta la fecha del evento.
  \end{itemize} 


\subsubsection{Tratamiento de datos faltantes}

Se identificó que los conjuntos de datos de floración y cosecha contienen una cantidad 
significativa de valores faltantes. Específicamente, el conjunto de datos de floración
presenta 150 registros nulos, que equivale al 9.42\% del total de sus datos. En el caso 
del conjunto de datos de cosecha tiene 535 valores faltantes, lo que representa 
un 33.61\% del total. 

En principio, se evaluaron distintos métodos de imputación, analizando en detalle cuál resultaba más
adecuado para este caso. Luego de probar algunas alternativas, se determinó que la interpolación lineal 
era la opción más conveniente debido a la naturaleza de los datos. Este método estima los valores faltantes
calculando una media ponderada a partir de los valores válidos que se encuentran a ambos lados. Para los 
registros nulos ubicados al inicio o al final de una fila, se aplicó un método de relleno hacia adelante (ffill) 
y relleno hacia atrás (bfill), respectivamente. Este enfoque resulta especialmente efectivo en conjuntos 
de datos que presentan una tendencia clara, como en este estudio.

\subsubsection{Reestructuración de datos}

El conjunto de datos inicial de los datos de floración y cosecha presentan un formato ancho donde una única variable 
se distribuye en múltiples columnas, con el año codificado dentro del nombre de cada columna (por ejemplo, dia-floracion-17, 
dia-floracion-18). El paso de transformación realizado, consiste en el proceso de desapilamiento de estas columnas. Esto convierte el formato 
inicial en un formato largo, que es más ordenado para el análisis. Se detallan las nuevas columnas creadas:

\begin{itemize}
    \item Variable año: el valor se extrae del nombre original de la columna.
    \item Variables dia-floracion y dia-cosecha: valores de las variables.
\end{itemize}

En la tabla \ref{tab:largedataset} se muestra el nuevo formato, lo que facilita la posterior unión con el \textit{dataset} de índices de vegetación 
utilizando las claves ID y Año.


	\begin{table}[h]
		\centering
		\caption{Reestructuración de datos de floración y cosecha.}
		\begin{tabular}{l c c c c c}    
			\toprule
			\textbf{\#} & \textbf{ID} & \textbf{año} & \textbf{dia-floracion} & \textbf{\dots} & \textbf{dia-cosecha} \\
			\midrule
			0 & clv2 & 2017 & 60 & \dots & 151 \\		
			1 & clv3 & 2018 & 30 & \dots & 153 \\
			2 & clv5 & 2019 & 65 & \dots & 231 \\
      \dots & \dots & \dots & \dots & \dots & \dots \\
      199 & Flam Art & 2023 & 26 & \dots & 173\\
			\bottomrule
		\end{tabular}
		\label{tab:largedataset}
	\end{table}

\section{Caracterización de montes frutales}

El objetivo de este trabajo fue la búsqueda y recopilación de información satelital para 
vincularlas con características específicas de montes frutales. En la figura \ref{fig:evolucion-indices} se muestra la evolución temporal de distintos índices
 de vegetación calculados a partir de imágenes satelitales para seis parcelas de durazneros 
 (identificadas por su código ID). Cada panel corresponde a un lote específico y representa
la variación de los valores de los índices a lo largo del tiempo durante la campaña 2023. Las curvas de colores indican la evolución de diferentes índices de vegetación, mientras que
la línea vertical roja marca la fecha de floración registrada para cada parcela.
Se observa que, en general, los valores de los índices presentan fluctuaciones a lo largo
del ciclo fenológico, con una disminución marcada en los periodos previos a la floración. 


\begin{figure}[!htbp]
	\centering
	\includegraphics[width=0.9\textwidth]{./Figures/evolucion_indices_2023.png}
	\caption{Evolución de índices de vegetación\protect\footnotemark.}
	\label{fig:evolucion-indices}
\end{figure}
\footnotetext{Imagen de elaboración propia.}


\section{Limitaciones y soluciones propuestas}

El siguiente paso, después del monitoreo de los cultivos, consistió en explorar la información satelital con el objetivo de predecir la fecha de floración.
En una primera instancia, con el conjunto de datos que se muestra en la tabla \ref{tab:datasetsatelite}, se implementaron
varios modelos basados en árboles, entre ellos \textit{Decision Tree}, \textit{Random Forest}, \textit{LGBM},  \textit{Gradient Boosting} 
y \textit{XGBoost} en sus versiones de regresión.
Sin embargo, los resultados obtenidos no alcanzaron el nivel de desempeño esperado. Esto permitió comprobar que, aunque los 
datos correspondientes a determinados meses del año, en un rango temporal de aproximadamente cinco días, son útiles
para analizar el comportamiento de los índices de vegetación en etapas previas a la floración, no resultaron adecuados como
variables predictoras en este contexto.

Acá aparece la idea de aplicar \textit{feature engineering} sobre las variables disponibles, con 
el objetivo de generar descriptores más representativos y relevantes para los modelos.

\subsection{Ingeniería de características}

Para cada unidad experimental, se obtuvieron estadísticos descriptivos a partir de los índices 
de vegetación, considerando un rango temporal de observación acotado hasta la fecha de floración inclusive.
\begin{itemize}
\item Pendiente de crecimiento: calculada mediante una regresión lineal ajustada a los valores del índice de vegetación en el 
período seleccionado. Este parámetro representa la evolución del vigor vegetativo previo a la floración, proporcionando una medida
 cuantitativa de la dinámica de crecimiento del cultivo.
\item Valor máximo: corresponde al punto más alto alcanzado por el índice dentro del mismo rango temporal. Este valor refleja el 
nivel máximo registrado antes de la floración, y resulta útil para identificar el momento de mayor desarrollo del dosel vegetal.
\end{itemize}

\section{Fusión de las fuentes de datos}

La figura \ref{fig:flujo-procesamiento} ilustra el flujo completo de integración de la información. Este flujo operativo
consta de dos fuentes de datos principales:

\begin{itemize}
  \item Datos de cultivo: archivos iniciales en formato excel que contienen los registros de floración y cosecha.
  \item Datos satelitales: archivos obtenidos mediante teledetección en formato tiff.
\end{itemize}

Ambas fuentes pasan por procesos de transformación detallados en las secciones anteriores.
El resultado final de este proceso de integración es la generación de un único archivo consolidado
 que reúne la totalidad de los datos transformados y combinados.


\begin{figure}[!htbp]
	\centering
	\includegraphics[width=0.9\textwidth]{./Figures/flujo-procesamiento.png}
	\caption{Flujo de integración de datos de cultivo y de satélite\protect\footnotemark.}
	\label{fig:flujo-procesamiento}
\end{figure}
\footnotetext{Imagen de elaboración propia.}


\subsection{Nuevo conjunto de datos}
Con el proceso de ingeniería de características, surgen dos nuevas variables (pediente y valor máximo) por cada
índice, la tabla \ref{tab:fulldataset} presenta la estructura del conjunto de datos resultante de la 
integración mostrada anteriormente. Este dataset consolidado reúne en un mismo formato las variables agronómicas junto con 
los indicadores derivados de los índices de vegetación. De esta manera, se dispone de una base de datos
unificada que permite analizar la relación entre las características fenológicas del cultivo y las
condiciones observadas mediante teledetección.


	\begin{table}[h]
		\centering
		\caption{Estructura de archivo con datos de cultivo y satélite.}
    \footnotesize
		\begin{tabular}{l c c c c c c c}    
			\toprule
			\textbf{\#} & \textbf{ID} & \textbf{año} & \textbf{dia-floracion} & \textbf{dia-cosecha} & \textbf{evi-median} & \textbf{\dots} & \textbf{pico-evi} \\
			\midrule
			0 & clv2 & 2017 & 60 & 151 & 0,3 & \dots & 0,2 \\		
			1 & clv3 & 2018 & 30 & 153 & 0,4 & \dots & 0,1 \\
			2 & clv5 & 2019 & 65 & 231 & 0,2 & \dots & 0,4 \\
      \dots & \dots & \dots & \dots & \dots & \dots & \dots & \dots \\
      6749 & Flam Art & 2023 & 26 & 173 & 0,5 & \dots & 0,5 \\
			\bottomrule
		\end{tabular}
		\label{tab:fulldataset}
	\end{table}


\section{Arquitectura de inteligencia artificial}

Una vez consolidada la totalidad de los datos provenientes de las diferentes fuentes, se implementa una cadena de procesamiento
que puede dividirse en dos etapas principales. Por un lado, se desarrolla el monitoreo previo a la floración, que consiste 
en el análisis temporal de los índices de vegetación con el objetivo de caracterizar el estado fenológico de los cultivos 
antes del evento de floración. Por otro lado, se aplican de modelos de aprendizaje automático, 
orientado a la predicción de la fecha de floración a partir de las variables derivadas del monitoreo satelital y 
de los registros agronómicos.

En la figura \ref{fig:pipeline-procesamiento} se presenta el flujo general del proceso, que 
inicia con la entrada de los datos integrados y continúa con la caracterización de los cultivos
mediante mapas de índices de vegetación. Para el análisis previo a la floración se consideran 
únicamente los registros del período anterior al evento. 

\begin{figure}[!htbp]
	\centering
	\includegraphics[width=0.8\textwidth]{./Figures/pipeline-procesamiento.png}
	\caption{Arquitectura de inteligencia artificial propuesta\protect\footnotemark.}
	\label{fig:pipeline-procesamiento}
\end{figure}
\footnotetext{Imagen de elaboración propia.}
