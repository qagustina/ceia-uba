% Chapter Template

\chapter{Conclusiones} % Main chapter title

\label{Chapter5} % Change X to a consecutive number; for referencing this chapter elsewhere, use \ref{ChapterX}
En este capítulo se presentan las conclusiones, los aportes más relevantes del trabajo y las posibles líneas de mejora a futuro.
%----------------------------------------------------------------------------------------

%----------------------------------------------------------------------------------------
%	SECTION 1
%----------------------------------------------------------------------------------------

\section{Resultados obtenidos}
A continuación, se resumen las principales actividades realizadas, junto con la forma en que se abordaron 
los problemas que surgieron durante el desarrollo y los logros obtenidos. 

La planificación se cumplió en cuanto a las actividades, aunque el tiempo de dedicación fue mayor al 
previsto inicialmente.

Respecto a la gestión de riesgos, se presentaron distintas situaciones:

\begin{itemize}
    \item Imágenes aéreas sin datos de bandas. Este es un problema que se presentó y no estaba
    contemplado. Se optó por excluir de la descarga de imágenes aquellas que no tengan
    las bandas requeridas para el cálculo de los índices.
    \item Suficiencia de datos. Este aspecto representó uno de los principales riesgos del proyecto.
     Si bien se logró realizar con éxito la descarga de imágenes satelitales correspondientes a 
     años históricos, se detectaron limitaciones específicas en el año 2017. En particular, 
     durante los meses previos a la floración del duraznero, la colección de datos seleccionada
     presentó amplios intervalos de fechas con imágenes incompletas o carentes de información 
     en varias bandas espectrales.
    \item Bajos recursos computacionales. Este aspecto se identificaba inicialmente como un posible
     riesgo, sin embargo, no representó una limitación en la práctica. Tanto la ejecución del 
     script para la obtención de imágenes satelitales como el entrenamiento de los modelos pudieron
      realizarse satisfactoriamente en una computadora personal.
\end{itemize}

Respecto a los requerimientos:

\begin{itemize}
    \item Se cumplieron de manera apropiada los funcionales, de manejo de datos, herramientas de
     codigo y documentación. 
    \item Caracterización de montes frutales. Esta etapa del trabajo pudo desarrollarse sin 
    inconvenientes. Si bien se había identificado como posible riesgo la incapacidad del 
    algoritmo para reconocer correctamente las etapas de floración, el problema no se presentó.  
    \item La predicción de la fecha de floración no formaba parte de los requerimientos iniciales 
    del proyecto, sin embargo, se realizó de manera satisfactoria como un aporte adicional. 
    Durante las primeras pruebas se evidenció que los datos satelitales disponibles no eran 
    suficientes para obtener resultados precisos, y esto se abordó mediante la ingeniería de
    características. En la etapa de evaluación de los modelos, la métrica de bondad de ajuste  
    arrojó un valor inferior al 50\%, lo que indica una capacidad moderada para explicar la 
    variabilidad de los datos. El equipo del cliente manifestó que, en este estudio, la métrica
    RMSE tiene mayor relevancia al momento de valorar el desempeño de los modelos, dado que 
    refleja de manera más directa el error promedio en días de predicción. En la sección \ref{sec:nextsteps}
    se presentan propuestas de mejora.
\end{itemize}

Entre los contenidos incorporados a lo largo de la especialización, algunos resultaron fundamentales
para la ejecución y el enfoque de este trabajo, especialmente los siguientes:

\begin{itemize}
    \item Análisis exploratorio de datos. Influyó directamente en la toma de decisiones respecto a las
     metodologías y herramientas aplicadas, además de aportar criterios sólidos para la interpretación
     y validación de los resultados.
    \item Aprendizaje de máquina. El aprendizaje sobre preparación y split  de datos, así como sobre métricas
     de evaluación, resultó de gran utilidad para el desarrollo del trabajo. De los algoritmos abordados
      durante la especialización, se implementaron modelos basados en árboles de decisión.

\end{itemize}
%----------------------------------------------------------------------------------------
%	SECTION 2
%----------------------------------------------------------------------------------------
\section{Próximos pasos}
\label{sec:nextsteps}
En una primera etapa de este estudio se descargaron imágenes satelitales correspondientes al
periodo de floración de los montes frutales, con el propósito de analizar el comportamiento
de los índices de vegetación durante ese momento fenológico. Sin embargo, no fueron 
utilizadas en el desarrollo de este trabajo, ya que el enfoque se centró en el análisis
previo a la floración. Estos datos representan un recurso valioso para futuros estudios,
ya que podrían emplearse para mejorar los modelos actuales.

En cuanto a líneas de trabajo futuras, sería de interés profundizar en la evaluación
temporal de los modelos mediante técnicas de validación más avanzadas. Entre ellas,
se propone la aplicación de \textit{rolling window}, que permitiría analizar el
rendimiento del modelo a lo largo del tiempo, considerando distintos periodos
de entrenamiento y prueba. Asi como también la implementación de un esquema de
validación cruzada para series temporales, contribuiría a obtener una estimación
más robusta del desempeño predictivo.

