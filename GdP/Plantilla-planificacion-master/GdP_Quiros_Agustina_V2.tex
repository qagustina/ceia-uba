\documentclass[
11pt, % The default document font size, options: 10pt, 11pt, 12pt
%codirector, % Uncomment to add a codirector to the title page
]{charter} 


% El títulos de la memoria, se usa en la carátula y se puede usar el cualquier lugar del documento con el comando \ttitle
\titulo{Caracterización de montes frutales a través del análisis de imágenes satelitales} 

% Nombre del posgrado, se usa en la carátula y se puede usar el cualquier lugar del documento con el comando \degreename
\posgrado{Carrera de Especialización en Inteligencia Artificial}
% IMPORTANTE: no omitir titulaciones ni tildación en los nombres, también se recomienda escribir los nombres completos (tal cual los tienen en su documento)
% Tu nombre, se puede usar el cualquier lugar del documento con el comando \authorname
\autor{Lic. Agustina Quiros}

% El nombre del director y co-director, se puede usar el cualquier lugar del documento con el comando \supname y \cosupname y \pertesupname y \pertecosupname
\director{Esp. Lic. Maria Carina Roldán}
\pertenenciaDirector{FIUBA}
\codirector{} % para que aparezca en la portada se debe descomentar la opción codirector en los parámetros de documentclass
\pertenenciaCoDirector{}

% Nombre del cliente, quien va a aprobar los resultados del proyecto, se puede usar con el comando \clientename y \empclientename
\cliente{Dr. Lic. Gerardo Sánchez}
\empresaCliente{Instituto Nacional de Tecnología Agropecuaria (INTA)}
\fechaINICIO{20 de agosto de 2024}		%Fecha de inicio de la cursada de GdP \fechaInicioName
\fechaFINALPlan{8 de octubre de 2024} 	%Fecha de final de cursada de GdP
\fechaFINALTrabajo{mes de junio de 2025}	%Fecha de defensa pública del trabajo final


\begin{document}

\maketitle
\thispagestyle{empty}
\pagebreak


\thispagestyle{empty}
{\setlength{\parskip}{0pt}
\tableofcontents{}
}
\pagebreak


\section*{Registros de cambios}
\label{sec:registro}


\begin{table}[ht]
\label{tab:registro}
\centering
\begin{tabularx}{\linewidth}{@{}|c|X|c|@{}}
\hline
\rowcolor[HTML]{C0C0C0} 
Revisión & \multicolumn{1}{c|}{\cellcolor[HTML]{C0C0C0}Detalles de los cambios realizados} & Fecha      \\ \hline
0      & Creación del documento                                 &\fechaInicioName \\ \hline
1      & Se completa hasta el punto 5 inclusive                & {05}/{09}/2024 \\ \hline
2      & Se completa hasta el punto 9 inclusive                & {10}/{09}/2024 \\ \hline
%		  Se puede agregar algo más \newline
%		  En distintas líneas \newline
%		  Así                                                    & {día} de {mes} de 202X \\ \hline
%3      & Se completa hasta el punto 12 inclusive                & {día} de {mes} de 202X \\ \hline
%4      & Se completa el plan	                                 & {día} de {mes} de 202X \\ \hline

% Si hay más correcciones pasada la versión 4 también se deben especificar acá

\end{tabularx}
\end{table}

\pagebreak



\section*{Acta de constitución del proyecto}
\label{sec:acta}

\begin{flushright}
Buenos Aires, \fechaInicioName
\end{flushright}

\vspace{2cm}

Por medio de la presente se acuerda con la \authorname\hspace{1px} que su Trabajo Final de la \degreename\hspace{1px} se 
titulará ``\ttitle'', consistirá esencialmente en el desarrollo de un algoritmo que permita descargar y analizar imágenes satelitales para 
determinar el progreso de las etapas de floración y maduración de los frutos en el árbol, y tendrá un presupuesto preliminar estimado de {TBD} hs de trabajo y un costo estimado de
{\$ TBD}, con fecha de inicio el \fechaInicioName\hspace{1px} y fecha de presentación pública 
el \fechaFinalName.

Se adjunta a esta acta la planificación inicial.

\vfill

% Esta parte se construye sola con la información que hayan cargado en el preámbulo del documento y no debe modificarla
\begin{table}[ht]
\centering
\begin{tabular}{ccc}
\begin{tabular}[c]{@{}c@{}}Dr. Ing. Ariel Lutenberg \\ Director posgrado FIUBA\end{tabular} & \hspace{2cm} & \begin{tabular}[c]{@{}c@{}}\clientename \\ \empclientename \end{tabular} \vspace{2.5cm} \\ 
\multicolumn{3}{c}{\begin{tabular}[c]{@{}c@{}} \supname \\ Director del Trabajo Final\end{tabular}} \vspace{2.5cm} \\
\end{tabular}
\end{table}




\section{1. Descripción técnica-conceptual del proyecto a realizar}
\label{sec:descripcion}
El perfil de trabajo del Laboratorio de Biotecnología, Estación Experimental Agropecuaria (EEA) San Pedro, perteneciente al Instituto 
Nacional de Tecnología Agropecuaria (INTA), se centra en la intensificación sustentable de las producciones de frutales, hortalizas y viveros. 

En la actualidad, la gestión eficiente de los montes frutales enfrenta varios desafíos significativos, entre los que se incluye la necesidad 
de monitoreo constante para asegurar la salud y productividad de los cultivos. Los métodos tradicionales de inspección y evaluación, 
que dependen en gran medida de visitas de campo y observaciones manuales, son laboriosos, costosos y a menudo limitados en alcance y frecuencia. 
Estos métodos también pueden ser subjetivos, dependiendo de la experiencia y percepción del observador, lo que puede llevar a inconsistencias 
en la evaluación. Además, factores como el cambio climático y la variabilidad en las condiciones meteorológicas añaden capas adicionales de 
complejidad a la gestión agronómica, ya que afectan directamente la salud y el rendimiento de los cultivos.

En este contexto, las imágenes satelitales emergen como una herramienta poderosa para abordar estos problemas. Estas imágenes proporcionan datos 
de observación de la Tierra que son consistentes, repetitivos y de amplia cobertura, algo esencial para un monitoreo efectivo. 
A través de imágenes satelitales, se pueden obtener índices de vegetación, como el NDVI (Índice de Vegetación de Diferencia Normalizada), 
que permiten evaluar la vegetación de manera precisa y objetiva. Específicamente, expresa el vigor vegetativo de la planta, por lo que a mayor valor de éste, denota que la planta es fotosintéticamente 
activa y su morfología y estructura interna no sufre problemas significativos. Estos índices son cruciales para detectar cambios en la vegetación mucho antes 
de que sean visibles a simple vista. 

En la figura \ref{fig:ndvi} se presenta la interpretación de lo descrito.

\begin{figure}[htpb]
\centering 
\includegraphics[width=.65\textwidth]{./Figuras/ndvi.png}
\caption{Ejemplificación de interpretación colores imágenes NDVI.}
\label{fig:ndvi}
\end{figure}

\vspace{25px}

En el marco de este proyecto, se hará uso de imágenes satelitales para verificar si es posible identificar índices de utilidad
que permitan caracterizar la evolución de los árboles disponibles dentro del monte frutal. 

% diagrama TBD 
\pagebreak

\section{2. Identificación y análisis de los interesados}
\label{sec:interesados}


\begin{table}[ht]
%\caption{Identificación de los interesados}
%\label{tab:interesados}
\begin{tabularx}{\linewidth}{@{}|l|X|X|p{4cm}|@{}}
\hline
\rowcolor[HTML]{C0C0C0} 
Rol           & Nombre y Apellido & Organización 	& Puesto 	\\ \hline
Auspiciante   & \clientename      & \empclientename & Director del laboratorio de biotecnología EEA San Pedro\\ \hline
Cliente       & \clientename      & \empclientename	& Director del laboratorio de biotecnología EEA San Pedro  \\ \hline
Responsable   & \authorname       & FIUBA        	& Alumno 	\\ \hline
Colaboradores & Dr. Lic. Maximiliano Aballay& INTA-Conicet& Becario doctoral \\ \hline
Orientador    & \supname	      & \pertesupname 	& Director del Trabajo Final \\ \hline
Usuario final & Equipo del laboratorio de biotecnología de la EEA San Pedro& INTA& - \\ \hline
\end{tabularx}
\end{table}
 

%\begin{itemize}
%	\item Orientador: TBD
%\end{itemize}



\section{3. Propósito del proyecto}
\label{sec:proposito}

El objetivo principal del proyecto es el de automatizar los procesos de análisis de imágenes satelitales para caracterizar
la evolución de los árboles, específicamente el progreso de las etapas de floración y maduración de los frutos.  


\section{4. Alcance del proyecto}
\label{sec:alcance}

El proyecto incluye:
\begin{itemize}
	\item La evaluación de las diferentes fuentes de datos y determinar cuál es la opción más viable para la realización de este trabajo.
	\item La determinación del progreso de las etapas de floración y maduración de los frutos en el árbol.
	\item El desarrollo de una herramienta de fácil acceso para el equipo de INTA.
	\item La elaboración de un informe que detalle el procedimiento realizado y resultados.
\end{itemize}

Los siguientes elementos quedan fuera del alcance:
\begin{itemize}
	\item El desarrollo de una interfaz para el sistema. 	
	\item El despliegue de la solución en producción.
\end{itemize}


\section{5. Supuestos del proyecto}
\label{sec:supuestos}

Para el desarrollo del presente proyecto se supone que: 

\begin{itemize}
	\item Se contará con imágenes satelitales y el dataset con fechas de floración y cosecha 
	para todas las campañas que se encuentren disponibles.	
	\item Se tendrá acceso a publicaciones y datos climáticos para completar el proyecto.
	\item Se contará con el apoyo necesario del cliente cuando se requieran conocimientos específicos
	relacionados con el conjunto de imágenes y sus procesos.
	\item Todo el desarrollo del trabajo se podrá realizar con herramientas de código abierto.
	\item Se contará con la asesoría de expertos en frutales que puedan colaborar en la revisión
    de los resultados y la validación del enfoque del proyecto.
\end{itemize}



\section{6. Requerimientos}
\label{sec:requerimientos}

\begin{enumerate}
	\item Requerimientos funcionales:
		\begin{enumerate}
			\item El modelo implementado deberá identificar el progreso de las etapas de floración y maduración de los frutos en el árbol.
			\item La técnica implementada dará un informe de el nivel de confianza de los índices obtenidos.
		\end{enumerate}
	\item Requerimientos de documentación:
		\begin{enumerate}
			\item Se entregará un informe de avance y una memoria final del proyecto.
			\item El informe de avance será claro y conciso para su correcta interpretación.
			\item El código desarrollado tendrá los comentarios y será prolijo.
		\end{enumerate}
	\item Requerimiento sobre datos disponibles:
	\begin{enumerate}
        \item Por cuestiones de confidencialidad, el conjunto datos no será difundido y solo se utilizará
        en el contexto del trabajo final.
    \end{enumerate}
    \item Requerimientos de experimentos:
	\begin{enumerate}
        \item Se deberá desarrollar y probar al menos dos modelos distintos de aprendizaje profundo.
    \end{enumerate}
\end{enumerate}



\section{7. Historias de usuarios (\textit{Product backlog})}
\label{sec:backlog}
Se identifican los siguientes roles:
\begin{itemize}
    \item Director del laboratorio de biotecnología: es quien busca automatizar el proceso de monitoreo de
    montes frutales para hacerlo más eficiente.
    \item Experto de laboratorio: es quien está a cargo de la inspección y evaluación de los montes frutales.
\end{itemize}

Para este proyecto, se utilizará la siguiente escala de \textit{story points} basada en la serie de Fibonacci.
Para el cálculo de los puntos de historia, se realiza una evaluación para cada uno de los ejes
esfuerzo, complejidad e incertidumbre. Para cada uno los valores pueden ser 1 (Bajo), 3 (Medio) y 5 (Alto).
A continuación, se suman los 3 valores y se aproxima el resultado al valor más cercano en la escala elegida.

Se detallan las historias de usuarios:
\begin{itemize}
    \item Como director del laboratorio de biotecnología, quiero tener a disposición de manera detallada las técnicas y modelos de aprendizaje profundo
          que se utilizaron para la indentificación de características, para realizar a futuro el despliegue en producción.
        \begin{itemize}
            \item Esfuerzo: 3
            \item Complejidad: 1
            \item Incertidumbre: 1 
            \item Ponderación total: 5 \textit{story points} 
        \end{itemize}
    \item Como director del laboratorio de biotecnología, quiero saber la calidad del conjunto de datos para llevar 
        a cabo esta tarea, para considerar futuras mejoras.
        \begin{itemize}
            \item Esfuerzo: 3 
            \item Complejidad: 3
            \item Incertidumbre: 1
            \item Ponderación total: 8 \textit{story points} 
        \end{itemize}
    \item Como experto de laboratorio, quiero tener acceso a la confianza de acierto
    de cada característica que el proceso indentificó automaticamente, para saber si es necesario hacer revisiones manuales.
        \begin{itemize}
            \item Esfuerzo: 3 
            \item Complejidad: 5 
            \item Incertidumbre: 3
            \item Ponderación total: 13 \textit{story points} 
        \end{itemize}
    \item Como experto de laboratorio, quiero poder identificar automáticamente el progreso de las etapas
     de los frutos en los árboles, para determinar los próximos pasos a seguir.
        \begin{itemize}
            \item Esfuerzo: 5 
            \item Complejidad: 5 
            \item Incertidumbre: 5 
            \item Ponderación total: 21 \textit{story points} 
        \end{itemize}
\end{itemize}

\section{8. Entregables principales del proyecto}
\label{sec:entregables}

Los entregables del proyecto son:

\begin{itemize}
	\item Plan del proyecto.
	\item Código fuente documentado.
	\item Informe del avance.
	\item Informe para el cliente con resumen de etapas de progreso de los frutos del árbol, técnicas utilizadas y resultados.
	\item Memoria técnica del proyecto.
\end{itemize}


\section{9. Desglose del trabajo en tareas}
\label{sec:wbs}


\begin{enumerate}
\item Preparación de los datos. (105 h)
	\begin{enumerate}
	\item Construccón de análisis exploratorio. (20 h)
	\item Identificar los índices de vegetación. (15 h)
	\item Entender y elegir las arquitecturas de aprendizaje profundo a ser utilizadas. (20 h)
	\item Realizar la vinculación de características a los frutos del árbol. (40 h)
	\end{enumerate}
\item Desarrollo de modelos. (130 h)
	\begin{enumerate}
	\item Desarrollar código, pruebas, y corregir errores. (40 h)
	\item Documentar código. (20 h)
	\item Ajustar parámetros e iterar. (40 h)
	\item Comparar resultados y documentar su explicación. (40 h)
	\end{enumerate}
\item Diseño de experimentos. (140 h)
	\begin{enumerate}
	\item Investigar material bibliográfico. (40 h)
	\item Definir experimentos. (40 h)
	\item Determinar las técnicas de aprendizaje profundo a utilizar. (40 h)
	\item Identificar métricas de evaluación de los modelos. (20 h)
	\end{enumerate}
\item Evaluación de conjunto de datos. (60 h)
\begin{enumerate}
	\item Evaluar los distintos datasets propuestos por el cliente para identificar el más
    representativo para el desarrollo de este proyecto. (60 h)
	\end{enumerate}
\item Elaboración de documentos. (80 h)
	\begin{enumerate}
	\item Elaborar informe del avance del proyecto. (20 h)
	\item Redactar memoria del proyecto. (30 h)
	\item Preparación de presentación pública y defensa del trabajo final. (30 h) 
	\end{enumerate}
\item Planificación del proyecto. (90 h)
	\begin{enumerate}
	\item Programación de reuniones con el grupo del cliente para seguimiento del proceso
    y consultar información. (40 h)
	\item Elaborar documento de planificación. (40 h)
	\item Preparar la presentación del plan de trabajo. (10 h)
	\end{enumerate}  
\end{enumerate}

Cantidad total de horas: 605.


\section{10. Diagrama de Activity On Node}
\label{sec:AoN}

\begin{consigna}{red}
Armar el AoN a partir del WBS definido en la etapa anterior.

Una herramienta simple para desarrollar los diagramas es el Draw.io (\url{https://app.diagrams.net/}).
\href{https://app.diagrams.net}{Draw.io}


\begin{figure}[htpb]
\centering 
\includegraphics[width=.8\textwidth]{./Figuras/AoN.png}
\caption{Diagrama de \textit{Activity on Node}.}
\label{fig:AoN}
\end{figure}

Indicar claramente en qué unidades están expresados los tiempos.
De ser necesario indicar los caminos semi críticos y analizar sus tiempos mediante un cuadro.
Es recomendable usar colores y un cuadro indicativo describiendo qué representa cada color.

\end{consigna}

\section{11. Diagrama de Gantt}
\label{sec:gantt}

\begin{consigna}{red}
Existen muchos programas y recursos \textit{online} para hacer diagramas de Gantt, entre los cuales destacamos:

\begin{itemize}
\item Planner
\item GanttProject
\item Trello + \textit{plugins}. En el siguiente link hay un tutorial oficial: \\ \url{https://blog.trello.com/es/diagrama-de-gantt-de-un-proyecto}
\item Creately, herramienta online colaborativa. \\\url{https://creately.com/diagram/example/ieb3p3ml/LaTeX}
\item Se puede hacer en latex con el paquete \textit{pgfgantt}\\ \url{http://ctan.dcc.uchile.cl/graphics/pgf/contrib/pgfgantt/pgfgantt.pdf}
\end{itemize}

Pegar acá una captura de pantalla del diagrama de Gantt, cuidando que la letra sea suficientemente grande como para ser legible. 
Si el diagrama queda demasiado ancho, se puede pegar primero la ``tabla'' del Gantt y luego pegar la parte del diagrama de barras del diagrama de Gantt.

Configurar el software para que en la parte de la tabla muestre los códigos del EDT (WBS).\\
Configurar el software para que al lado de cada barra muestre el nombre de cada tarea.\\
Revisar que la fecha de finalización coincida con lo indicado en el Acta Constitutiva.

En la figura \ref{fig:gantt}, se muestra un ejemplo de diagrama de gantt realizado con el paquete de \textit{pgfgantt}. 
En la plantilla pueden ver el código que lo genera y usarlo de base para construir el propio.

Las fechas pueden ser calculadas utilizando alguna de las herramientas antes citadas. Sin embargo, el siguiente ejemplo
fue elaborado utilizando 
\href{https://docs.google.com/spreadsheets/d/1fBz8NhSpc4tkkhz3KjJCbh1nR_ltDkfEcZi4tZXduqs}{esta hoja de cálculo}.

Es importante destacar que el ancho del diagrama estará dado por la longitud del texto utilizado para las tareas 
(Ejemplo: tarea 1, tarea 2, etcétera) y el valor \textit{x unit}. Para mejorar la apariencia del diagrama, es necesario
ajustar este valor y, quizás, acortar los nombres de las tareas.

\begin{figure}[htpb]
  \begin{center}
    \begin{ganttchart}[
      time slot unit=day,
      time slot format=isodate,
      x unit=0.038cm,
      y unit title=0.7cm,
      y unit chart=0.6cm,
      milestone/.append style={xscale=4}
      ]{2021-03-05}{2021-12-16}
      \gantttitlecalendar*{2021-03-05}{2021-12-16}{year} \\
      \gantttitlecalendar*{2021-03-05}{2021-12-16}{month} \\
      \ganttgroup{Duración Total}{2021-03-05}{2021-12-16} \\
      %%%%%%%%%%%%%%%%%Organización
      \ganttgroup{Organización}{2021-03-05}{2021-04-16} \\
      \ganttbar{Planificación del proyecto}{2021-03-05}{2021-04-15} \\
      %%%%%%%%%%%%%%%%%Ejecución
      \ganttgroup{Ejecución}{2021-04-16}{2021-10-21} \\
      \ganttbar{Tarea 1}{2021-04-16}{2021-04-29} \\
      \ganttbar{Tarea 2}{2021-04-30}{2021-05-13} \\
      \ganttbar{Tarea 3}{2021-05-14}{2021-05-27} \\
      \ganttbar{Tarea 4}{2021-05-28}{2021-07-12} \\
      \ganttbar{Tarea 5}{2021-07-13}{2021-08-09} \\
      \ganttbar{Tarea 6}{2021-08-10}{2021-09-23} \\
      \ganttbar{Tarea 7}{2021-09-24}{2021-09-30} \\
      \ganttbar{Tarea 8}{2021-10-01}{2021-10-14} \\
      \ganttbar{Tarea 9}{2021-10-15}{2021-10-21} \\
      % %%%%%%%%%%%%%%%%%Finalización
      \ganttgroup{Finalización}{2021-10-22}{2021-12-16} \\
      \ganttbar{Memoria v1}{2021-10-22}{2021-11-04} \\
      \ganttbar{Memoria v2}{2021-11-05}{2021-11-18} \\
      \ganttbar{Memoria final}{2021-11-19}{2021-12-02} \\
      % La fecha del siguiente milestone es la fecha en que terminamos la memoria
      \ganttmilestone{Enviar memoria al director}{2021-12-02} \\
      \ganttbar{Elaborar la presentación}{2021-12-03}{2021-12-16} \\
      \ganttmilestone{Ensayo de la presentación}{2021-12-16} \\
      %%%%%%%%%%%%%%%%%%%%%%%%%%%%%%%%%%%%%%%%%%%%%%%%%%%%%%%%%%%%%%%
    \end{ganttchart}
  \end{center}
  \caption{Diagrama de gantt de ejemplo}
  \label{fig:gantt}
\end{figure}


\begin{landscape}
\begin{figure}[htpb]
\centering 
\includegraphics[height=.85\textheight]{./Figuras/Gantt-2.png}
\caption{Ejemplo de diagrama de Gantt (apaisado).} %Modificar este título acorde.
\label{fig:diagGantt}
\end{figure}

\end{landscape}

\end{consigna}


\section{12. Presupuesto detallado del proyecto}
\label{sec:presupuesto}

\begin{consigna}{red}
Si el proyecto es complejo entonces separarlo en partes:
\begin{itemize}
	\item Un total global, indicando el subtotal acumulado por cada una de las áreas.
	\item El desglose detallado del subtotal de cada una de las áreas.
\end{itemize}

IMPORTANTE: No olvidarse de considerar los COSTOS INDIRECTOS.

Incluir la aclaración de si se emplea como moneda el peso argentino (ARS) o si se usa moneda extranjera (USD, EUR, etc). Si es en moneda extranjera se debe indicar la tasa de conversión respecto a la moneda local en una fecha dada.

\end{consigna}

\begin{table}[htpb]
\centering
\begin{tabularx}{\linewidth}{@{}|X|c|r|r|@{}}
\hline
\rowcolor[HTML]{C0C0C0} 
\multicolumn{4}{|c|}{\cellcolor[HTML]{C0C0C0}COSTOS DIRECTOS} \\ \hline
\rowcolor[HTML]{C0C0C0} 
Descripción &
  \multicolumn{1}{c|}{\cellcolor[HTML]{C0C0C0}Cantidad} &
  \multicolumn{1}{c|}{\cellcolor[HTML]{C0C0C0}Valor unitario} &
  \multicolumn{1}{c|}{\cellcolor[HTML]{C0C0C0}Valor total} \\ \hline
 &
  \multicolumn{1}{c|}{} &
  \multicolumn{1}{c|}{} &
  \multicolumn{1}{c|}{} \\ \hline
 &
  \multicolumn{1}{c|}{} &
  \multicolumn{1}{c|}{} &
  \multicolumn{1}{c|}{} \\ \hline
\multicolumn{1}{|l|}{} &
   &
   &
   \\ \hline
\multicolumn{1}{|l|}{} &
   &
   &
   \\ \hline
\multicolumn{3}{|c|}{SUBTOTAL} &
  \multicolumn{1}{c|}{} \\ \hline
\rowcolor[HTML]{C0C0C0} 
\multicolumn{4}{|c|}{\cellcolor[HTML]{C0C0C0}COSTOS INDIRECTOS} \\ \hline
\rowcolor[HTML]{C0C0C0} 
Descripción &
  \multicolumn{1}{c|}{\cellcolor[HTML]{C0C0C0}Cantidad} &
  \multicolumn{1}{c|}{\cellcolor[HTML]{C0C0C0}Valor unitario} &
  \multicolumn{1}{c|}{\cellcolor[HTML]{C0C0C0}Valor total} \\ \hline
\multicolumn{1}{|l|}{} &
   &
   &
   \\ \hline
\multicolumn{1}{|l|}{} &
   &
   &
   \\ \hline
\multicolumn{1}{|l|}{} &
   &
   &
   \\ \hline
\multicolumn{3}{|c|}{SUBTOTAL} &
  \multicolumn{1}{c|}{} \\ \hline
\rowcolor[HTML]{C0C0C0}
\multicolumn{3}{|c|}{TOTAL} &
   \\ \hline
\end{tabularx}%
\end{table}


\section{13. Gestión de riesgos}
\label{sec:riesgos}

\begin{consigna}{red}
a) Identificación de los riesgos (al menos cinco) y estimación de sus consecuencias:
 
Riesgo 1: detallar el riesgo (riesgo es algo que si ocurre altera los planes previstos de forma negativa)
\begin{itemize}
	\item Severidad (S): mientras más severo, más alto es el número (usar números del 1 al 10).\\
	Justificar el motivo por el cual se asigna determinado número de severidad (S).
	\item Probabilidad de ocurrencia (O): mientras más probable, más alto es el número (usar del 1 al 10).\\
	Justificar el motivo por el cual se asigna determinado número de (O). 
\end{itemize}   

Riesgo 2:
\begin{itemize}
	\item Severidad (S): X.\\
	Justificación...
	\item Ocurrencia (O): Y.\\
	Justificación...
\end{itemize}

Riesgo 3:
\begin{itemize}
	\item Severidad (S):  X.\\
	Justificación...
	\item Ocurrencia (O): Y.\\
	Justificación...
\end{itemize}


b) Tabla de gestión de riesgos:      (El RPN se calcula como RPN=SxO)

\begin{table}[htpb]
\centering
\begin{tabularx}{\linewidth}{@{}|X|c|c|c|c|c|c|@{}}
\hline
\rowcolor[HTML]{C0C0C0} 
Riesgo & S & O & RPN & S* & O* & RPN* \\ \hline
       &   &   &     &    &    &      \\ \hline
       &   &   &     &    &    &      \\ \hline
       &   &   &     &    &    &      \\ \hline
       &   &   &     &    &    &      \\ \hline
       &   &   &     &    &    &      \\ \hline
\end{tabularx}%
\end{table}

Criterio adoptado: 

Se tomarán medidas de mitigación en los riesgos cuyos números de RPN sean mayores a...

Nota: los valores marcados con (*) en la tabla corresponden luego de haber aplicado la mitigación.

c) Plan de mitigación de los riesgos que originalmente excedían el RPN máximo establecido:
 
Riesgo 1: plan de mitigación (si por el RPN fuera necesario elaborar un plan de mitigación).
  Nueva asignación de S y O, con su respectiva justificación:
  \begin{itemize}
	\item Severidad (S*): mientras más severo, más alto es el número (usar números del 1 al 10).
          Justificar el motivo por el cual se asigna determinado número de severidad (S).
	\item Probabilidad de ocurrencia (O*): mientras más probable, más alto es el número (usar del 1 al 10).
          Justificar el motivo por el cual se asigna determinado número de (O).
	\end{itemize}

Riesgo 2: plan de mitigación (si por el RPN fuera necesario elaborar un plan de mitigación).
 
Riesgo 3: plan de mitigación (si por el RPN fuera necesario elaborar un plan de mitigación).

\end{consigna}


\section{14. Gestión de la calidad}
\label{sec:calidad}

\begin{consigna}{red}
Elija al menos diez requerimientos que a su criterio sean los más importantes/críticos/que aportan más valor y para cada uno de ellos indique las acciones de verificación y validación que permitan asegurar su cumplimiento.

\begin{itemize} 
\item Req \#1: copiar acá el requerimiento con su correspondiente número.

\begin{itemize}
	\item Verificación para confirmar si se cumplió con lo requerido antes de mostrar el sistema al cliente. Detallar.
	\item Validación con el cliente para confirmar que está de acuerdo en que se cumplió con lo requerido. Detallar. 
\end{itemize}

\end{itemize}

Tener en cuenta que en este contexto se pueden mencionar simulaciones, cálculos, revisión de hojas de datos, consulta con expertos, mediciones, etc.  

Las acciones de verificación suelen considerar al entregable como ``caja blanca'', es decir se conoce en profundidad su funcionamiento interno.  

En cambio, las acciones de validación suelen considerar al entregable como ``caja negra'', es decir, que no se conocen los detalles de su funcionamiento interno.

\end{consigna}

\section{15. Procesos de cierre}    
\label{sec:cierre}

\begin{consigna}{red}
Establecer las pautas de trabajo para realizar una reunión final de evaluación del proyecto, tal que contemple las siguientes actividades:

\begin{itemize}
	\item Pautas de trabajo que se seguirán para analizar si se respetó el Plan de Proyecto original:\\
	 - Indicar quién se ocupará de hacer esto y cuál será el procedimiento a aplicar. 
	\item Identificación de las técnicas y procedimientos útiles e inútiles que se emplearon, los problemas que surgieron y cómo se solucionaron:\\
	 - Indicar quién se ocupará de hacer esto y cuál será el procedimiento para dejar registro.
	\item Indicar quién organizará el acto de agradecimiento a todos los interesados, y en especial al equipo de trabajo y colaboradores:\\
	  - Indicar esto y quién financiará los gastos correspondientes.
\end{itemize}

\end{consigna}

\end{document}