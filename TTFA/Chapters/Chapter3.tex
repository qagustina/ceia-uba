\chapter{Diseño e implementación} % Main chapter title

\label{Chapter3} % Change X to a consecutive number; for referencing this chapter elsewhere, use \ref{ChapterX}

En este capítulo se describe la problemática planteada, seguidamente se describe el enfoque 
utilizado para resolver el problema, detallando las estrategias adoptadas. Luego, se expone 
un análisis exploratorio inicial de los datos y el proceso de descarga de imágenes satelitales. 
Finalmente, se presentan los modelos de aprendizaje automático utilizados, junto con el proceso 
de desarrollo de los scripts para su implementación.
\definecolor{mygreen}{rgb}{0,0.6,0}
\definecolor{mygray}{rgb}{0.5,0.5,0.5}
\definecolor{mymauve}{rgb}{0.58,0,0.82}

%%%%%%%%%%%%%%%%%%%%%%%%%%%%%%%%%%%%%%%%%%%%%%%%%%%%%%%%%%%%%%%%%%%%%%%%%%%%%
% parámetros para configurar el formato del código en los entornos lstlisting
%%%%%%%%%%%%%%%%%%%%%%%%%%%%%%%%%%%%%%%%%%%%%%%%%%%%%%%%%%%%%%%%%%%%%%%%%%%%%
\lstset{ %
  backgroundcolor=\color{white},   % choose the background color; you must add \usepackage{color} or \usepackage{xcolor}
  basicstyle=\footnotesize,        % the size of the fonts that are used for the code
  breakatwhitespace=false,         % sets if automatic breaks should only happen at whitespace
  breaklines=true,                 % sets automatic line breaking
  captionpos=b,                    % sets the caption-position to bottom
  commentstyle=\color{mygreen},    % comment style
  deletekeywords={...},            % if you want to delete keywords from the given language
  %escapeinside={\%*}{*)},          % if you want to add LaTeX within your code
  %extendedchars=true,              % lets you use non-ASCII characters; for 8-bits encodings only, does not work with UTF-8
  %frame=single,	                % adds a frame around the code
  keepspaces=true,                 % keeps spaces in text, useful for keeping indentation of code (possibly needs columns=flexible)
  keywordstyle=\color{blue},       % keyword style
  language=[ANSI]C,                % the language of the code
  %otherkeywords={*,...},           % if you want to add more keywords to the set
  numbers=left,                    % where to put the line-numbers; possible values are (none, left, right)
  numbersep=5pt,                   % how far the line-numbers are from the code
  numberstyle=\tiny\color{mygray}, % the style that is used for the line-numbers
  rulecolor=\color{black},         % if not set, the frame-color may be changed on line-breaks within not-black text (e.g. comments (green here))
  showspaces=false,                % show spaces everywhere adding particular underscores; it overrides 'showstringspaces'
  showstringspaces=false,          % underline spaces within strings only
  showtabs=false,                  % show tabs within strings adding particular underscores
  stepnumber=1,                    % the step between two line-numbers. If it's 1, each line will be numbered
  stringstyle=\color{mymauve},     % string literal style
  tabsize=2,	                   % sets default tabsize to 2 spaces
  title=\lstname,                  % show the filename of files included with \lstinputlisting; also try caption instead of title
  morecomment=[s]{/*}{*/}
}


%----------------------------------------------------------------------------------------
%	SECTION 1
%----------------------------------------------------------------------------------------
\section{Enfoque para resolver el problema}

El Laboratorio de Biotecnología recopiló datos sobre las fechas de floración de durazneros
y las ubicaciones de las parcelas durante un período de cinco años. Con base en esta 
información, se evaluó la viabilidad de descargar imágenes satelitales correspondientes
a dichas parcelas en las fechas de floración, con el propósito de calcular índices de
vegetación y mejorar la precisión en las predicciones.

El proceso del trabajo se compone en: 

\begin{enumerate}
  \item Adquisición y preprocesamiento de datos
      \begin{itemize}
        \item Integración y limpieza de datos crudos: fechas de floración y ubicaciones de parcelas.
      \end{itemize}
  \item Obtención y procesamiento de imágenes satelitales
      \begin{itemize}
        \item Desarrollo de un script en Python para interactuar con la API de Google Earth Engine.
        \item Descarga de imágenes satelitales correspondientes a las parcelas en sus respectivas fechas de floración.
        \item Lectura y procesamiento de los archivos .TIFF para extraer índices de vegetación.
      \end{itemize}
  \item Construcción del dataset
      \begin{itemize}
        \item Confección de un nuevo conjunto de datos enriquecido con índices de vegetación.
        \item Estructuración y limpieza del dataset final.
      \end{itemize}
  \item  Modelado y entrenamiento
      \begin{itemize}
        \item Preprocesamiento de datos.
        \item Entrenamiento de modelos.
      \end{itemize}
  \item Evaluación y ajuste del modelo
      \begin{itemize}
        \item Cálculo de métricas de desempeño del modelo.
        \item Validación de resultados. Retroalimentación para mejora, y reentrenamiento de modelo.
      \end{itemize}
	\end{enumerate}

\section{Análisis exploratorio inicial}

Inicialmente, se procedió a la extracción de los registros de floración correspondientes 
a todas las parcelas en un lapso de 5 años. Una representación parcial de la 
estructura de dichos datos se observa en la tabla \ref{tab:firstdataset}.

	\begin{table}[h]
		\centering
		\caption{Estructura de datos de floración.}
		\begin{tabular}{l c c c c c}    
			\toprule
			\textbf{Número} & \textbf{ID} & \textbf{Dias-floracion-17} & \textbf{...} & \textbf{Dias-floracion-23} \\
			\midrule
			0 & clv2 & 67 & ... & 65 \\		
			1 & clv3 & NaN & ... & 31 \\
			2 & clv5 & 73 & ... & 68 \\
      ... & ... & ... & ... & ... \\
      184 & htv10 & 30 & ... & 73 \\
			\bottomrule
		\end{tabular}
		\label{tab:firstdataset}
	\end{table}

A continuación, se listan aspectos a tener en cuenta:
\begin{itemize}
  \item Existen 185 ids que corresponden a una parcela única.
  \item Cada polígono tiene de 1 a 3 árboles que corresponden a una misma variedad, es decir se tienen 185 variedades distintas
   de durazneros.
  \item La fecha (número de la celda) se toma cuando aproximadamente el \%50 de las flores del árbol están abiertas. 
 \end{itemize} 

En la tabla \ref{tab:nullsdataset} se presenta la insuficiencia de datos que presenta el dataset.

\begin{table}[h]
  \centering
  \caption{Cantidad de datos faltantes en el dataset original.}
  \begin{tabular}{l c c}    
    \toprule
     \textbf{Columna} & \textbf{Cantidad de Nulos} \\
    \midrule
    Dias-floracion-17 & 35 \\		
    Dias-floracion-18	 & 19  \\
    Dias-floracion-19	& 38  \\
    Dias-floracion-20	 & 47 \\
    Dias-floracion-21	 & 1 \\
    Dias-floracion-22	 & 2 \\
    Dias-floracion-23	 & 8 \\
    \bottomrule
  \end{tabular}
  \label{tab:nullsdataset}
\end{table}

La presencia de valores faltantes puede afectar negativamente el rendimiento de los modelos
de aprendizaje automático, introduciendo sesgos o impidiendo que ciertos algoritmos funcionen
correctamente. Identificar y gestionar estos valores ya sea imputándolos, eliminándolos o 
analizando su patrón de aparición permite mejorar la calidad del dataset, aumentar la 
robustez del modelo y garantizar que las predicciones sean más precisas y confiables.

\section{Proceso de descarga de imágenes}
El proceso completo de descarga de imágenes satelitales para cada parcela en las distintas fechas de 
floración puede desglosarse en dos etapas principales. La primera abarca el análisis y las consideraciones 
fundamentales para la selección de la fuente de adquisición de imágenes. La segunda se centra en la 
problemática detectada respecto a la proximidad entre parcelas y la estrategia adoptada para su resolución.

\subsection{Consideraciones}
En la sección \ref{sec:seguimientofloracion} se analizan las distintas plataformas de teledetección 
disponibles. A continuación, se detallan los criterios considerados para 
seleccionar la más adecuada:

\begin{itemize}
  \item Intervalo de revisión: Se refiere a la periodicidad con la que el satélite captura
    nuevas imágenes. Dado que intervalos largos pueden generar variaciones significativas 
    en la vegetación, se priorizó un satélite con mayor frecuencia de captura.
  \item Disponibilidad temporal de los datos: Se evaluó el rango de fechas en que los datos
    están disponibles. Como el análisis requiere información desde 2017, se seleccionó el 
    conjunto de datos que cubre dicho período.
  \item Cantidad de bandas espectrales: Las bandas espectrales representan distintos rangos 
    del espectro electromagnético y permiten analizar diversas características de la superficie
    terrestre. Se optó por la plataforma que ofrece el mayor número de bandas, ya que son 
    fundamentales para el cálculo de los índices de vegetación.
  \item Accesibilidad: Se consideró si el acceso a los datos es gratuito o pago. En este caso, 
    se priorizó el uso de fuentes de acceso libre.
  \item Resolución espacial: Hace referencia al tamaño del píxel en cada banda espectral, lo 
    que influye en el nivel de detalle de la imagen. Aunque un menor tamaño de píxel proporciona
     mayor precisión, este criterio no fue determinante en la selección.
\end{itemize}

Tras el análisis comparativo de las distintas plataformas se optó por Sentinel-2,
era la opción más adecuada para este trabajo, al reunir las condiciones 
necesarias en términos de resolución temporal, acceso a datos históricos, riqueza
espectral y disponibilidad gratuita.

\subsection{Elección de parcelas}

En la figura \ref{fig:parcelasSP} se observa el total de parcelas de árboles de durazno ubicadas 
en INTA San Pedro identificadas a través de Google Earth Engine. 

Luego, se determinó que la proximidad entre las parcelas era demasiado reducida para una correcta
extracción de los índices de vegetación, debido a que la resolución espacial del satélite 
resultaba mayor al área individual de cada parcela. Para abordar esta limitación, se llevó a cabo una 
subselección de parcelas que presentaran una distancia mínima adecuada entre sí. La figura \ref{fig:parcelasfinalSP} 
ilustra el conjunto final de parcelas seleccionadas tras este proceso.

\begin{figure}[h]
	\centering
	\includegraphics[width=0.8\textwidth]{./Figures/parcelas_san_pedro.PNG}
	\caption{Parcelas de durazneros de INTA San Pedro.}
	\label{fig:parcelasSP}
\end{figure}

\begin{figure}[h]
	\centering
	\includegraphics[width=0.8\textwidth]{./Figures/recorte_lotes_completo_sentinel.PNG}
	\caption{Parcelas seleccionadas de durazneros de INTA San Pedro.}
	\label{fig:parcelasfinalSP}
\end{figure}
