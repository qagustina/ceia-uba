\chapter{Introducción específica} % Main chapter title

\label{Chapter2}

%----------------------------------------------------------------------------------------
%	SECTION 1
%----------------------------------------------------------------------------------------
En esta sección se aborda el estudio de redes neuronales profundas aplicadas al análisis y predicción en 
agricultura, con un enfoque en la floración del duraznero. Se presenta una comparación entre diferentes 
arquitecturas de \textit{deep learning}, se destacan sus ventajas y limitaciones en este contexto. Además, se 
exploran los índices de vegetación más relevantes, utilizados para monitorear el estado de la cobertura 
vegetal y mejorar la precisión de los modelos predictivos.

\section{Aprendizaje profundo}

El uso de arquitecturas de aprendizaje profundo ha demostrado ser efectivo permitiendo 
una gestión agrícola más precisa y eficiente. A continuación, se describen algunas de las principales
arquitecturas aplicadas en este ámbito:


\begin{enumerate}
	\item Redes Neuronales Convolucionales (CNN): son ampliamente utilizadas para analizar datos 
	visuales, como imágenes satelitales o fotografías de cultivos. En el contexto de la floración del duraznero,
	pueden identificar patrones y características en imágenes que indican el inicio del proceso de
	floración. Por ejemplo, un estudio implementó dos arquitecturas de CNN para detectar la severidad de 
	lesiones causadas por enfermedades en hojas de durazno, y obtuvo una precisión del 87\,\% \citep{Rodriguez2023}. 
	\item Redes Generativas Antagónicas (GAN): esta arquitectura de aprendizaje profundo está compuesta 
	por dos modelos neuronales que compiten entre sí: un generador, que crea datos sintéticos, y un 
	discriminador, que evalúa la autenticidad de los datos. Esta dinámica permite generar datos 
	realistas similares a los de entrenamiento. En el ámbito agrícola, se han utilizado para generar imágenes 
	sintéticas de cultivos, lo cual facilitó la ampliación de conjuntos de datos para entrenar otros modelos de aprendizaje 
	automático \citep{Goodfellow2014}.
	\item Redes Neuronales de Base Radial (RBF): han sido aplicadas en estudios agrícolas para modelar relaciones
	 no lineales entre variables climáticas y fenológicas. Poseen gran capacidad para aproximar funciones complejas.
	 Por ejemplo, un estudio empleó RBF para estimar la evapotranspiración de referencia, y demostró su capacidad
	para modelar relaciones no lineales entre variables climáticas y fenómenos agrícolas \citep{Cerv2012}.
\end{enumerate}
	
En la tabla \ref{tab:redes_dl} se presenta un cuadro comparativo de las distintas arquitecturas menciondas.

\begin{table}[h]
    \centering
    \caption{Comparación de arquitecturas de Deep Learning}
    \renewcommand{\arraystretch}{1} % Aumenta el espacio entre filas
    \begin{tabular}{|>{\raggedright}m{2cm}|m{5cm}|m{5cm}|} % Usa celdas de tamaño fijo
        \hline
        \textbf{\large Modelo} & \textbf{\large Ventaja} & \textbf{\large Desventaja} \\
        \hline
        \large CNN & \large Excelente para extracción de características espaciales, especialmente en imágenes. 
		& \large Requiere gran cantidad de datos etiquetados y es computacionalmente costoso. \\
        \hline
        \large GAN & \large Genera datos sintéticos realistas, útil en aumento de datos y generación de imágenes. 
		& \large Difícil de entrenar y puede sufrir de colapso de modo, lo que afecta la calidad de la generación. \\
        \hline
        \large RBF & \large Buena aproximación de funciones no lineales, útil en problemas de clasificación y regresión.
		 & \large No escala bien con grandes volúmenes de datos y su entrenamiento puede ser ineficiente. \\
        \hline
        \large RNN & \large Captura dependencias temporales en secuencias, aplicable en procesamiento de lenguaje natural 
		y series temporales. & \large Problemas con gradientes desaparecidos en largas secuencias, requiere técnicas como 
		LSTM o GRU para mejorar rendimiento.\\
        \hline
    \end{tabular}
    \label{tab:redes_dl}
\end{table}


	Al seleccionar la arquitectura adecuada para un modelo de deep learning en tareas como la predicción de
	floración de plantas, es crucial considerar varios factores, entre ellos la naturaleza de los datos, el tipo
	de problema a resolver y los recursos disponibles. Las redes neuronales convolucionales (CNN) son 
  	ideales para tareas relacionadas con el procesamiento de imágenes o datos espaciales. En cambio, 
  	las redes neuronales recurrentes (RNN) y sus variantes como LSTM, son más adecuadas para modelar 
  	secuencias temporales, como los datos fenológicos de las plantas. Por otro lado, las redes generativas
   	antagónicas (GAN) pueden ser útiles cuando se requiere generar datos sintéticos, especialmente 
   	cuando hay escasez de datos etiquetados. La comparación entre modelos permite identificar cuál se adapta
	mejor a los objetivos del trabajo, con el fin de maximizar la precisión y eficiencia del sistema, y 
	minimizando los errores o el sobreajuste. La elección adecuada y la experimentación con diferentes
	enfoques garantizan un análisis robusto y confiable, fundamental para obtener resultados
	aplicables en la práctica agrícola.

\section{Procesamiento de Imágenes}
	El procesamiento de imágenes satelitales se ha convertido en una herramienta esencial para diversos
	campos, incluyendo la agricultura, la gestión de recursos naturales, la monitorización ambiental y 
 	la planificación urbana. Actualmente, existen varios métodos y técnicas avanzadas utilizados para
  	analizar y extraer información útil de estas imágenes. 
	

\subsection{Índices de vegetación}

La composición espectral del flujo radiante que emana de la superficie terrestre 
proporciona información sobre las propiedades físicas del suelo, el agua y la vegetación 
en entornos terrestres. Las técnicas, modelos e índices de teledetección están diseñados 
para convertir esta información espectral en una forma fácilmente interpretable \citep{Bannari1995}.
La información obtenida por teledetección sobre el crecimiento, el vigor y la dinámica de
la vegetación terrestre puede ser de gran provecho para el monitoreo del medio ambiente, 
la conservación de la biodiversidad, la agricultura, la silvicultura, las infraestructuras 
verdes urbanas y otros campos relacionados.

A continuación se presentan índices de vegetación relevantes para este trabajo:

\begin{itemize}
	\item \textit{Normalized Difference Vegetation Index} (NDVI) \citep{Xue2017}: se calcula como 
	   	relación normalizada entre las bandas roja e infrarroja cercana. Sus valores van en el rango
	   	entre 0 y 1. Tiene una reacción sensible a la vegetación verde, incluso en zonas con cobertura 
		vegetal escasa. El NDVI es sensible a los efectos del brillo y color del suelo,
		la atmósfera, las nubes y la sombra de las nubes, y la sombra del dosel foliar. Por ello su
		aplicación requiere la calibración de la teledetección. Se expresa como en la ecuación \ref{eq:ndvi}:
	
		\begin{equation}
			\label{eq:ndvi}
			NDVI = \left( \frac{p_{NIR} - p_{R}}{p_{NIR}} \right) + p_{R}
		\end{equation}

	\item \textit{Enhanced Vegetation Index} (EVI) \citep{Xue2017}: actúa como un parámetro que 
		corrige simultáneamente los efectos del suelo y de la atmósfera. Su formulación incluye los valores de NIR,
		 $R$ y $B$, previamente corregidos por la atmósfera. El término $L$ representa un parámetro de ajuste asociado
		  al suelo, cuyo valor se establece en 1. Además, se incorporan dos parámetros constantes con valores de 6 y 
		  7.5, respectivamente. La expresión matemática del índice se presenta en la ecuación \ref{eq:evi}:

		\begin{equation}
			\label{eq:evi}
			EVI = 2.5 \times \frac{(P_n - P_r)}{P_n + C_1 P_r - C_2 P_b + L}
		\end{equation}

	\item \textit{Atmospherically Resistant Vegetation Index} (ARVI) \citep{Xue2017}: se utiliza habitualmente para eliminar los efectos de los
		aerosoles atmosféricos. Se basa en el supuesto de que la atmósfera afecta significativamente a $R$ en comparación con el 
		NIR y puede reducir eficazmente la dependencia de este índice de vegetación de los efectos atmosféricos. En su fórmula 
		$RB$ es la diferencia entre $B$ y $R$, y está relacionada con la reflectancia influenciada por la dispersión molecular y la absorción gaseosa para 
		las correcciones por ozono, y representa los parámetros de climatización. La expresión correspondiente se muestra en la ecuación \ref{eq:arvi}:

		\begin{equation}
			\label{eq:arvi}
			ARVI = \frac{(NIR - RB)}{(NIR + RB)}
		\end{equation}
	
	\item \textit{Soil-Adjusted Vegetation Index} (SAVI) \citep{Xue2017}: se estableció para mejorar la 
		sensibilidad del NDVI al fondo del suelo, donde $L$ es el índice de condicionamiento del suelo.
		El rango de $L$ es de 0 a 1. Su fórmula se puede ver expresada en la ecuación \ref{eq:savi}:

		\begin{equation}
			\label{eq:savi}
			SAVI = \frac{(P_n - P_r)(1 + L)}{P_n + P_r + L}
		\end{equation}
\end{itemize}

	Los índices de vegetación son herramientas fundamentales en el estudio de la agricultura, ya que permiten
 	evaluar de manera objetiva y eficiente el estado de la cobertura vegetal mediante el análisis de imágenes
  	satelitales. Estos índices, como el NDVI, SAVI, ARVI y EVI, proporcionan información clave sobre la salud de
	los cultivos, la disponibilidad de agua, el estrés hídrico y la productividad agrícola. Su uso facilita 
	la toma de decisiones en la gestión de los cultivos y son esenciales para mejorar la eficiencia y 
	sostenibilidad en la agricultura moderna.

\section{Teledetección}
	\label{sec:seguimientofloracion}
	En la tabla \ref{tab:plataformas} se presentan plataformas de teledetección disponibles y requisitos sugeridos 
	(*marginal, **óptimo) para supervisar la fenología de la floración forestal\citep{Dixon2023}.
	
	
	\begin{table}[h]
		\centering
		\caption{Plataformas de teledetección disponibles.}
		\begin{tabular}{l c c c c}    
			\toprule
			\textbf{Sensor} & \textbf{Resolución} & \textbf{Frecuencia} & \textbf{Extensión} & \textbf{Coste} \\
			\midrule
			PlanetScope & 3m*  & Diariamente**  & Regional** & Moderado* \\		
			Sentinel-2	 & 10 - 20 m  & 5 días  & Continental**   & Bajo**  \\
			Landsat	 & 30 m  & 16 días  & Continental** & Bajo** \\
			\bottomrule
		\end{tabular}
		\label{tab:plataformas}
	\end{table}

\section{Dependencias y Herramientas}

\subsection{Python}
Es uno de los lenguajes de programación más utilizados en el análisis de datos, incluidas las 
imágenes satelitales, gracias a su amplia variedad de bibliotecas de procesamiento de imágenes y análisis
geoespacial. Para el preprocesamiento de imágenes satelitales, algunas de las bibliotecas más destacadas 
incluyen:

\begin{enumerate}
	\item Rasterio: biblioteca que permite leer y escribir archivos raster, que son fundamentales en el 
	análisis de imágenes satelitales. Rasterio facilita la manipulación de datos en formatos comunes como
	GeoTIFF.
	\item NumPy: utilizada para el manejo de matrices y operaciones matemáticas sobre las imágenes, especialmente
	útil cuando se trabaja con grandes volúmenes de datos raster.  
	\item GeoPandas: es una extensión de Pandas diseñada para el manejo de datos geoespaciales. Agrega soporte 
	para trabajar con geometrías como puntos, líneas y polígonos dentro de un DataFrame, facilita la
	manipulación de datos espaciales en Python.
	\item Plotly: es una biblioteca de visualización interactiva en Python que permite la creación de gráficos
	 avanzados, incluye mapas geoespaciales. Es una herramienta clave para la exploración y análisis de 
	 datos provenientes de imágenes satelitales, ya que facilita la representación de patrones y
	tendencias en mapas interactivos.
	\end{enumerate}

Ventajas:

\begin{itemize}
	\item Python es flexible y permite integrar múltiples bibliotecas para un flujo de trabajo personalizado.
	\item Tiene una gran comunidad y documentación, lo que facilita la solución de problemas y la optimización del código.
	\item Es gratuito y de código abierto, lo que lo hace accesible para cualquier usuario. 
\end{itemize}

\subsection{Google Earth Engine}

Es una plataforma de procesamiento de imágenes geoespaciales en la nube que permite acceder a grandes 
volúmenes de datos geoespaciales, incluidos los conjuntos de datos satelitales históricos. GEE es utilizado
para realizar análisis geoespaciales y monitoreo ambiental a gran escala.

Características Principales

\begin{itemize}
	\item Acceso a grandes bases de datos de satélites: ofrece acceso a varios conjuntos de datos satelitales,
	 como Landsat, MODIS, Sentinel-1 y Sentinel-2, entre otros.
	\item Preprocesamiento automatizado: facilita el preprocesamiento de imágenes satelitales mediante funciones predefinidas
	 para corrección atmosférica, resampling, y mosaicos de imágenes, lo que permite un análisis rápido y 
	 preciso.
	\item Análisis a gran escala: permite realizar análisis a nivel global o en áreas de interés específicas,
	 utiliza algoritmos potentes de análisis de imágenes, como la clasificación de imágenes y la detección 
	 de cambios.
\end{itemize}




