% Chapter 1

\chapter{Introducción general} % Main chapter title

\label{Chapter1} % For referencing the chapter elsewhere, use \ref{Chapter1} 
\label{IntroGeneral}

En este capítulo se introduce la problemática y se interpreta la importancia que
implica el trabajo. Luego, se realiza un análisis del estado de arte sobre la ges-
tión eficiente de salud de cultivos y se puntualizan los objetivos y el alcance del
trabajo.

%----------------------------------------------------------------------------------------

% Define some commands to keep the formatting separated from the content 
\newcommand{\keyword}[1]{\textbf{#1}}
\newcommand{\tabhead}[1]{\textbf{#1}}
\newcommand{\code}[1]{\texttt{#1}}
\newcommand{\file}[1]{\texttt{\bfseries#1}}
\newcommand{\option}[1]{\texttt{\itshape#1}}
\newcommand{\grados}{$^{\circ}$}

%----------------------------------------------------------------------------------------

%\section{Introducción}

%----------------------------------------------------------------------------------------
\section{Contexto del trabajo}

En la EEA San Pedro de INTA se busca intensificar de manera sustentable la producción
de frutales, hortalizas y viveros. Actualmente, la inspección y evaluación
dependen de visitas de campo y observaciones manuales, procesos costosos, laboriosos
 y con alcance limitado. Además, el cambio climático y la variabilidad
meteorológica añaden capas adicionales de complejidad la gestión agronómica.

El Laboratorio de Biotecnología cuenta con datos históricos sobre las etapas de
floración y maduración de frutos en durazneros. En este contexto, las imágenes
satelitales se presentan como una herramienta poderosa, ya que ofrecen datos
consistentes y de amplia cobertura, permitiendo evaluar la vegetación de forma
precisa y objetiva mediante índices especializados.

La intención de este trabajo es automatizar la gestión de montes frutales mediante el 
uso de datos disponibles e índices obtenidos de imágenes satelitales. Para lograr este
objetivo, se hará uso de métodos de aprendizaje automático, aprendizaje profundo y 
técnicas de preprocesamiento de datos. Se propone una herramienta accesible para los 
usuarios, que optimiza los resultados y reduce el tiempo y esfuerzo requeridos.


\section{Estado del arte}

La recopilación de grandes cantidades de datos agrícolas ayuda a mejorar la toma
de decisiones para enriquecer la salud de los cultivos. Al mismo tiempo, el
aprendizaje profundo ha experimentado una gran popularidad en muchas áreas
de investigación y en diferentes modalidades de datos. Las imágenes por satélite
están disponibles en cantidades sin precedentes, lo que ha impulsado la investigación
en el ámbito de la teledetección. La naturaleza ávida de datos de los
modelos de aprendizaje profundo y este enorme volumen de datos resultan una
combinación perfecta.

\subsection{Seguimiento de floración mediante teledetección}
En la tabla \ref{tab:plataformas} se presentan plataformas de teledetección disponibles y requisitos sugeridos 
(*marginal, **óptimo) para supervisar la fenología de la floración forestal\citep{Dixon2023}.


\begin{table}[h]
	\centering
	\caption{Plataformas de teledetección disponibles.}
	\begin{tabular}{l c c c c}    
		\toprule
		\textbf{Sensor} & \textbf{Resolución} & \textbf{Frecuencia} & \textbf{Extensión} & \textbf{Coste} \\
		\midrule
		PlanetScope & 3m*  & Diariamente**  & Regional** & Moderado* \\		
		Sentinel-2	 & 10 - 20 m  & 5 días  & Continental**   & Bajo**  \\
		Landsat	 & 30 m  & 16 días  & Continental** & Bajo** \\
		\bottomrule
	\end{tabular}
	\label{tab:plataformas}
\end{table}



\subsection{Aprendizaje automático para pronóstico}

\begin{itemize}
  \item ARIMA \citep{Siami-Namini2018}: es un modelo de series temporales que combina autoregresión,
   diferenciación y media móvil para analizar y predecir datos no estacionarios. Ha demostrado 
   su superioridad en precisión y exactitud a la hora de predecir los próximos intervalos de 
   las series temporales.
   \item MLP \citep{Feng2020}: es un algoritmo de aprendizaje supervisado que aprende una función no lineal para
   problemas de clasficiación o regresión. Este modelo ha demostrado un buen rendimiento en 
   pronóstico.
  \item RNN \citep{Sebaa2020}: los datos de entrada de esta arquitectura son datos pasados
   y actuales, están diseñadas específicamente para tratar datos secuenciales, como secuencias de 
   palabras en problemas relacionados con la traducción automática, datos de audio en el 
   reconocimiento del habla o series temporales en problemas de pronóstico.
  \item LSTM \citep{Siami-Namini2018}: es un tipo de red neuronal Red Neuronal Recurrente (RNN) con la capacidad de
   de recordar los valores de etapas anteriores para predicción de la siguiente secuencia.
  \item SVR \citep{Makridakis2018}: es un proceso de regresión realizado por una máquina de vectores soporte que intenta
   identificar el hiperplano que maximiza el margen entre dos clases y minimiza el error total bajo 
   tolerancia. Se introduce una penalización de complejidad que equilibra el nivel de precisión en 
   pronósticos.
  \item Random Forest Regressor \citep{Nyemeche2023}: es un algoritmo que utiliza múltiples árboles de decisión para
  encontrar la salida del conjunto de datos de entrenamiento, ha demostrado dar buenos resultados en pronóstico.
  \end{itemize}

\subsection{Comparación de técnicas}

En la tabla \ref{tab:rmse} se muestra una comparación del RMSE (Error cuadrático medio) que se realizó
con datos de series temporales. Los valores de RMSE indican que los modelos basados en LSTM superan a los 
basados en ARIMA con un margen elevado \citep{Siami-Namini2018}.

\begin{table}[h]
	\centering
	\caption{Los RMSE de los modelos ARIMA y LSTM.}
	\begin{tabular}{l c c c}    
		\toprule
		\textbf{Metric} & \textbf{ARIMA} & \textbf{LSTM} & \textbf{\% Reducción en RMSE} \\
		\midrule
		RMSE Avg & 511.481  & 64.213  & -87.445 \\		
		\bottomrule
	\end{tabular}
	\label{tab:rmse}
\end{table}

\section{Alcance y Objetivos}
El objetivo principal del trabajo consistió en desarrollar un algoritmo que permita descargar y analizar 
imágenes satelitales para vincularlas con características específicas de durazneros. Específicamente, 
determinar, a partir de los datos disponibles, el progreso de las etapas de floración y maduración de 
los frutos en el árbol.

\subsection{Alcance del proyecto}
A continuación, se detallan las actividades incluidas en este trabajo:

\begin{itemize}
  \item Evaluación de las diferentes fuentes de datos disponibles:
  \begin{itemize}
    \item Datos tabulares, que corresponden a mediciones de campo
    de las etapas de floración y maduración de frutos, obtenidos durante 5
    años.
    \item Imágenes satelitales, correspondientes a los lotes de durazneros.
    \end{itemize}
  \item La determinación del progreso de las etapas de floración y maduración de
  los frutos en el duraznero.
  \item El desarrollo de una herramienta de fácil acceso para el equipo del cliente.
  \item La elaboración de un informe que detalle el procedimiento realizado y los resultados obtenidos.  
  \end{itemize}

  Los siguientes elementos quedan fuera del alcance:
  \begin{itemize}
    \item El desarrollo de una interfaz web para el sistema.
    \item El despligue del desarrollo en producción.
    \end{itemize}

