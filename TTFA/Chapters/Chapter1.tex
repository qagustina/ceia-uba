% Chapter 1

\chapter{Introducción general} % Main chapter title

\label{Chapter1} % For referencing the chapter elsewhere, use \ref{Chapter1} 
\label{IntroGeneral}

En este capítulo se introduce a la problemática y se interpreta la importancia que implica el proyecto. 
Luego, se realiza un análisis del estado de arte sobre la gestión eficiente de salud de cultivos
y se puntualizan los objetivos y el alcance del trabajo.

%----------------------------------------------------------------------------------------

% Define some commands to keep the formatting separated from the content 
\newcommand{\keyword}[1]{\textbf{#1}}
\newcommand{\tabhead}[1]{\textbf{#1}}
\newcommand{\code}[1]{\texttt{#1}}
\newcommand{\file}[1]{\texttt{\bfseries#1}}
\newcommand{\option}[1]{\texttt{\itshape#1}}
\newcommand{\grados}{$^{\circ}$}

%----------------------------------------------------------------------------------------

%\section{Introducción}

%----------------------------------------------------------------------------------------
\section{Contexto}

La gestión eficiente de los montes frutales enfrenta varios desafíos significativos, entre los que se 
incluye la necesidad de monitoreo constante para asegurar la salud y productividad de los cultivos,  
en la figura \ref{fig:duraznero} se presenta el ciclo de vida del árbol de durazno. Los métodos
tradicionales de inspección y evaluación, que dependen en gran medida de visitas de campo 
y observaciones manuales, son laboriosos, costosos y a menudo limitados en alcance y frecuencia. 
Estos métodos también pueden ser subjetivos, dependiendo de la experiencia y percepción del observador,
lo que puede llevar a inconsistencias en la evaluación. Además, factores como el cambio climático y
la variabilidad en las condiciones meteorológicas añaden capas adicionales de complejidad a la gestión
agronómica, ya que afectan directamente la salud y el rendimiento de los cultivos.

\vspace{1cm}

\begin{figure}[htbp]
	\centering
	\includegraphics[width=.5\textwidth]{./Figures/ciclo-vida-durazno.jpg}
  \caption{Ciclo de vida del árbol de durazno\protect\footnotemark.}
	\label{fig:duraznero}
\end{figure}
\footnotetext{Imagen tomada de \url{https://pt.dreamstime.com/}}
\vspace{1cm}


En este contexto, las imágenes satelitales emergen como una herramienta poderosa para abordar estos
problemas. Estas imágenes proporcionan datos de observación de la Tierra que son consistentes, 
repetitivos y de amplia cobertura, lo cual es esencial para un monitoreo efectivo. A través de 
imágenes satelitales, se pueden obtener índices de vegetación, como el NDVI (Índice de Vegetación 
de Diferencia Normalizada), que permiten evaluar la vegetación de manera precisa y objetiva.
Estos índices son cruciales para detectar cambios en la vegetación mucho antes de que sean
visibles a simple vista.  

Para este proyecto, se pretende automatizar estas tareas con una herramienta de fácil acceso 
a los usuarios, donde además de poder obtener mejores resultados, se facilitará el labor y 
tiempo que requieren las mismas.

%----------------------------------------------------------------------------------------

\section{Estado del arte}

La recopilación de grandes cantidades de datos agrícolas ayuda a mejorar la toma de decisiones
para mejorar la salud de los cultivos. Al mismo tiempo, el aprendizaje profundo ha experimentado
una gran popularidad en muchas áreas de investigación y en diferentes modalidades de datos.
Las imágenes por satélite están disponibles en cantidades sin precedentes, lo que ha impulsado
la investigación en el ámbito de la teledetección. La naturaleza ávida de datos de los 
modelos de aprendizaje profundo y este enorme volumen de datos resultan una combinación perfecta.

\textit{Imágenes Satelitales}

Casi todas las imágenes satelitales tienen al menos 4 canales de color (rojo, verde, azul e 
infrarrojo cercano), muchas tienen más de 10 canales de color (por ejemplo, Sentinel-2), 
lo que proporciona mucha más información por píxel que las fuentes terrestres tradicionales. 
En los sensores existe un ruido significativo por la falta de resoluciones espaciales y 
espectrales, además de los errores que calculan la reflectancia. Estas importantes fuentes 
de ruido han llevado al dominio de los algoritmos Aprendizaje Automático para aprender las
variadas apariencias de las superficies a partir de los datos. 

Existen varias fuentes de imágenes satelitales disponibles gratuitamente con 
cobertura mundial para entrenar estos algoritmos. 

 
\begin{enumerate}
	\item Imágenes MODIS a una resolución de 250-1000 m que ha estado disponible públicamente desde 2000.
	\item Imágenes de Landsat que han estado disponibles gratuitamente para el público desde 2008.
  \item Imágenes del programa Sentinel de la Agencia Espacial Europea que ha proporcionado
  imágenes ópticas a una resolución de 10-60 m e imágenes de radar de apertura sintética (SAR) 
  a una resolución de 5-40 m desde 2014.
\end{enumerate}

%----------------------------------------------------------------------------------------

\section{Alcance y objetivos}

En esta sección se describe el propósito general del trabajo, así como las tareas llevadas 
a cabo a lo largo de todo el desarrollo. Se detallan los principales objetivos que motivan 
esta investigación, incluyendo la importancia del monitoreo de los montes frutales y el uso 
de modelos de aprendizaje automático para la predicción de la floración.

\subsection{Propósito del proyecto}

El objetivo principal del proyecto consistió en desarrollar un algoritmo que permita descargar y 
analizar imágenes satelitales para vincularlas con características específicas de durazneros. 
Específicamente, determinar, a partir de los datos disponibles, el progreso de las etapas de 
floración y maduración de los frutos en el árbol. Este trabajo implicó la aplicación de distintas 
arquitecturas de modelos de aprendizaje automático y redes neuronales artificiales. Un indicador
importante para medir el éxito del trabajo es el aumento de las métricas que indican el rendimiento 
de los modelos implementados. 

\subsection{Alcance del proyecto}

A continuación, se detallan las actividades incluidas en este trabajo:

\begin{itemize}
  \item La evaluación de las diferentes fuentes de datos para la realización de este trabajo.
  \begin{itemize}
    \item Evaluación de datos tabulares que corresponden mediciones a 
    campo de las etapas de floración y maduración de frutos, obtenidos durante 5 años.
    \item Evaluación de imágenes satelitales correspondientes a los lotes de durazneros.
  \end{itemize}
  \item La determinación del progreso de las etapas de floración y maduración de los frutos en el
  duraznero.
  \item El desarrollo de una herramienta de fácil acceso para el equipo del cliente.
  \item La elaboración de un informe que detalle el procedimiento realizado y resultados.
\end{itemize}

Los siguientes elementos quedan fuera del alcance:

\begin{itemize}
  \item El desarrollo de una interfaz para el sistema.
  \item El despligue del desarrollo en producción.
  \item La difusión de los datos, son confidenciales y propiedad del INTA.
\end{itemize}

