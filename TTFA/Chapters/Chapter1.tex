% Chapter 1

\chapter{Introducción general} % Main chapter title

\label{Chapter1} % For referencing the chapter elsewhere, use \ref{Chapter1} 
\label{IntroGeneral}

En este capítulo se introduce a la problemática y se interpreta la importancia que implica el proyecto. 
Luego, se realiza un análisis del estado de arte sobre la gestión eficiente de salud de cultivos
y se puntualizan los objetivos y el alcance del trabajo.

%----------------------------------------------------------------------------------------

% Define some commands to keep the formatting separated from the content 
\newcommand{\keyword}[1]{\textbf{#1}}
\newcommand{\tabhead}[1]{\textbf{#1}}
\newcommand{\code}[1]{\texttt{#1}}
\newcommand{\file}[1]{\texttt{\bfseries#1}}
\newcommand{\option}[1]{\texttt{\itshape#1}}
\newcommand{\grados}{$^{\circ}$}

%----------------------------------------------------------------------------------------

%\section{Introducción}

%----------------------------------------------------------------------------------------
\section{Contexto}

La gestión eficiente de los montes frutales enfrenta varios desafíos significativos, entre los que se 
incluye la necesidad de monitoreo constante para asegurar la salud y productividad de los cultivos.
Los métodos tradicionales de inspección y evaluación, que dependen en gran medida de visitas de campo 
y observaciones manuales, son laboriosos, costosos y a menudo limitados en alcance y frecuencia. 
Estos métodos también pueden ser subjetivos, dependiendo de la experiencia y percepción del observador,
lo que puede llevar a inconsistencias en la evaluación. Además, factores como el cambio climático y
la variabilidad en las condiciones meteorológicas añaden capas adicionales de complejidad a la gestión
agronómica, ya que afectan directamente la salud y el rendimiento de los cultivos.

En este contexto, las imágenes satelitales emergen como una herramienta poderosa para abordar estos
problemas. Estas imágenes proporcionan datos de observación de la Tierra que son consistentes, 
repetitivos y de amplia cobertura, lo cual es esencial para un monitoreo efectivo. A través de 
imágenes satelitales, se pueden obtener índices de vegetación, como el NDVI (Índice de Vegetación 
de Diferencia Normalizada), que permiten evaluar la vegetación de manera precisa y objetiva.
Estos índices son cruciales para detectar cambios en la vegetación mucho antes de que sean
visibles a simple vista.  

Para este proyecto, se pretende automatizar estas tareas con una herramienta de fácil acceso 
a los usuarios, donde además de poder obtener mejores resultados, se facilitará el labor y 
tiempo que requieren las mismas.

\subsection{sec}

Si sos nuevo en \LaTeX{}, hay un muy buen libro electrónico - disponible gratuitamente en Internet como un archivo PDF - llamado, \enquote{A (not so short) Introduction to \LaTeX{}}. El título del libro es generalmente acortado a simplemente \emph{lshort}. Puede descargar la versión más reciente en inglés (ya que se actualiza de vez en cuando) desde aquí:
\url{http://www.ctan.org/tex-archive/info/lshort/english/lshort.pdf}

Se puede encontrar la versión en español en la lista en esta página: \url{http://www.ctan.org/tex-archive/info/lshort/}

\subsubsection{Una subsubsección}

Acá tiene un ejemplo de una ``subsubsección'' que es el cuarto nivel de ordenamiento del texto, después de capítulo, sección y subsección.  Como se puede ver, las subsubsecciones no van numeradas en el cuerpo del documento ni en el índice.  El formato está definido por la plantilla y no debe ser modificado.

\subsection{sec}

Si estás escribiendo un documento con mucho contenido matemático, entonces es posible que desees leer el documento de la AMS (American Mathematical Society) llamado, \enquote{A Short Math Guide for \LaTeX{}}. Se puede encontrar en línea en el siguiente link: \url{http://www.ams.org/tex/amslatex.html} en la sección \enquote{Additional Documentation} hacia la parte inferior de la página.


%----------------------------------------------------------------------------------------

\section{Estado del arte}

Si estás familiarizado con \LaTeX{}, entonces podés explorar la estructura de directorios de esta plantilla y proceder a personalizarla agregando tu información en el bloque \emph{INFORMACIÓN DE LA PORTADA} en el archivo \file{memoria.tex}.  

Se puede continuar luego modificando el resto de los archivos siguiendo los lineamientos que se describen en la sección \ref{sec:FillingFile} en la página \pageref{sec:FillingFile}.

Debés asegurarte de leer el capítulo \ref{Chapter2} acerca de las convenciones utilizadas para las Memoria de los Trabajos Finales de la \degreename.

Si sos nuevo en \LaTeX{}, se recomienda que continúes leyendo el documento ya que contiene información básica para aprovechar el potencial de esta herramienta.


%----------------------------------------------------------------------------------------

\section{Alcance y objetivos}

blablabla ...
\subsection{Propósito del proyecto}

El objetivo principal del proyecto consistió en desarrollar un algoritmo que permita descargar y 
analizar imágenes satelitales para vincularlas con características específicas de durazneros. 
Específicamente, determinar, a partir de los datos disponibles, el progreso de las etapas de 
floración y maduración de los frutos en el árbol. Este trabajo implicó la aplicación de distintas 
arquitecturas de modelos de aprendizaje automático y redes neuronales artificiales. Los indicadores
clave para medir el éxito del trabajo es el aumento de las métricas que indican el rendimiento 
de los modelos ejecutados. 

\subsection{Alcance del proyecto}

A continuación, se detallan las actividades incluidas en este trabajo:

\begin{itemize}
  \item La evaluación de las diferentes fuentes de datos para la realización de este trabajo.
  \begin{itemize}
    \item Evaluación de datos tabulares que corresponden mediciones a 
    campo de las etapas de floración y maduración de frutos, obtenidos durante 5 años.
    \item Evaluación de imágenes satelitales correspondientes a los lotes de durazneros.
  \end{itemize}
  \item La determinación del progreso de las etapas de floración y maduración de los frutos en el
  duraznero.
  \item El desarrollo de una herramienta de fácil acceso para el equipo del cliente.
  \item La elaboración de un informe que detalle el procedimiento realizado y resultados.
\end{itemize}

Los siguientes elementos quedan fuera del alcance:

\begin{itemize}
  \item El desarrollo de una interfaz para el sistema.
  \item El despligue del desarrollo en producción.
  \item La difusión de los datos, son confidenciales y propiedad del INTA.
\end{itemize}

%----------------------------------------------------------------------------------------

\section{Entorno de trabajo}

Ante de comenzar a editar la plantilla debemos tener un editor \LaTeX{} instalado en nuestra computadora.  En forma análoga a lo que sucede en lenguaje C, que se puede crear y editar código con casi cualquier editor, existen ciertos entornos de trabajo que nos pueden simplificar mucho la tarea.  En este sentido, se recomienda, sobre todo para los principiantes en \LaTeX{} la utilización de TexMaker, un programa gratuito y multi-plantaforma que está disponible tanto para windows como para sistemas GNU/linux.

La versión más reciente de TexMaker es la 4.5 y se puede descargar del siguiente link: \url{http://www.xm1math.net/texmaker/download.html}. Se puede consultar el manual de usuario en el siguiente link: \url{http://www.xm1math.net/texmaker/doc.html}.
 

\subsection{Paquetes adicionales}

Si bien durante el proceso de instalación de TexMaker, o cualquier otro editor que se haya elegido, se instalarán en el sistema los paquetes básicos necesarios para trabajar con \LaTeX{}, la plantilla de los trabajos de Especialización y Maestría requieren de paquete adicionales.

Se indican a continuación los comandos que se deben introducir en la consola de Ubuntu (ctrl + alt + t) para instalarlos:

\begin{lstlisting}[language=bash]
  $ sudo apt install texlive-lang-spanish texlive-science 
  $ sudo apt install texlive-bibtex-extra biber
  $ sudo apt install texlive texlive-fonts-recommended
  $ sudo apt install texlive-latex-extra
\end{lstlisting}


\subsection{Configurando TexMaker}
\label{subsec:configurando}



Una vez instalado el programa y los paquetes adicionales se debe abrir el archivo memoria.tex con el editor para ver una pantalla similar a la que se puede apreciar en la figura \ref{fig:texmaker}. 
Una vez instalado el programa y los paquetes adicionales se debe abrir el archivo memoria.tex con el editor para ver una pantalla similar a la que se puede apreciar en la figura \ref{fig:texmaker}. 
Una vez instalado el programa y los paquetes adicionales se debe abrir el archivo memoria.tex con el editor para ver una pantalla similar a la que se puede apreciar en la figura \ref{fig:texmaker}. 
Una vez instalado el programa y los paquetes adicionales se debe abrir el archivo memoria.tex con el editor para ver una pantalla similar a la que se puede apreciar en la figura \ref{fig:texmaker}. 

\vspace{1cm}

\begin{figure}[htbp]
	\centering
	\includegraphics[width=.5\textwidth]{./Figures/texmaker.png}
	\caption{Entorno de trabajo de texMaker.}
	\label{fig:texmaker}
\end{figure}

\vspace{1cm}

Notar que existe una vista llamada Estructura a la izquierda de la interfaz que nos permite abrir desde dentro del programa los archivos individuales de los capítulos.  A la derecha se encuentra una vista con el archivo propiamente dicho para su edición. Hacia la parte inferior se encuentra una vista del log con información de los resultados de la compilación.  En esta última vista pueden aparecen advertencias o \textit{warning}, que normalmente pueden ser ignorados, y los errores que se indican en color rojo y deben resolverse para que se genere el PDF de salida.

Recordar que el archivo que se debe compilar con PDFLaTeX es \file{memoria.tex}, si se tratara de compilar alguno de los capítulos saldría un error.  Para salvar la molestia de tener que cambiar de archivo para compilar cada vez que se realice una modificación en un capítulo, se puede definir el archivo \file{memoria.tex} como ``documento maestro'' yendo al menú opciones -> ``definir documento actual como documento maestro'', lo que permite compilar con PDFLaTeX memoria.tex directamente desde cualquier archivo que se esté modificando . Se muestra esta opción en la figura \ref{fig:docMaestro}.

\begin{figure}[ht]
	\centering
	\includegraphics[width=\textwidth]{./Figures/docMaestro.png}
	\caption{Definir memoria.tex como documento maestro.}
	\label{fig:docMaestro}
\end{figure}

En el menú herramientas se encuentran las opciones de compilación.  Para producir un archivo PDF a partir de un archivo .tex se debe ejecutar PDFLaTeX (el shortcut es F6). Para incorporar nueva bibliografía se debe utilizar la opción BibTeX del mismo menú herramientas (el shortcut es F11).

Notar que para actualizar las tablas de contenidos se debe ejecutar PDFLaTeX dos veces.  Esto se debe a que es necesario actualizar algunos archivos auxiliares antes de obtener el resultado final.  En forma similar, para actualizar las referencias bibliográficas se debe ejecutar primero PDFLaTeX, después BibTeX y finalmente PDFLaTeX dos veces por idénticos motivos.

\section{Personalizando la plantilla, el archivo \file{memoria.tex}}
\label{sec:FillingFile}

Para personalizar la plantilla se debe incorporar la información propia en los distintos archivos \file{.tex}. 

Primero abrir \file{memoria.tex} con TexMaker (o el editor de su preferencia). Se debe ubicar dentro del archivo el bloque de código titulado \emph{INFORMACIÓN DE LA PORTADA} donde se deben incorporar los primeros datos personales con los que se construirá automáticamente la portada.


%----------------------------------------------------------------------------------------

\section{El código del archivo \file{memoria.tex} explicado}

El archivo \file{memoria.tex} contiene la estructura del documento y es el archivo de mayor jerarquía de la memoria.  Podría ser equiparable a la función \emph{main()} de un programa en C, o mejor dicho al archivo fuente .c donde se encuentra definida la función main().

La estructura básica de cualquier documento de \LaTeX{} comienza con la definición de clase del documento, es seguida por un preámbulo donde se pueden agregar funcionalidades con el uso de \texttt{paquetes} (equiparables a bibliotecas de C), y finalmente, termina con el cuerpo del documento, donde irá el contenido de la memoria.

\lstset{%
  basicstyle=\small\ttfamily,
  language=[LaTeX]{TeX}
}

\begin{lstlisting}
\documentclass{article}  <- Definicion de clase
\usepackage{listings}	 <- Preambulo

\begin{document}	 <- Comienzo del contenido propio 
	Hello world!
\end{document}
\end{lstlisting}


El archivo \file{memoria.tex} se encuentra densamente comentado para explicar qué páginas, secciones y elementos de formato está creando el código \LaTeX{} en cada línea. El código está dividido en bloques con nombres en mayúsculas para que resulte evidente qué es lo que hace esa porción de código en particular. Inicialmente puede parecer que hay mucho código \LaTeX{}, pero es principalmente código para dar formato a la memoria por lo que no requiere intervención del usuario de la plantilla.  Sí se deben personalizar con su información los bloques indicados como:

\begin{itemize}
	\item Informacion de la memoria
	\item Resumen
	\item Agradecimientos
	\item Dedicatoria
\end{itemize}

El índice de contenidos, las listas de figura de tablas se generan en forma automática y no requieren intervención ni edición manual por parte del usuario de la plantilla. 

En la parte final del documento se encuentran los capítulos y los apéndices.  Por defecto se incluyen los 5 capítulos propuestos que se encuentran en la carpeta /Chapters. Cada capítulo se debe escribir en un archivo .tex separado y se debe poner en la carpeta \emph{Chapters} con el nombre \file{Chapter1}, \file{Chapter2}, etc\ldots El código para incluir capítulos desde archivos externos se muestra a continuación.

\begin{verbatim}
	% Chapter 1

\chapter{Introducción general} % Main chapter title

\label{Chapter1} % For referencing the chapter elsewhere, use \ref{Chapter1} 
\label{IntroGeneral}

En este capítulo se introduce a la problemática y se interpreta la importancia que implica el proyecto. 
Luego, se realiza un análisis del estado de arte sobre la gestión eficiente de salud de cultivos
y se puntualizan los objetivos y el alcance del trabajo.

%----------------------------------------------------------------------------------------

% Define some commands to keep the formatting separated from the content 
\newcommand{\keyword}[1]{\textbf{#1}}
\newcommand{\tabhead}[1]{\textbf{#1}}
\newcommand{\code}[1]{\texttt{#1}}
\newcommand{\file}[1]{\texttt{\bfseries#1}}
\newcommand{\option}[1]{\texttt{\itshape#1}}
\newcommand{\grados}{$^{\circ}$}

%----------------------------------------------------------------------------------------

%\section{Introducción}

%----------------------------------------------------------------------------------------
\section{Contexto}

La gestión eficiente de los montes frutales enfrenta varios desafíos significativos, entre los que se 
incluye la necesidad de monitoreo constante para asegurar la salud y productividad de los cultivos.
Los métodos tradicionales de inspección y evaluación, que dependen en gran medida de visitas de campo 
y observaciones manuales, son laboriosos, costosos y a menudo limitados en alcance y frecuencia. 
Estos métodos también pueden ser subjetivos, dependiendo de la experiencia y percepción del observador,
lo que puede llevar a inconsistencias en la evaluación. Además, factores como el cambio climático y
la variabilidad en las condiciones meteorológicas añaden capas adicionales de complejidad a la gestión
agronómica, ya que afectan directamente la salud y el rendimiento de los cultivos.

En este contexto, las imágenes satelitales emergen como una herramienta poderosa para abordar estos
problemas. Estas imágenes proporcionan datos de observación de la Tierra que son consistentes, 
repetitivos y de amplia cobertura, lo cual es esencial para un monitoreo efectivo. A través de 
imágenes satelitales, se pueden obtener índices de vegetación, como el NDVI (Índice de Vegetación 
de Diferencia Normalizada), que permiten evaluar la vegetación de manera precisa y objetiva.
Estos índices son cruciales para detectar cambios en la vegetación mucho antes de que sean
visibles a simple vista.  

Para este proyecto, se pretende automatizar estas tareas con una herramienta de fácil acceso 
a los usuarios, donde además de poder obtener mejores resultados, se facilitará el labor y 
tiempo que requieren las mismas.

\subsection{sec}

Si sos nuevo en \LaTeX{}, hay un muy buen libro electrónico - disponible gratuitamente en Internet como un archivo PDF - llamado, \enquote{A (not so short) Introduction to \LaTeX{}}. El título del libro es generalmente acortado a simplemente \emph{lshort}. Puede descargar la versión más reciente en inglés (ya que se actualiza de vez en cuando) desde aquí:
\url{http://www.ctan.org/tex-archive/info/lshort/english/lshort.pdf}

Se puede encontrar la versión en español en la lista en esta página: \url{http://www.ctan.org/tex-archive/info/lshort/}

\subsubsection{Una subsubsección}

Acá tiene un ejemplo de una ``subsubsección'' que es el cuarto nivel de ordenamiento del texto, después de capítulo, sección y subsección.  Como se puede ver, las subsubsecciones no van numeradas en el cuerpo del documento ni en el índice.  El formato está definido por la plantilla y no debe ser modificado.

\subsection{sec}

Si estás escribiendo un documento con mucho contenido matemático, entonces es posible que desees leer el documento de la AMS (American Mathematical Society) llamado, \enquote{A Short Math Guide for \LaTeX{}}. Se puede encontrar en línea en el siguiente link: \url{http://www.ams.org/tex/amslatex.html} en la sección \enquote{Additional Documentation} hacia la parte inferior de la página.


%----------------------------------------------------------------------------------------

\section{Estado del arte}

Si estás familiarizado con \LaTeX{}, entonces podés explorar la estructura de directorios de esta plantilla y proceder a personalizarla agregando tu información en el bloque \emph{INFORMACIÓN DE LA PORTADA} en el archivo \file{memoria.tex}.  

Se puede continuar luego modificando el resto de los archivos siguiendo los lineamientos que se describen en la sección \ref{sec:FillingFile} en la página \pageref{sec:FillingFile}.

Debés asegurarte de leer el capítulo \ref{Chapter2} acerca de las convenciones utilizadas para las Memoria de los Trabajos Finales de la \degreename.

Si sos nuevo en \LaTeX{}, se recomienda que continúes leyendo el documento ya que contiene información básica para aprovechar el potencial de esta herramienta.


%----------------------------------------------------------------------------------------

\section{Alcance y objetivos}

blablabla ...
\subsection{Propósito del proyecto}

El objetivo principal del proyecto consistió en desarrollar un algoritmo que permita descargar y 
analizar imágenes satelitales para vincularlas con características específicas de durazneros. 
Específicamente, determinar, a partir de los datos disponibles, el progreso de las etapas de 
floración y maduración de los frutos en el árbol. Este trabajo implicó la aplicación de distintas 
arquitecturas de modelos de aprendizaje automático y redes neuronales artificiales. Los indicadores
clave para medir el éxito del trabajo es el aumento de las métricas que indican el rendimiento 
de los modelos ejecutados. 

\subsection{Alcance del proyecto}

A continuación, se detallan las actividades incluidas en este trabajo:

\begin{itemize}
  \item La evaluación de las diferentes fuentes de datos para la realización de este trabajo.
  \begin{itemize}
    \item Evaluación de datos tabulares que corresponden mediciones a 
    campo de las etapas de floración y maduración de frutos, obtenidos durante 5 años.
    \item Evaluación de imágenes satelitales correspondientes a los lotes de durazneros.
  \end{itemize}
  \item La determinación del progreso de las etapas de floración y maduración de los frutos en el
  duraznero.
  \item El desarrollo de una herramienta de fácil acceso para el equipo del cliente.
  \item La elaboración de un informe que detalle el procedimiento realizado y resultados.
\end{itemize}

Los siguientes elementos quedan fuera del alcance:

\begin{itemize}
  \item El desarrollo de una interfaz para el sistema.
  \item El despligue del desarrollo en producción.
  \item La difusión de los datos, son confidenciales y propiedad del INTA.
\end{itemize}

%----------------------------------------------------------------------------------------

\section{Entorno de trabajo}

Ante de comenzar a editar la plantilla debemos tener un editor \LaTeX{} instalado en nuestra computadora.  En forma análoga a lo que sucede en lenguaje C, que se puede crear y editar código con casi cualquier editor, existen ciertos entornos de trabajo que nos pueden simplificar mucho la tarea.  En este sentido, se recomienda, sobre todo para los principiantes en \LaTeX{} la utilización de TexMaker, un programa gratuito y multi-plantaforma que está disponible tanto para windows como para sistemas GNU/linux.

La versión más reciente de TexMaker es la 4.5 y se puede descargar del siguiente link: \url{http://www.xm1math.net/texmaker/download.html}. Se puede consultar el manual de usuario en el siguiente link: \url{http://www.xm1math.net/texmaker/doc.html}.
 

\subsection{Paquetes adicionales}

Si bien durante el proceso de instalación de TexMaker, o cualquier otro editor que se haya elegido, se instalarán en el sistema los paquetes básicos necesarios para trabajar con \LaTeX{}, la plantilla de los trabajos de Especialización y Maestría requieren de paquete adicionales.

Se indican a continuación los comandos que se deben introducir en la consola de Ubuntu (ctrl + alt + t) para instalarlos:

\begin{lstlisting}[language=bash]
  $ sudo apt install texlive-lang-spanish texlive-science 
  $ sudo apt install texlive-bibtex-extra biber
  $ sudo apt install texlive texlive-fonts-recommended
  $ sudo apt install texlive-latex-extra
\end{lstlisting}


\subsection{Configurando TexMaker}
\label{subsec:configurando}



Una vez instalado el programa y los paquetes adicionales se debe abrir el archivo memoria.tex con el editor para ver una pantalla similar a la que se puede apreciar en la figura \ref{fig:texmaker}. 
Una vez instalado el programa y los paquetes adicionales se debe abrir el archivo memoria.tex con el editor para ver una pantalla similar a la que se puede apreciar en la figura \ref{fig:texmaker}. 
Una vez instalado el programa y los paquetes adicionales se debe abrir el archivo memoria.tex con el editor para ver una pantalla similar a la que se puede apreciar en la figura \ref{fig:texmaker}. 
Una vez instalado el programa y los paquetes adicionales se debe abrir el archivo memoria.tex con el editor para ver una pantalla similar a la que se puede apreciar en la figura \ref{fig:texmaker}. 

\vspace{1cm}

\begin{figure}[htbp]
	\centering
	\includegraphics[width=.5\textwidth]{./Figures/texmaker.png}
	\caption{Entorno de trabajo de texMaker.}
	\label{fig:texmaker}
\end{figure}

\vspace{1cm}

Notar que existe una vista llamada Estructura a la izquierda de la interfaz que nos permite abrir desde dentro del programa los archivos individuales de los capítulos.  A la derecha se encuentra una vista con el archivo propiamente dicho para su edición. Hacia la parte inferior se encuentra una vista del log con información de los resultados de la compilación.  En esta última vista pueden aparecen advertencias o \textit{warning}, que normalmente pueden ser ignorados, y los errores que se indican en color rojo y deben resolverse para que se genere el PDF de salida.

Recordar que el archivo que se debe compilar con PDFLaTeX es \file{memoria.tex}, si se tratara de compilar alguno de los capítulos saldría un error.  Para salvar la molestia de tener que cambiar de archivo para compilar cada vez que se realice una modificación en un capítulo, se puede definir el archivo \file{memoria.tex} como ``documento maestro'' yendo al menú opciones -> ``definir documento actual como documento maestro'', lo que permite compilar con PDFLaTeX memoria.tex directamente desde cualquier archivo que se esté modificando . Se muestra esta opción en la figura \ref{fig:docMaestro}.

\begin{figure}[ht]
	\centering
	\includegraphics[width=\textwidth]{./Figures/docMaestro.png}
	\caption{Definir memoria.tex como documento maestro.}
	\label{fig:docMaestro}
\end{figure}

En el menú herramientas se encuentran las opciones de compilación.  Para producir un archivo PDF a partir de un archivo .tex se debe ejecutar PDFLaTeX (el shortcut es F6). Para incorporar nueva bibliografía se debe utilizar la opción BibTeX del mismo menú herramientas (el shortcut es F11).

Notar que para actualizar las tablas de contenidos se debe ejecutar PDFLaTeX dos veces.  Esto se debe a que es necesario actualizar algunos archivos auxiliares antes de obtener el resultado final.  En forma similar, para actualizar las referencias bibliográficas se debe ejecutar primero PDFLaTeX, después BibTeX y finalmente PDFLaTeX dos veces por idénticos motivos.

\section{Personalizando la plantilla, el archivo \file{memoria.tex}}
\label{sec:FillingFile}

Para personalizar la plantilla se debe incorporar la información propia en los distintos archivos \file{.tex}. 

Primero abrir \file{memoria.tex} con TexMaker (o el editor de su preferencia). Se debe ubicar dentro del archivo el bloque de código titulado \emph{INFORMACIÓN DE LA PORTADA} donde se deben incorporar los primeros datos personales con los que se construirá automáticamente la portada.


%----------------------------------------------------------------------------------------

\section{El código del archivo \file{memoria.tex} explicado}

El archivo \file{memoria.tex} contiene la estructura del documento y es el archivo de mayor jerarquía de la memoria.  Podría ser equiparable a la función \emph{main()} de un programa en C, o mejor dicho al archivo fuente .c donde se encuentra definida la función main().

La estructura básica de cualquier documento de \LaTeX{} comienza con la definición de clase del documento, es seguida por un preámbulo donde se pueden agregar funcionalidades con el uso de \texttt{paquetes} (equiparables a bibliotecas de C), y finalmente, termina con el cuerpo del documento, donde irá el contenido de la memoria.

\lstset{%
  basicstyle=\small\ttfamily,
  language=[LaTeX]{TeX}
}

\begin{lstlisting}
\documentclass{article}  <- Definicion de clase
\usepackage{listings}	 <- Preambulo

\begin{document}	 <- Comienzo del contenido propio 
	Hello world!
\end{document}
\end{lstlisting}


El archivo \file{memoria.tex} se encuentra densamente comentado para explicar qué páginas, secciones y elementos de formato está creando el código \LaTeX{} en cada línea. El código está dividido en bloques con nombres en mayúsculas para que resulte evidente qué es lo que hace esa porción de código en particular. Inicialmente puede parecer que hay mucho código \LaTeX{}, pero es principalmente código para dar formato a la memoria por lo que no requiere intervención del usuario de la plantilla.  Sí se deben personalizar con su información los bloques indicados como:

\begin{itemize}
	\item Informacion de la memoria
	\item Resumen
	\item Agradecimientos
	\item Dedicatoria
\end{itemize}

El índice de contenidos, las listas de figura de tablas se generan en forma automática y no requieren intervención ni edición manual por parte del usuario de la plantilla. 

En la parte final del documento se encuentran los capítulos y los apéndices.  Por defecto se incluyen los 5 capítulos propuestos que se encuentran en la carpeta /Chapters. Cada capítulo se debe escribir en un archivo .tex separado y se debe poner en la carpeta \emph{Chapters} con el nombre \file{Chapter1}, \file{Chapter2}, etc\ldots El código para incluir capítulos desde archivos externos se muestra a continuación.

\begin{verbatim}
	% Chapter 1

\chapter{Introducción general} % Main chapter title

\label{Chapter1} % For referencing the chapter elsewhere, use \ref{Chapter1} 
\label{IntroGeneral}

En este capítulo se introduce a la problemática y se interpreta la importancia que implica el proyecto. 
Luego, se realiza un análisis del estado de arte sobre la gestión eficiente de salud de cultivos
y se puntualizan los objetivos y el alcance del trabajo.

%----------------------------------------------------------------------------------------

% Define some commands to keep the formatting separated from the content 
\newcommand{\keyword}[1]{\textbf{#1}}
\newcommand{\tabhead}[1]{\textbf{#1}}
\newcommand{\code}[1]{\texttt{#1}}
\newcommand{\file}[1]{\texttt{\bfseries#1}}
\newcommand{\option}[1]{\texttt{\itshape#1}}
\newcommand{\grados}{$^{\circ}$}

%----------------------------------------------------------------------------------------

%\section{Introducción}

%----------------------------------------------------------------------------------------
\section{Contexto}

La gestión eficiente de los montes frutales enfrenta varios desafíos significativos, entre los que se 
incluye la necesidad de monitoreo constante para asegurar la salud y productividad de los cultivos.
Los métodos tradicionales de inspección y evaluación, que dependen en gran medida de visitas de campo 
y observaciones manuales, son laboriosos, costosos y a menudo limitados en alcance y frecuencia. 
Estos métodos también pueden ser subjetivos, dependiendo de la experiencia y percepción del observador,
lo que puede llevar a inconsistencias en la evaluación. Además, factores como el cambio climático y
la variabilidad en las condiciones meteorológicas añaden capas adicionales de complejidad a la gestión
agronómica, ya que afectan directamente la salud y el rendimiento de los cultivos.

En este contexto, las imágenes satelitales emergen como una herramienta poderosa para abordar estos
problemas. Estas imágenes proporcionan datos de observación de la Tierra que son consistentes, 
repetitivos y de amplia cobertura, lo cual es esencial para un monitoreo efectivo. A través de 
imágenes satelitales, se pueden obtener índices de vegetación, como el NDVI (Índice de Vegetación 
de Diferencia Normalizada), que permiten evaluar la vegetación de manera precisa y objetiva.
Estos índices son cruciales para detectar cambios en la vegetación mucho antes de que sean
visibles a simple vista.  

Para este proyecto, se pretende automatizar estas tareas con una herramienta de fácil acceso 
a los usuarios, donde además de poder obtener mejores resultados, se facilitará el labor y 
tiempo que requieren las mismas.

\subsection{sec}

Si sos nuevo en \LaTeX{}, hay un muy buen libro electrónico - disponible gratuitamente en Internet como un archivo PDF - llamado, \enquote{A (not so short) Introduction to \LaTeX{}}. El título del libro es generalmente acortado a simplemente \emph{lshort}. Puede descargar la versión más reciente en inglés (ya que se actualiza de vez en cuando) desde aquí:
\url{http://www.ctan.org/tex-archive/info/lshort/english/lshort.pdf}

Se puede encontrar la versión en español en la lista en esta página: \url{http://www.ctan.org/tex-archive/info/lshort/}

\subsubsection{Una subsubsección}

Acá tiene un ejemplo de una ``subsubsección'' que es el cuarto nivel de ordenamiento del texto, después de capítulo, sección y subsección.  Como se puede ver, las subsubsecciones no van numeradas en el cuerpo del documento ni en el índice.  El formato está definido por la plantilla y no debe ser modificado.

\subsection{sec}

Si estás escribiendo un documento con mucho contenido matemático, entonces es posible que desees leer el documento de la AMS (American Mathematical Society) llamado, \enquote{A Short Math Guide for \LaTeX{}}. Se puede encontrar en línea en el siguiente link: \url{http://www.ams.org/tex/amslatex.html} en la sección \enquote{Additional Documentation} hacia la parte inferior de la página.


%----------------------------------------------------------------------------------------

\section{Estado del arte}

Si estás familiarizado con \LaTeX{}, entonces podés explorar la estructura de directorios de esta plantilla y proceder a personalizarla agregando tu información en el bloque \emph{INFORMACIÓN DE LA PORTADA} en el archivo \file{memoria.tex}.  

Se puede continuar luego modificando el resto de los archivos siguiendo los lineamientos que se describen en la sección \ref{sec:FillingFile} en la página \pageref{sec:FillingFile}.

Debés asegurarte de leer el capítulo \ref{Chapter2} acerca de las convenciones utilizadas para las Memoria de los Trabajos Finales de la \degreename.

Si sos nuevo en \LaTeX{}, se recomienda que continúes leyendo el documento ya que contiene información básica para aprovechar el potencial de esta herramienta.


%----------------------------------------------------------------------------------------

\section{Alcance y objetivos}

blablabla ...
\subsection{Propósito del proyecto}

El objetivo principal del proyecto consistió en desarrollar un algoritmo que permita descargar y 
analizar imágenes satelitales para vincularlas con características específicas de durazneros. 
Específicamente, determinar, a partir de los datos disponibles, el progreso de las etapas de 
floración y maduración de los frutos en el árbol. Este trabajo implicó la aplicación de distintas 
arquitecturas de modelos de aprendizaje automático y redes neuronales artificiales. Los indicadores
clave para medir el éxito del trabajo es el aumento de las métricas que indican el rendimiento 
de los modelos ejecutados. 

\subsection{Alcance del proyecto}

A continuación, se detallan las actividades incluidas en este trabajo:

\begin{itemize}
  \item La evaluación de las diferentes fuentes de datos para la realización de este trabajo.
  \begin{itemize}
    \item Evaluación de datos tabulares que corresponden mediciones a 
    campo de las etapas de floración y maduración de frutos, obtenidos durante 5 años.
    \item Evaluación de imágenes satelitales correspondientes a los lotes de durazneros.
  \end{itemize}
  \item La determinación del progreso de las etapas de floración y maduración de los frutos en el
  duraznero.
  \item El desarrollo de una herramienta de fácil acceso para el equipo del cliente.
  \item La elaboración de un informe que detalle el procedimiento realizado y resultados.
\end{itemize}

Los siguientes elementos quedan fuera del alcance:

\begin{itemize}
  \item El desarrollo de una interfaz para el sistema.
  \item El despligue del desarrollo en producción.
  \item La difusión de los datos, son confidenciales y propiedad del INTA.
\end{itemize}

%----------------------------------------------------------------------------------------

\section{Entorno de trabajo}

Ante de comenzar a editar la plantilla debemos tener un editor \LaTeX{} instalado en nuestra computadora.  En forma análoga a lo que sucede en lenguaje C, que se puede crear y editar código con casi cualquier editor, existen ciertos entornos de trabajo que nos pueden simplificar mucho la tarea.  En este sentido, se recomienda, sobre todo para los principiantes en \LaTeX{} la utilización de TexMaker, un programa gratuito y multi-plantaforma que está disponible tanto para windows como para sistemas GNU/linux.

La versión más reciente de TexMaker es la 4.5 y se puede descargar del siguiente link: \url{http://www.xm1math.net/texmaker/download.html}. Se puede consultar el manual de usuario en el siguiente link: \url{http://www.xm1math.net/texmaker/doc.html}.
 

\subsection{Paquetes adicionales}

Si bien durante el proceso de instalación de TexMaker, o cualquier otro editor que se haya elegido, se instalarán en el sistema los paquetes básicos necesarios para trabajar con \LaTeX{}, la plantilla de los trabajos de Especialización y Maestría requieren de paquete adicionales.

Se indican a continuación los comandos que se deben introducir en la consola de Ubuntu (ctrl + alt + t) para instalarlos:

\begin{lstlisting}[language=bash]
  $ sudo apt install texlive-lang-spanish texlive-science 
  $ sudo apt install texlive-bibtex-extra biber
  $ sudo apt install texlive texlive-fonts-recommended
  $ sudo apt install texlive-latex-extra
\end{lstlisting}


\subsection{Configurando TexMaker}
\label{subsec:configurando}



Una vez instalado el programa y los paquetes adicionales se debe abrir el archivo memoria.tex con el editor para ver una pantalla similar a la que se puede apreciar en la figura \ref{fig:texmaker}. 
Una vez instalado el programa y los paquetes adicionales se debe abrir el archivo memoria.tex con el editor para ver una pantalla similar a la que se puede apreciar en la figura \ref{fig:texmaker}. 
Una vez instalado el programa y los paquetes adicionales se debe abrir el archivo memoria.tex con el editor para ver una pantalla similar a la que se puede apreciar en la figura \ref{fig:texmaker}. 
Una vez instalado el programa y los paquetes adicionales se debe abrir el archivo memoria.tex con el editor para ver una pantalla similar a la que se puede apreciar en la figura \ref{fig:texmaker}. 

\vspace{1cm}

\begin{figure}[htbp]
	\centering
	\includegraphics[width=.5\textwidth]{./Figures/texmaker.png}
	\caption{Entorno de trabajo de texMaker.}
	\label{fig:texmaker}
\end{figure}

\vspace{1cm}

Notar que existe una vista llamada Estructura a la izquierda de la interfaz que nos permite abrir desde dentro del programa los archivos individuales de los capítulos.  A la derecha se encuentra una vista con el archivo propiamente dicho para su edición. Hacia la parte inferior se encuentra una vista del log con información de los resultados de la compilación.  En esta última vista pueden aparecen advertencias o \textit{warning}, que normalmente pueden ser ignorados, y los errores que se indican en color rojo y deben resolverse para que se genere el PDF de salida.

Recordar que el archivo que se debe compilar con PDFLaTeX es \file{memoria.tex}, si se tratara de compilar alguno de los capítulos saldría un error.  Para salvar la molestia de tener que cambiar de archivo para compilar cada vez que se realice una modificación en un capítulo, se puede definir el archivo \file{memoria.tex} como ``documento maestro'' yendo al menú opciones -> ``definir documento actual como documento maestro'', lo que permite compilar con PDFLaTeX memoria.tex directamente desde cualquier archivo que se esté modificando . Se muestra esta opción en la figura \ref{fig:docMaestro}.

\begin{figure}[ht]
	\centering
	\includegraphics[width=\textwidth]{./Figures/docMaestro.png}
	\caption{Definir memoria.tex como documento maestro.}
	\label{fig:docMaestro}
\end{figure}

En el menú herramientas se encuentran las opciones de compilación.  Para producir un archivo PDF a partir de un archivo .tex se debe ejecutar PDFLaTeX (el shortcut es F6). Para incorporar nueva bibliografía se debe utilizar la opción BibTeX del mismo menú herramientas (el shortcut es F11).

Notar que para actualizar las tablas de contenidos se debe ejecutar PDFLaTeX dos veces.  Esto se debe a que es necesario actualizar algunos archivos auxiliares antes de obtener el resultado final.  En forma similar, para actualizar las referencias bibliográficas se debe ejecutar primero PDFLaTeX, después BibTeX y finalmente PDFLaTeX dos veces por idénticos motivos.

\section{Personalizando la plantilla, el archivo \file{memoria.tex}}
\label{sec:FillingFile}

Para personalizar la plantilla se debe incorporar la información propia en los distintos archivos \file{.tex}. 

Primero abrir \file{memoria.tex} con TexMaker (o el editor de su preferencia). Se debe ubicar dentro del archivo el bloque de código titulado \emph{INFORMACIÓN DE LA PORTADA} donde se deben incorporar los primeros datos personales con los que se construirá automáticamente la portada.


%----------------------------------------------------------------------------------------

\section{El código del archivo \file{memoria.tex} explicado}

El archivo \file{memoria.tex} contiene la estructura del documento y es el archivo de mayor jerarquía de la memoria.  Podría ser equiparable a la función \emph{main()} de un programa en C, o mejor dicho al archivo fuente .c donde se encuentra definida la función main().

La estructura básica de cualquier documento de \LaTeX{} comienza con la definición de clase del documento, es seguida por un preámbulo donde se pueden agregar funcionalidades con el uso de \texttt{paquetes} (equiparables a bibliotecas de C), y finalmente, termina con el cuerpo del documento, donde irá el contenido de la memoria.

\lstset{%
  basicstyle=\small\ttfamily,
  language=[LaTeX]{TeX}
}

\begin{lstlisting}
\documentclass{article}  <- Definicion de clase
\usepackage{listings}	 <- Preambulo

\begin{document}	 <- Comienzo del contenido propio 
	Hello world!
\end{document}
\end{lstlisting}


El archivo \file{memoria.tex} se encuentra densamente comentado para explicar qué páginas, secciones y elementos de formato está creando el código \LaTeX{} en cada línea. El código está dividido en bloques con nombres en mayúsculas para que resulte evidente qué es lo que hace esa porción de código en particular. Inicialmente puede parecer que hay mucho código \LaTeX{}, pero es principalmente código para dar formato a la memoria por lo que no requiere intervención del usuario de la plantilla.  Sí se deben personalizar con su información los bloques indicados como:

\begin{itemize}
	\item Informacion de la memoria
	\item Resumen
	\item Agradecimientos
	\item Dedicatoria
\end{itemize}

El índice de contenidos, las listas de figura de tablas se generan en forma automática y no requieren intervención ni edición manual por parte del usuario de la plantilla. 

En la parte final del documento se encuentran los capítulos y los apéndices.  Por defecto se incluyen los 5 capítulos propuestos que se encuentran en la carpeta /Chapters. Cada capítulo se debe escribir en un archivo .tex separado y se debe poner en la carpeta \emph{Chapters} con el nombre \file{Chapter1}, \file{Chapter2}, etc\ldots El código para incluir capítulos desde archivos externos se muestra a continuación.

\begin{verbatim}
	% Chapter 1

\chapter{Introducción general} % Main chapter title

\label{Chapter1} % For referencing the chapter elsewhere, use \ref{Chapter1} 
\label{IntroGeneral}

En este capítulo se introduce a la problemática y se interpreta la importancia que implica el proyecto. 
Luego, se realiza un análisis del estado de arte sobre la gestión eficiente de salud de cultivos
y se puntualizan los objetivos y el alcance del trabajo.

%----------------------------------------------------------------------------------------

% Define some commands to keep the formatting separated from the content 
\newcommand{\keyword}[1]{\textbf{#1}}
\newcommand{\tabhead}[1]{\textbf{#1}}
\newcommand{\code}[1]{\texttt{#1}}
\newcommand{\file}[1]{\texttt{\bfseries#1}}
\newcommand{\option}[1]{\texttt{\itshape#1}}
\newcommand{\grados}{$^{\circ}$}

%----------------------------------------------------------------------------------------

%\section{Introducción}

%----------------------------------------------------------------------------------------
\section{Contexto}

La gestión eficiente de los montes frutales enfrenta varios desafíos significativos, entre los que se 
incluye la necesidad de monitoreo constante para asegurar la salud y productividad de los cultivos.
Los métodos tradicionales de inspección y evaluación, que dependen en gran medida de visitas de campo 
y observaciones manuales, son laboriosos, costosos y a menudo limitados en alcance y frecuencia. 
Estos métodos también pueden ser subjetivos, dependiendo de la experiencia y percepción del observador,
lo que puede llevar a inconsistencias en la evaluación. Además, factores como el cambio climático y
la variabilidad en las condiciones meteorológicas añaden capas adicionales de complejidad a la gestión
agronómica, ya que afectan directamente la salud y el rendimiento de los cultivos.

En este contexto, las imágenes satelitales emergen como una herramienta poderosa para abordar estos
problemas. Estas imágenes proporcionan datos de observación de la Tierra que son consistentes, 
repetitivos y de amplia cobertura, lo cual es esencial para un monitoreo efectivo. A través de 
imágenes satelitales, se pueden obtener índices de vegetación, como el NDVI (Índice de Vegetación 
de Diferencia Normalizada), que permiten evaluar la vegetación de manera precisa y objetiva.
Estos índices son cruciales para detectar cambios en la vegetación mucho antes de que sean
visibles a simple vista.  

Para este proyecto, se pretende automatizar estas tareas con una herramienta de fácil acceso 
a los usuarios, donde además de poder obtener mejores resultados, se facilitará el labor y 
tiempo que requieren las mismas.

\subsection{sec}

Si sos nuevo en \LaTeX{}, hay un muy buen libro electrónico - disponible gratuitamente en Internet como un archivo PDF - llamado, \enquote{A (not so short) Introduction to \LaTeX{}}. El título del libro es generalmente acortado a simplemente \emph{lshort}. Puede descargar la versión más reciente en inglés (ya que se actualiza de vez en cuando) desde aquí:
\url{http://www.ctan.org/tex-archive/info/lshort/english/lshort.pdf}

Se puede encontrar la versión en español en la lista en esta página: \url{http://www.ctan.org/tex-archive/info/lshort/}

\subsubsection{Una subsubsección}

Acá tiene un ejemplo de una ``subsubsección'' que es el cuarto nivel de ordenamiento del texto, después de capítulo, sección y subsección.  Como se puede ver, las subsubsecciones no van numeradas en el cuerpo del documento ni en el índice.  El formato está definido por la plantilla y no debe ser modificado.

\subsection{sec}

Si estás escribiendo un documento con mucho contenido matemático, entonces es posible que desees leer el documento de la AMS (American Mathematical Society) llamado, \enquote{A Short Math Guide for \LaTeX{}}. Se puede encontrar en línea en el siguiente link: \url{http://www.ams.org/tex/amslatex.html} en la sección \enquote{Additional Documentation} hacia la parte inferior de la página.


%----------------------------------------------------------------------------------------

\section{Estado del arte}

Si estás familiarizado con \LaTeX{}, entonces podés explorar la estructura de directorios de esta plantilla y proceder a personalizarla agregando tu información en el bloque \emph{INFORMACIÓN DE LA PORTADA} en el archivo \file{memoria.tex}.  

Se puede continuar luego modificando el resto de los archivos siguiendo los lineamientos que se describen en la sección \ref{sec:FillingFile} en la página \pageref{sec:FillingFile}.

Debés asegurarte de leer el capítulo \ref{Chapter2} acerca de las convenciones utilizadas para las Memoria de los Trabajos Finales de la \degreename.

Si sos nuevo en \LaTeX{}, se recomienda que continúes leyendo el documento ya que contiene información básica para aprovechar el potencial de esta herramienta.


%----------------------------------------------------------------------------------------

\section{Alcance y objetivos}

blablabla ...
\subsection{Propósito del proyecto}

El objetivo principal del proyecto consistió en desarrollar un algoritmo que permita descargar y 
analizar imágenes satelitales para vincularlas con características específicas de durazneros. 
Específicamente, determinar, a partir de los datos disponibles, el progreso de las etapas de 
floración y maduración de los frutos en el árbol. Este trabajo implicó la aplicación de distintas 
arquitecturas de modelos de aprendizaje automático y redes neuronales artificiales. Los indicadores
clave para medir el éxito del trabajo es el aumento de las métricas que indican el rendimiento 
de los modelos ejecutados. 

\subsection{Alcance del proyecto}

A continuación, se detallan las actividades incluidas en este trabajo:

\begin{itemize}
  \item La evaluación de las diferentes fuentes de datos para la realización de este trabajo.
  \begin{itemize}
    \item Evaluación de datos tabulares que corresponden mediciones a 
    campo de las etapas de floración y maduración de frutos, obtenidos durante 5 años.
    \item Evaluación de imágenes satelitales correspondientes a los lotes de durazneros.
  \end{itemize}
  \item La determinación del progreso de las etapas de floración y maduración de los frutos en el
  duraznero.
  \item El desarrollo de una herramienta de fácil acceso para el equipo del cliente.
  \item La elaboración de un informe que detalle el procedimiento realizado y resultados.
\end{itemize}

Los siguientes elementos quedan fuera del alcance:

\begin{itemize}
  \item El desarrollo de una interfaz para el sistema.
  \item El despligue del desarrollo en producción.
  \item La difusión de los datos, son confidenciales y propiedad del INTA.
\end{itemize}

%----------------------------------------------------------------------------------------

\section{Entorno de trabajo}

Ante de comenzar a editar la plantilla debemos tener un editor \LaTeX{} instalado en nuestra computadora.  En forma análoga a lo que sucede en lenguaje C, que se puede crear y editar código con casi cualquier editor, existen ciertos entornos de trabajo que nos pueden simplificar mucho la tarea.  En este sentido, se recomienda, sobre todo para los principiantes en \LaTeX{} la utilización de TexMaker, un programa gratuito y multi-plantaforma que está disponible tanto para windows como para sistemas GNU/linux.

La versión más reciente de TexMaker es la 4.5 y se puede descargar del siguiente link: \url{http://www.xm1math.net/texmaker/download.html}. Se puede consultar el manual de usuario en el siguiente link: \url{http://www.xm1math.net/texmaker/doc.html}.
 

\subsection{Paquetes adicionales}

Si bien durante el proceso de instalación de TexMaker, o cualquier otro editor que se haya elegido, se instalarán en el sistema los paquetes básicos necesarios para trabajar con \LaTeX{}, la plantilla de los trabajos de Especialización y Maestría requieren de paquete adicionales.

Se indican a continuación los comandos que se deben introducir en la consola de Ubuntu (ctrl + alt + t) para instalarlos:

\begin{lstlisting}[language=bash]
  $ sudo apt install texlive-lang-spanish texlive-science 
  $ sudo apt install texlive-bibtex-extra biber
  $ sudo apt install texlive texlive-fonts-recommended
  $ sudo apt install texlive-latex-extra
\end{lstlisting}


\subsection{Configurando TexMaker}
\label{subsec:configurando}



Una vez instalado el programa y los paquetes adicionales se debe abrir el archivo memoria.tex con el editor para ver una pantalla similar a la que se puede apreciar en la figura \ref{fig:texmaker}. 
Una vez instalado el programa y los paquetes adicionales se debe abrir el archivo memoria.tex con el editor para ver una pantalla similar a la que se puede apreciar en la figura \ref{fig:texmaker}. 
Una vez instalado el programa y los paquetes adicionales se debe abrir el archivo memoria.tex con el editor para ver una pantalla similar a la que se puede apreciar en la figura \ref{fig:texmaker}. 
Una vez instalado el programa y los paquetes adicionales se debe abrir el archivo memoria.tex con el editor para ver una pantalla similar a la que se puede apreciar en la figura \ref{fig:texmaker}. 

\vspace{1cm}

\begin{figure}[htbp]
	\centering
	\includegraphics[width=.5\textwidth]{./Figures/texmaker.png}
	\caption{Entorno de trabajo de texMaker.}
	\label{fig:texmaker}
\end{figure}

\vspace{1cm}

Notar que existe una vista llamada Estructura a la izquierda de la interfaz que nos permite abrir desde dentro del programa los archivos individuales de los capítulos.  A la derecha se encuentra una vista con el archivo propiamente dicho para su edición. Hacia la parte inferior se encuentra una vista del log con información de los resultados de la compilación.  En esta última vista pueden aparecen advertencias o \textit{warning}, que normalmente pueden ser ignorados, y los errores que se indican en color rojo y deben resolverse para que se genere el PDF de salida.

Recordar que el archivo que se debe compilar con PDFLaTeX es \file{memoria.tex}, si se tratara de compilar alguno de los capítulos saldría un error.  Para salvar la molestia de tener que cambiar de archivo para compilar cada vez que se realice una modificación en un capítulo, se puede definir el archivo \file{memoria.tex} como ``documento maestro'' yendo al menú opciones -> ``definir documento actual como documento maestro'', lo que permite compilar con PDFLaTeX memoria.tex directamente desde cualquier archivo que se esté modificando . Se muestra esta opción en la figura \ref{fig:docMaestro}.

\begin{figure}[ht]
	\centering
	\includegraphics[width=\textwidth]{./Figures/docMaestro.png}
	\caption{Definir memoria.tex como documento maestro.}
	\label{fig:docMaestro}
\end{figure}

En el menú herramientas se encuentran las opciones de compilación.  Para producir un archivo PDF a partir de un archivo .tex se debe ejecutar PDFLaTeX (el shortcut es F6). Para incorporar nueva bibliografía se debe utilizar la opción BibTeX del mismo menú herramientas (el shortcut es F11).

Notar que para actualizar las tablas de contenidos se debe ejecutar PDFLaTeX dos veces.  Esto se debe a que es necesario actualizar algunos archivos auxiliares antes de obtener el resultado final.  En forma similar, para actualizar las referencias bibliográficas se debe ejecutar primero PDFLaTeX, después BibTeX y finalmente PDFLaTeX dos veces por idénticos motivos.

\section{Personalizando la plantilla, el archivo \file{memoria.tex}}
\label{sec:FillingFile}

Para personalizar la plantilla se debe incorporar la información propia en los distintos archivos \file{.tex}. 

Primero abrir \file{memoria.tex} con TexMaker (o el editor de su preferencia). Se debe ubicar dentro del archivo el bloque de código titulado \emph{INFORMACIÓN DE LA PORTADA} donde se deben incorporar los primeros datos personales con los que se construirá automáticamente la portada.


%----------------------------------------------------------------------------------------

\section{El código del archivo \file{memoria.tex} explicado}

El archivo \file{memoria.tex} contiene la estructura del documento y es el archivo de mayor jerarquía de la memoria.  Podría ser equiparable a la función \emph{main()} de un programa en C, o mejor dicho al archivo fuente .c donde se encuentra definida la función main().

La estructura básica de cualquier documento de \LaTeX{} comienza con la definición de clase del documento, es seguida por un preámbulo donde se pueden agregar funcionalidades con el uso de \texttt{paquetes} (equiparables a bibliotecas de C), y finalmente, termina con el cuerpo del documento, donde irá el contenido de la memoria.

\lstset{%
  basicstyle=\small\ttfamily,
  language=[LaTeX]{TeX}
}

\begin{lstlisting}
\documentclass{article}  <- Definicion de clase
\usepackage{listings}	 <- Preambulo

\begin{document}	 <- Comienzo del contenido propio 
	Hello world!
\end{document}
\end{lstlisting}


El archivo \file{memoria.tex} se encuentra densamente comentado para explicar qué páginas, secciones y elementos de formato está creando el código \LaTeX{} en cada línea. El código está dividido en bloques con nombres en mayúsculas para que resulte evidente qué es lo que hace esa porción de código en particular. Inicialmente puede parecer que hay mucho código \LaTeX{}, pero es principalmente código para dar formato a la memoria por lo que no requiere intervención del usuario de la plantilla.  Sí se deben personalizar con su información los bloques indicados como:

\begin{itemize}
	\item Informacion de la memoria
	\item Resumen
	\item Agradecimientos
	\item Dedicatoria
\end{itemize}

El índice de contenidos, las listas de figura de tablas se generan en forma automática y no requieren intervención ni edición manual por parte del usuario de la plantilla. 

En la parte final del documento se encuentran los capítulos y los apéndices.  Por defecto se incluyen los 5 capítulos propuestos que se encuentran en la carpeta /Chapters. Cada capítulo se debe escribir en un archivo .tex separado y se debe poner en la carpeta \emph{Chapters} con el nombre \file{Chapter1}, \file{Chapter2}, etc\ldots El código para incluir capítulos desde archivos externos se muestra a continuación.

\begin{verbatim}
	\include{Chapters/Chapter1}
	\include{Chapters/Chapter2} 
	\include{Chapters/Chapter3}
	\include{Chapters/Chapter4} 
	\include{Chapters/Chapter5} 
\end{verbatim}

Los apéndices también deben escribirse en archivos .tex separados, que se deben ubicar dentro de la carpeta \emph{Appendices}. Los apéndices vienen comentados por defecto con el caracter \code{\%} y para incluirlos simplemente se debe eliminar dicho caracter.

Finalmente, se encuentra el código para incluir la bibliografía en el documento final.  Este código tampoco debe modificarse. La metodología para trabajar las referencias bibliográficas se desarrolla en la sección \ref{sec:biblio}.
%----------------------------------------------------------------------------------------

\section{Bibliografía}
\label{sec:biblio}

Las opciones de formato de la bibliografía se controlan a través del paquete de latex \option{biblatex} que se incluye en la memoria en el archivo memoria.tex.  Estas opciones determinan cómo se generan las citas bibliográficas en el cuerpo del documento y cómo se genera la bibliografía al final de la memoria.

En el preámbulo se puede encontrar el código que incluye el paquete biblatex, que no requiere ninguna modificación del usuario de la plantilla, y que contiene las siguientes opciones:

\begin{lstlisting}
\usepackage[backend=bibtex,
	natbib=true, 
	style=numeric, 
	sorting=none]
{biblatex}
\end{lstlisting}

En el archivo \file{reference.bib} se encuentran las referencias bibliográficas que se pueden citar en el documento.  Para incorporar una nueva cita al documento lo primero es agregarla en este archivo con todos los campos necesario.  Todas las entradas bibliográficas comienzan con $@$ y una palabra que define el formato de la entrada.  Para cada formato existen campos obligatorios que deben completarse. No importa el orden en que las entradas estén definidas en el archivo .bib.  Tampoco es importante el orden en que estén definidos los campos de una entrada bibliográfica. A continuación se muestran algunos ejemplos:

\begin{lstlisting}
@ARTICLE{ARTICLE:1,
    AUTHOR="John Doe",
    TITLE="Title",
    JOURNAL="Journal",
    YEAR="2017",
}
\end{lstlisting}


\begin{lstlisting}
@BOOK{BOOK:1,
    AUTHOR="John Doe",
    TITLE="The Book without Title",
    PUBLISHER="Dummy Publisher",
    YEAR="2100",
}
\end{lstlisting}


\begin{lstlisting}
@INBOOK{BOOK:2,
    AUTHOR="John Doe",
    TITLE="The Book without Title",
    PUBLISHER="Dummy Publisher",
    YEAR="2100",
    PAGES="100-200",
}
\end{lstlisting}


\begin{lstlisting}
@MISC{WEBSITE:1,
    HOWPUBLISHED = "\url{http://example.com}",
    AUTHOR = "Intel",
    TITLE = "Example Website",
    MONTH = "12",
    YEAR = "1988",
    URLDATE = {2012-11-26}
}
\end{lstlisting}

Se debe notar que los nombres \emph{ARTICLE:1}, \emph{BOOK:1}, \emph{BOOK:2} y \emph{WEBSITE:1} son nombres de fantasía que le sirve al autor del documento para identificar la entrada. En este sentido, se podrían reemplazar por cualquier otro nombre.  Tampoco es necesario poner : seguido de un número, en los ejemplos sólo se incluye como un posible estilo para identificar las entradas.

La entradas se citan en el documento con el comando: 

\begin{verbatim}
\citep{nombre_de_la_entrada}
\end{verbatim}

Y cuando se usan, se muestran así: \citep{ARTICLE:1}, \citep{BOOK:1}, \citep{BOOK:2}, \citep{WEBSITE:1}.  Notar cómo se conforma la sección Bibliografía al final del documento.

Finalmente y como se mencionó en la subsección \ref{subsec:configurando}, para actualizar las referencias bibliográficas tanto en la sección bibliografía como las citas en el cuerpo del documento, se deben ejecutar las herramientas de compilación PDFLaTeX, BibTeX, PDFLaTeX, PDFLaTeX, en ese orden.  Este procedimiento debería resolver cualquier mensaje "Citation xxxxx on page x undefined".

	\chapter{Introducción específica} % Main chapter title

\label{Chapter2}

%----------------------------------------------------------------------------------------
%	SECTION 1
%----------------------------------------------------------------------------------------
En esta sección se aborda el estudio de redes neuronales profundas aplicadas al análisis y predicción en 
agricultura, con un enfoque en la floración del duraznero. Se presenta una comparación entre diferentes 
arquitecturas de \textit{deep learning}, se destacan sus ventajas y limitaciones en este contexto. Además, se 
exploran los índices de vegetación más relevantes, utilizados para monitorear el estado de la cobertura 
vegetal y mejorar la precisión de los modelos predictivos.

\section{Aprendizaje profundo}

El uso de arquitecturas de aprendizaje profundo ha demostrado ser efectivo permitiendo 
una gestión agrícola más precisa y eficiente. A continuación, se describen algunas de las principales
arquitecturas aplicadas en este ámbito:


\begin{enumerate}
	\item Redes Neuronales Convolucionales (CNN): son ampliamente utilizadas para analizar datos 
	visuales, como imágenes satelitales o fotografías de cultivos. En el contexto de la floración del duraznero,
	pueden identificar patrones y características en imágenes que indican el inicio del proceso de
	floración. Por ejemplo, un estudio implementó dos arquitecturas de CNN para detectar la severidad de 
	lesiones causadas por enfermedades en hojas de durazno, y obtuvo una precisión del 87\,\% \citep{Rodriguez2023}. 
	\item Redes Generativas Antagónicas (GAN): esta arquitectura de aprendizaje profundo está compuesta 
	por dos modelos neuronales que compiten entre sí: un generador, que crea datos sintéticos, y un 
	discriminador, que evalúa la autenticidad de los datos. Esta dinámica permite generar datos 
	realistas similares a los de entrenamiento. En el ámbito agrícola, se han utilizado para generar imágenes 
	sintéticas de cultivos, lo cual facilitó la ampliación de conjuntos de datos para entrenar otros modelos de aprendizaje 
	automático \citep{Goodfellow2014}.
	\item Redes Neuronales de Base Radial (RBF): han sido aplicadas en estudios agrícolas para modelar relaciones
	 no lineales entre variables climáticas y fenológicas. Poseen gran capacidad para aproximar funciones complejas.
	 Por ejemplo, un estudio empleó RBF para estimar la evapotranspiración de referencia, y demostró su capacidad
	para modelar relaciones no lineales entre variables climáticas y fenómenos agrícolas \citep{Cerv2012}.
\end{enumerate}
	
En la tabla \ref{tab:redes_dl} se presenta un cuadro comparativo de las distintas arquitecturas menciondas.

\begin{table}[h]
    \centering
    \caption{Comparación de arquitecturas de Deep Learning}
    \renewcommand{\arraystretch}{1} % Aumenta el espacio entre filas
    \begin{tabular}{|>{\raggedright}m{2cm}|m{5cm}|m{5cm}|} % Usa celdas de tamaño fijo
        \hline
        \textbf{\large Modelo} & \textbf{\large Ventaja} & \textbf{\large Desventaja} \\
        \hline
        \large CNN & \large Excelente para extracción de características espaciales, especialmente en imágenes. 
		& \large Requiere gran cantidad de datos etiquetados y es computacionalmente costoso. \\
        \hline
        \large GAN & \large Genera datos sintéticos realistas, útil en aumento de datos y generación de imágenes. 
		& \large Difícil de entrenar y puede sufrir de colapso de modo, lo que afecta la calidad de la generación. \\
        \hline
        \large RBF & \large Buena aproximación de funciones no lineales, útil en problemas de clasificación y regresión.
		 & \large No escala bien con grandes volúmenes de datos y su entrenamiento puede ser ineficiente. \\
        \hline
        \large RNN & \large Captura dependencias temporales en secuencias, aplicable en procesamiento de lenguaje natural 
		y series temporales. & \large Problemas con gradientes desaparecidos en largas secuencias, requiere técnicas como 
		LSTM o GRU para mejorar rendimiento.\\
        \hline
    \end{tabular}
    \label{tab:redes_dl}
\end{table}


	Al seleccionar la arquitectura adecuada para un modelo de deep learning en tareas como la predicción de
	floración de plantas, es crucial considerar varios factores, entre ellos la naturaleza de los datos, el tipo
	de problema a resolver y los recursos disponibles. Las redes neuronales convolucionales (CNN) son 
  	ideales para tareas relacionadas con el procesamiento de imágenes o datos espaciales. En cambio, 
  	las redes neuronales recurrentes (RNN) y sus variantes como LSTM, son más adecuadas para modelar 
  	secuencias temporales, como los datos fenológicos de las plantas. Por otro lado, las redes generativas
   	antagónicas (GAN) pueden ser útiles cuando se requiere generar datos sintéticos, especialmente 
   	cuando hay escasez de datos etiquetados. La comparación entre modelos permite identificar cuál se adapta
	mejor a los objetivos del trabajo, con el fin de maximizar la precisión y eficiencia del sistema, y 
	minimizando los errores o el sobreajuste. La elección adecuada y la experimentación con diferentes
	enfoques garantizan un análisis robusto y confiable, fundamental para obtener resultados
	aplicables en la práctica agrícola.

\section{Procesamiento de Imágenes}
	El procesamiento de imágenes satelitales se ha convertido en una herramienta esencial para diversos
	campos, incluyendo la agricultura, la gestión de recursos naturales, la monitorización ambiental y 
 	la planificación urbana. Actualmente, existen varios métodos y técnicas avanzadas utilizados para
  	analizar y extraer información útil de estas imágenes. 
	

\subsection{Índices de vegetación}

La composición espectral del flujo radiante que emana de la superficie terrestre 
proporciona información sobre las propiedades físicas del suelo, el agua y la vegetación 
en entornos terrestres. Las técnicas, modelos e índices de teledetección están diseñados 
para convertir esta información espectral en una forma fácilmente interpretable \citep{Bannari1995}.
La información obtenida por teledetección sobre el crecimiento, el vigor y la dinámica de
la vegetación terrestre puede ser de gran provecho para el monitoreo del medio ambiente, 
la conservación de la biodiversidad, la agricultura, la silvicultura, las infraestructuras 
verdes urbanas y otros campos relacionados.

A continuación se presentan índices de vegetación relevantes para este trabajo:

\begin{itemize}
	\item \textit{Normalized Difference Vegetation Index} (NDVI) \citep{Xue2017}: se calcula como 
	   	relación normalizada entre las bandas roja e infrarroja cercana. Sus valores van en el rango
	   	entre 0 y 1. Tiene una reacción sensible a la vegetación verde, incluso en zonas con cobertura 
		vegetal escasa. El NDVI es sensible a los efectos del brillo y color del suelo,
		la atmósfera, las nubes y la sombra de las nubes, y la sombra del dosel foliar. Por ello su
		aplicación requiere la calibración de la teledetección. Se expresa como en la ecuación \ref{eq:ndvi}:
	
		\begin{equation}
			\label{eq:ndvi}
			NDVI = \left( \frac{p_{NIR} - p_{R}}{p_{NIR}} \right) + p_{R}
		\end{equation}

	\item \textit{Enhanced Vegetation Index} (EVI) \citep{Xue2017}: actúa como un parámetro que 
		corrige simultáneamente los efectos del suelo y de la atmósfera. Su formulación incluye los valores de NIR,
		 $R$ y $B$, previamente corregidos por la atmósfera. El término $L$ representa un parámetro de ajuste asociado
		  al suelo, cuyo valor se establece en 1. Además, se incorporan dos parámetros constantes con valores de 6 y 
		  7.5, respectivamente. La expresión matemática del índice se presenta en la ecuación \ref{eq:evi}:

		\begin{equation}
			\label{eq:evi}
			EVI = 2.5 \times \frac{(P_n - P_r)}{P_n + C_1 P_r - C_2 P_b + L}
		\end{equation}

	\item \textit{Atmospherically Resistant Vegetation Index} (ARVI) \citep{Xue2017}: se utiliza habitualmente para eliminar los efectos de los
		aerosoles atmosféricos. Se basa en el supuesto de que la atmósfera afecta significativamente a $R$ en comparación con el 
		NIR y puede reducir eficazmente la dependencia de este índice de vegetación de los efectos atmosféricos. En su fórmula 
		$RB$ es la diferencia entre $B$ y $R$, y está relacionada con la reflectancia influenciada por la dispersión molecular y la absorción gaseosa para 
		las correcciones por ozono, y representa los parámetros de climatización. La expresión correspondiente se muestra en la ecuación \ref{eq:arvi}:

		\begin{equation}
			\label{eq:arvi}
			ARVI = \frac{(NIR - RB)}{(NIR + RB)}
		\end{equation}
	
	\item \textit{Soil-Adjusted Vegetation Index} (SAVI) \citep{Xue2017}: se estableció para mejorar la 
		sensibilidad del NDVI al fondo del suelo, donde $L$ es el índice de condicionamiento del suelo.
		El rango de $L$ es de 0 a 1. Su fórmula se puede ver expresada en la ecuación \ref{eq:savi}:

		\begin{equation}
			\label{eq:savi}
			SAVI = \frac{(P_n - P_r)(1 + L)}{P_n + P_r + L}
		\end{equation}
\end{itemize}

	Los índices de vegetación son herramientas fundamentales en el estudio de la agricultura, ya que permiten
 	evaluar de manera objetiva y eficiente el estado de la cobertura vegetal mediante el análisis de imágenes
  	satelitales. Estos índices, como el NDVI, SAVI, ARVI y EVI, proporcionan información clave sobre la salud de
	los cultivos, la disponibilidad de agua, el estrés hídrico y la productividad agrícola. Su uso facilita 
	la toma de decisiones en la gestión de los cultivos y son esenciales para mejorar la eficiencia y 
	sostenibilidad en la agricultura moderna.

\section{Teledetección}
	\label{sec:seguimientofloracion}
	En la tabla \ref{tab:plataformas} se presentan plataformas de teledetección disponibles y requisitos sugeridos 
	(*marginal, **óptimo) para supervisar la fenología de la floración forestal\citep{Dixon2023}.
	
	
	\begin{table}[h]
		\centering
		\caption{Plataformas de teledetección disponibles.}
		\begin{tabular}{l c c c c}    
			\toprule
			\textbf{Sensor} & \textbf{Resolución} & \textbf{Frecuencia} & \textbf{Extensión} & \textbf{Coste} \\
			\midrule
			PlanetScope & 3m*  & Diariamente**  & Regional** & Moderado* \\		
			Sentinel-2	 & 10 - 20 m  & 5 días  & Continental**   & Bajo**  \\
			Landsat	 & 30 m  & 16 días  & Continental** & Bajo** \\
			\bottomrule
		\end{tabular}
		\label{tab:plataformas}
	\end{table}

\section{Dependencias y Herramientas}

\subsection{Python}
Es uno de los lenguajes de programación más utilizados en el análisis de datos, incluidas las 
imágenes satelitales, gracias a su amplia variedad de bibliotecas de procesamiento de imágenes y análisis
geoespacial. Para el preprocesamiento de imágenes satelitales, algunas de las bibliotecas más destacadas 
incluyen:

\begin{enumerate}
	\item Rasterio: biblioteca que permite leer y escribir archivos raster, que son fundamentales en el 
	análisis de imágenes satelitales. Rasterio facilita la manipulación de datos en formatos comunes como
	GeoTIFF.
	\item NumPy: utilizada para el manejo de matrices y operaciones matemáticas sobre las imágenes, especialmente
	útil cuando se trabaja con grandes volúmenes de datos raster.  
	\item GeoPandas: es una extensión de Pandas diseñada para el manejo de datos geoespaciales. Agrega soporte 
	para trabajar con geometrías como puntos, líneas y polígonos dentro de un DataFrame, facilita la
	manipulación de datos espaciales en Python.
	\item Plotly: es una biblioteca de visualización interactiva en Python que permite la creación de gráficos
	 avanzados, incluye mapas geoespaciales. Es una herramienta clave para la exploración y análisis de 
	 datos provenientes de imágenes satelitales, ya que facilita la representación de patrones y
	tendencias en mapas interactivos.
	\end{enumerate}

Ventajas:

\begin{itemize}
	\item Python es flexible y permite integrar múltiples bibliotecas para un flujo de trabajo personalizado.
	\item Tiene una gran comunidad y documentación, lo que facilita la solución de problemas y la optimización del código.
	\item Es gratuito y de código abierto, lo que lo hace accesible para cualquier usuario. 
\end{itemize}

\subsection{Google Earth Engine}

Es una plataforma de procesamiento de imágenes geoespaciales en la nube que permite acceder a grandes 
volúmenes de datos geoespaciales, incluidos los conjuntos de datos satelitales históricos. GEE es utilizado
para realizar análisis geoespaciales y monitoreo ambiental a gran escala.

Características Principales

\begin{itemize}
	\item Acceso a grandes bases de datos de satélites: ofrece acceso a varios conjuntos de datos satelitales,
	 como Landsat, MODIS, Sentinel-1 y Sentinel-2, entre otros.
	\item Preprocesamiento automatizado: facilita el preprocesamiento de imágenes satelitales mediante funciones predefinidas
	 para corrección atmosférica, resampling, y mosaicos de imágenes, lo que permite un análisis rápido y 
	 preciso.
	\item Análisis a gran escala: permite realizar análisis a nivel global o en áreas de interés específicas,
	 utiliza algoritmos potentes de análisis de imágenes, como la clasificación de imágenes y la detección 
	 de cambios.
\end{itemize}




 
	\chapter{Diseño e implementación} % Main chapter title

\label{Chapter3} % Change X to a consecutive number; for referencing this chapter elsewhere, use \ref{ChapterX}

Este capítulo presenta la problemática abordada y expone el enfoque adoptado para su resolución, junto con las estrategias implementadas. Luego, se describe 
el análisis exploratorio inicial de los datos y el proceso de descarga de imágenes satelitales. 
Finalmente, se presentan los modelos de aprendizaje automático utilizados, junto con el proceso 
de desarrollo de los scripts para su implementación.
\definecolor{mygreen}{rgb}{0,0.6,0}
\definecolor{mygray}{rgb}{0.5,0.5,0.5}
\definecolor{mymauve}{rgb}{0.58,0,0.82}

%%%%%%%%%%%%%%%%%%%%%%%%%%%%%%%%%%%%%%%%%%%%%%%%%%%%%%%%%%%%%%%%%%%%%%%%%%%%%
% parámetros para configurar el formato del código en los entornos lstlisting
%%%%%%%%%%%%%%%%%%%%%%%%%%%%%%%%%%%%%%%%%%%%%%%%%%%%%%%%%%%%%%%%%%%%%%%%%%%%%
\lstset{ %
  backgroundcolor=\color{white},   % choose the background color; you must add \usepackage{color} or \usepackage{xcolor}
  basicstyle=\footnotesize,        % the size of the fonts that are used for the code
  breakatwhitespace=false,         % sets if automatic breaks should only happen at whitespace
  breaklines=true,                 % sets automatic line breaking
  captionpos=b,                    % sets the caption-position to bottom
  commentstyle=\color{mygreen},    % comment style
  deletekeywords={...},            % if you want to delete keywords from the given language
  %escapeinside={\%*}{*)},          % if you want to add LaTeX within your code
  %extendedchars=true,              % lets you use non-ASCII characters; for 8-bits encodings only, does not work with UTF-8
  %frame=single,	                % adds a frame around the code
  keepspaces=true,                 % keeps spaces in text, useful for keeping indentation of code (possibly needs columns=flexible)
  keywordstyle=\color{blue},       % keyword style
  language=[ANSI]C,                % the language of the code
  %otherkeywords={*,...},           % if you want to add more keywords to the set
  numbers=left,                    % where to put the line-numbers; possible values are (none, left, right)
  numbersep=5pt,                   % how far the line-numbers are from the code
  numberstyle=\tiny\color{mygray}, % the style that is used for the line-numbers
  rulecolor=\color{black},         % if not set, the frame-color may be changed on line-breaks within not-black text (e.g. comments (green here))
  showspaces=false,                % show spaces everywhere adding particular underscores; it overrides 'showstringspaces'
  showstringspaces=false,          % underline spaces within strings only
  showtabs=false,                  % show tabs within strings adding particular underscores
  stepnumber=1,                    % the step between two line-numbers. If it's 1, each line will be numbered
  stringstyle=\color{mymauve},     % string literal style
  tabsize=2,	                   % sets default tabsize to 2 spaces
  title=\lstname,                  % show the filename of files included with \lstinputlisting; also try caption instead of title
  morecomment=[s]{/*}{*/}
}


%----------------------------------------------------------------------------------------
%	SECTION 1
%----------------------------------------------------------------------------------------
\section{Enfoque para resolver el problema}

El Laboratorio de Biotecnología recopiló datos sobre las fechas de floración de durazneros
y las ubicaciones de las parcelas durante un período de cinco años. A partir de esta información, 
se evaluó la viabilidad de descargar imágenes satelitales correspondientes a las parcelas en las 
fechas de floración. El objetivo fue calcular índices de vegetación y aumentar la precisión de las predicciones.

El proceso del trabajo se compone en: 

\begin{enumerate}
  \item Adquisición y preprocesamiento de datos
      \begin{itemize}
        \item Integración y limpieza de datos crudos: fechas de floración y ubicaciones de parcelas.
      \end{itemize}
  \item Obtención y procesamiento de imágenes satelitales
      \begin{itemize}
        \item Desarrollo de un script en Python para interactuar con la API de Google Earth Engine.
        \item Descarga de imágenes satelitales correspondientes a las parcelas en sus respectivas fechas de floración.
        \item Lectura y procesamiento de los archivos .TIFF para extraer índices de vegetación.
      \end{itemize}
  \item Construcción del \textit{dataset}
      \begin{itemize}
        \item Confección de un nuevo conjunto de datos enriquecido con índices de vegetación.
        \item Estructuración y limpieza del dataset final.
      \end{itemize}
  \item  Modelado y entrenamiento
      \begin{itemize}
        \item Preprocesamiento de datos.
        \item Entrenamiento de modelos.
      \end{itemize}
  \item Evaluación y ajuste del modelo
      \begin{itemize}
        \item Cálculo de métricas de desempeño del modelo.
        \item Validación de resultados. Retroalimentación para mejora y reentrenamiento de modelo.
      \end{itemize}
	\end{enumerate}

\section{Análisis exploratorio inicial}

Inicialmente, se procedió a la extracción de los registros de floración correspondientes 
a todas las parcelas en un lapso de 5 años. Una representación parcial de la 
estructura de dichos datos se observa en la tabla \ref{tab:firstdataset}.

	\begin{table}[h]
		\centering
		\caption{Estructura de datos de floración.}
		\begin{tabular}{l c c c c c}    
			\toprule
			\textbf{Número} & \textbf{ID} & \textbf{Dias-floracion-17} & \textbf{...} & \textbf{Dias-floracion-23} \\
			\midrule
			0 & clv2 & 67 & ... & 65 \\		
			1 & clv3 & NaN & ... & 31 \\
			2 & clv5 & 73 & ... & 68 \\
      ... & ... & ... & ... & ... \\
      184 & htv10 & 30 & ... & 73 \\
			\bottomrule
		\end{tabular}
		\label{tab:firstdataset}
	\end{table}

  En este conjunto de datos, cada observación está asociada a un identificador único (ID), que permite distinguir las parcelas entre sí.  
  A continuación, se listan aspectos a tener en cuenta:
\begin{itemize}
  \item Existen 185 IDs que corresponden a una parcela única.
  \item Cada polígono contiene entre uno y tres árboles de una misma variedad. En total, se registraron 185 variedades distintas de durazneros.
  \item La fecha (número de la celda) se toma cuando aproximadamente el 50\,\% de las flores del árbol están abiertas. 
 \end{itemize} 

En la tabla \ref{tab:nullsdataset} se presenta la insuficiencia de datos que presenta el dataset.

\begin{table}[h]
  \centering
  \caption{Cantidad de datos faltantes en el dataset original.}
  \begin{tabular}{l c c}    
    \toprule
     \textbf{Columna} & \textbf{Cantidad de Nulos} \\
    \midrule
    Dias-floracion-17 & 35 \\		
    Dias-floracion-18	 & 19  \\
    Dias-floracion-19	& 38  \\
    Dias-floracion-20	 & 47 \\
    Dias-floracion-21	 & 1 \\
    Dias-floracion-22	 & 2 \\
    Dias-floracion-23	 & 8 \\
    \bottomrule
  \end{tabular}
  \label{tab:nullsdataset}
\end{table}

La presencia de valores faltantes puede afectar de forma negativa el rendimiento
de los modelos de aprendizaje automático, ya que introduce sesgos o impide el
funcionamiento correcto de ciertos algoritmos. La identificación y el tratamiento 
adecuado de estos valores, mediante su imputación, eliminación o el análisis de su patrón 
de aparición, contribuyen a mejorar la calidad del conjunto de datos. Esto, a su vez, incrementa 
la robustez del modelo y favorece predicciones más precisas y confiables.

\section{Proceso de descarga de imágenes}
El proceso completo de descarga de imágenes satelitales para cada parcela en las distintas fechas de 
floración puede desglosarse en dos etapas principales. La primera, abarca el análisis y las consideraciones 
fundamentales para la selección de la fuente de adquisición de imágenes. La segunda, se centra en la 
problemática detectada respecto a la proximidad entre parcelas y la estrategia adoptada para su resolución.

\subsection{Consideraciones}
En la sección \ref{sec:seguimientofloracion} se analizan las distintas plataformas de teledetección 
disponibles. A continuación, se detallan los criterios considerados para 
seleccionar la más adecuada:

\begin{itemize}
  \item Intervalo de revisión: se refiere a la periodicidad con la que el satélite captura
    nuevas imágenes. Dado que intervalos largos pueden generar variaciones significativas 
    en la vegetación, se priorizó un satélite con mayor frecuencia de captura.
  \item Disponibilidad temporal de los datos: se evaluó el rango de fechas en que los datos
    están disponibles. Como el análisis requiere información desde 2017, se seleccionó el 
    conjunto de datos que cubre dicho período.
  \item Cantidad de bandas espectrales: las bandas espectrales representan distintos rangos 
    del espectro electromagnético y permiten analizar diversas características de la superficie
    terrestre. Se optó por la plataforma que ofrece el mayor número de bandas, ya que son 
    fundamentales para el cálculo de los índices de vegetación.
  \item Accesibilidad: se consideró si el acceso a los datos es gratuito o pago. En este caso, 
    se priorizó el uso de fuentes de acceso libre.
  \item Resolución espacial: hace referencia al tamaño del píxel en cada banda espectral, lo 
    que influye en el nivel de detalle de la imagen. Aunque un menor tamaño de píxel proporciona
     mayor precisión, este criterio no fue determinante en la selección.
\end{itemize}

Tras analizar las distintas plataformas, se optó por Sentinel-2. Esta opción resultó más adecuada 
para el trabajo, ya que reúne las condiciones necesarias en cuanto a resolución temporal, acceso a 
datos históricos, riqueza espectral y disponibilidad gratuita.

\subsection{Elección de parcelas}

En la figura \ref{fig:parcelasSP} se observa el total de parcelas de árboles de durazno ubicadas 
en INTA San Pedro, identificadas a través de Google Earth Engine. 

Luego, se observó que las parcelas están demasiado próximas entre sí. Esto dificultaba la correcta
extracción de los índices de vegetación, ya que la resolución espacial del satélite superaba el tamaño
individual de cada parcela. Para abordar esta limitación, se llevó a cabo una subselección de 
parcelas que presentaran una distancia mínima adecuada entre sí. La figura \ref{fig:parcelasfinalSP} 
ilustra el conjunto final de parcelas seleccionadas tras este proceso.

\begin{figure}[h]
	\centering
	\includegraphics[width=0.8\textwidth]{./Figures/parcelas_san_pedro.PNG}
	\caption{Parcelas de durazneros de INTA San Pedro \protect\footnotemark.}
	\label{fig:parcelasSP}
\end{figure}

\begin{figure}[h]
	\centering
	\includegraphics[width=0.8\textwidth]{./Figures/recorte_lotes_completo_sentinel.PNG}
	\caption{Parcelas seleccionadas de durazneros de INTA San Pedro \protect\footnotemark.}
	\label{fig:parcelasfinalSP}
\end{figure}


\footnotetext{Imagenes de elaboración propia a partir del proveedor Sentinel-2.}
	\include{Chapters/Chapter4} 
	\include{Chapters/Chapter5} 
\end{verbatim}

Los apéndices también deben escribirse en archivos .tex separados, que se deben ubicar dentro de la carpeta \emph{Appendices}. Los apéndices vienen comentados por defecto con el caracter \code{\%} y para incluirlos simplemente se debe eliminar dicho caracter.

Finalmente, se encuentra el código para incluir la bibliografía en el documento final.  Este código tampoco debe modificarse. La metodología para trabajar las referencias bibliográficas se desarrolla en la sección \ref{sec:biblio}.
%----------------------------------------------------------------------------------------

\section{Bibliografía}
\label{sec:biblio}

Las opciones de formato de la bibliografía se controlan a través del paquete de latex \option{biblatex} que se incluye en la memoria en el archivo memoria.tex.  Estas opciones determinan cómo se generan las citas bibliográficas en el cuerpo del documento y cómo se genera la bibliografía al final de la memoria.

En el preámbulo se puede encontrar el código que incluye el paquete biblatex, que no requiere ninguna modificación del usuario de la plantilla, y que contiene las siguientes opciones:

\begin{lstlisting}
\usepackage[backend=bibtex,
	natbib=true, 
	style=numeric, 
	sorting=none]
{biblatex}
\end{lstlisting}

En el archivo \file{reference.bib} se encuentran las referencias bibliográficas que se pueden citar en el documento.  Para incorporar una nueva cita al documento lo primero es agregarla en este archivo con todos los campos necesario.  Todas las entradas bibliográficas comienzan con $@$ y una palabra que define el formato de la entrada.  Para cada formato existen campos obligatorios que deben completarse. No importa el orden en que las entradas estén definidas en el archivo .bib.  Tampoco es importante el orden en que estén definidos los campos de una entrada bibliográfica. A continuación se muestran algunos ejemplos:

\begin{lstlisting}
@ARTICLE{ARTICLE:1,
    AUTHOR="John Doe",
    TITLE="Title",
    JOURNAL="Journal",
    YEAR="2017",
}
\end{lstlisting}


\begin{lstlisting}
@BOOK{BOOK:1,
    AUTHOR="John Doe",
    TITLE="The Book without Title",
    PUBLISHER="Dummy Publisher",
    YEAR="2100",
}
\end{lstlisting}


\begin{lstlisting}
@INBOOK{BOOK:2,
    AUTHOR="John Doe",
    TITLE="The Book without Title",
    PUBLISHER="Dummy Publisher",
    YEAR="2100",
    PAGES="100-200",
}
\end{lstlisting}


\begin{lstlisting}
@MISC{WEBSITE:1,
    HOWPUBLISHED = "\url{http://example.com}",
    AUTHOR = "Intel",
    TITLE = "Example Website",
    MONTH = "12",
    YEAR = "1988",
    URLDATE = {2012-11-26}
}
\end{lstlisting}

Se debe notar que los nombres \emph{ARTICLE:1}, \emph{BOOK:1}, \emph{BOOK:2} y \emph{WEBSITE:1} son nombres de fantasía que le sirve al autor del documento para identificar la entrada. En este sentido, se podrían reemplazar por cualquier otro nombre.  Tampoco es necesario poner : seguido de un número, en los ejemplos sólo se incluye como un posible estilo para identificar las entradas.

La entradas se citan en el documento con el comando: 

\begin{verbatim}
\citep{nombre_de_la_entrada}
\end{verbatim}

Y cuando se usan, se muestran así: \citep{ARTICLE:1}, \citep{BOOK:1}, \citep{BOOK:2}, \citep{WEBSITE:1}.  Notar cómo se conforma la sección Bibliografía al final del documento.

Finalmente y como se mencionó en la subsección \ref{subsec:configurando}, para actualizar las referencias bibliográficas tanto en la sección bibliografía como las citas en el cuerpo del documento, se deben ejecutar las herramientas de compilación PDFLaTeX, BibTeX, PDFLaTeX, PDFLaTeX, en ese orden.  Este procedimiento debería resolver cualquier mensaje "Citation xxxxx on page x undefined".

	\chapter{Introducción específica} % Main chapter title

\label{Chapter2}

%----------------------------------------------------------------------------------------
%	SECTION 1
%----------------------------------------------------------------------------------------
En esta sección se aborda el estudio de redes neuronales profundas aplicadas al análisis y predicción en 
agricultura, con un enfoque en la floración del duraznero. Se presenta una comparación entre diferentes 
arquitecturas de \textit{deep learning}, se destacan sus ventajas y limitaciones en este contexto. Además, se 
exploran los índices de vegetación más relevantes, utilizados para monitorear el estado de la cobertura 
vegetal y mejorar la precisión de los modelos predictivos.

\section{Aprendizaje profundo}

El uso de arquitecturas de aprendizaje profundo ha demostrado ser efectivo permitiendo 
una gestión agrícola más precisa y eficiente. A continuación, se describen algunas de las principales
arquitecturas aplicadas en este ámbito:


\begin{enumerate}
	\item Redes Neuronales Convolucionales (CNN): son ampliamente utilizadas para analizar datos 
	visuales, como imágenes satelitales o fotografías de cultivos. En el contexto de la floración del duraznero,
	pueden identificar patrones y características en imágenes que indican el inicio del proceso de
	floración. Por ejemplo, un estudio implementó dos arquitecturas de CNN para detectar la severidad de 
	lesiones causadas por enfermedades en hojas de durazno, y obtuvo una precisión del 87\,\% \citep{Rodriguez2023}. 
	\item Redes Generativas Antagónicas (GAN): esta arquitectura de aprendizaje profundo está compuesta 
	por dos modelos neuronales que compiten entre sí: un generador, que crea datos sintéticos, y un 
	discriminador, que evalúa la autenticidad de los datos. Esta dinámica permite generar datos 
	realistas similares a los de entrenamiento. En el ámbito agrícola, se han utilizado para generar imágenes 
	sintéticas de cultivos, lo cual facilitó la ampliación de conjuntos de datos para entrenar otros modelos de aprendizaje 
	automático \citep{Goodfellow2014}.
	\item Redes Neuronales de Base Radial (RBF): han sido aplicadas en estudios agrícolas para modelar relaciones
	 no lineales entre variables climáticas y fenológicas. Poseen gran capacidad para aproximar funciones complejas.
	 Por ejemplo, un estudio empleó RBF para estimar la evapotranspiración de referencia, y demostró su capacidad
	para modelar relaciones no lineales entre variables climáticas y fenómenos agrícolas \citep{Cerv2012}.
\end{enumerate}
	
En la tabla \ref{tab:redes_dl} se presenta un cuadro comparativo de las distintas arquitecturas menciondas.

\begin{table}[h]
    \centering
    \caption{Comparación de arquitecturas de Deep Learning}
    \renewcommand{\arraystretch}{1} % Aumenta el espacio entre filas
    \begin{tabular}{|>{\raggedright}m{2cm}|m{5cm}|m{5cm}|} % Usa celdas de tamaño fijo
        \hline
        \textbf{\large Modelo} & \textbf{\large Ventaja} & \textbf{\large Desventaja} \\
        \hline
        \large CNN & \large Excelente para extracción de características espaciales, especialmente en imágenes. 
		& \large Requiere gran cantidad de datos etiquetados y es computacionalmente costoso. \\
        \hline
        \large GAN & \large Genera datos sintéticos realistas, útil en aumento de datos y generación de imágenes. 
		& \large Difícil de entrenar y puede sufrir de colapso de modo, lo que afecta la calidad de la generación. \\
        \hline
        \large RBF & \large Buena aproximación de funciones no lineales, útil en problemas de clasificación y regresión.
		 & \large No escala bien con grandes volúmenes de datos y su entrenamiento puede ser ineficiente. \\
        \hline
        \large RNN & \large Captura dependencias temporales en secuencias, aplicable en procesamiento de lenguaje natural 
		y series temporales. & \large Problemas con gradientes desaparecidos en largas secuencias, requiere técnicas como 
		LSTM o GRU para mejorar rendimiento.\\
        \hline
    \end{tabular}
    \label{tab:redes_dl}
\end{table}


	Al seleccionar la arquitectura adecuada para un modelo de deep learning en tareas como la predicción de
	floración de plantas, es crucial considerar varios factores, entre ellos la naturaleza de los datos, el tipo
	de problema a resolver y los recursos disponibles. Las redes neuronales convolucionales (CNN) son 
  	ideales para tareas relacionadas con el procesamiento de imágenes o datos espaciales. En cambio, 
  	las redes neuronales recurrentes (RNN) y sus variantes como LSTM, son más adecuadas para modelar 
  	secuencias temporales, como los datos fenológicos de las plantas. Por otro lado, las redes generativas
   	antagónicas (GAN) pueden ser útiles cuando se requiere generar datos sintéticos, especialmente 
   	cuando hay escasez de datos etiquetados. La comparación entre modelos permite identificar cuál se adapta
	mejor a los objetivos del trabajo, con el fin de maximizar la precisión y eficiencia del sistema, y 
	minimizando los errores o el sobreajuste. La elección adecuada y la experimentación con diferentes
	enfoques garantizan un análisis robusto y confiable, fundamental para obtener resultados
	aplicables en la práctica agrícola.

\section{Procesamiento de Imágenes}
	El procesamiento de imágenes satelitales se ha convertido en una herramienta esencial para diversos
	campos, incluyendo la agricultura, la gestión de recursos naturales, la monitorización ambiental y 
 	la planificación urbana. Actualmente, existen varios métodos y técnicas avanzadas utilizados para
  	analizar y extraer información útil de estas imágenes. 
	

\subsection{Índices de vegetación}

La composición espectral del flujo radiante que emana de la superficie terrestre 
proporciona información sobre las propiedades físicas del suelo, el agua y la vegetación 
en entornos terrestres. Las técnicas, modelos e índices de teledetección están diseñados 
para convertir esta información espectral en una forma fácilmente interpretable \citep{Bannari1995}.
La información obtenida por teledetección sobre el crecimiento, el vigor y la dinámica de
la vegetación terrestre puede ser de gran provecho para el monitoreo del medio ambiente, 
la conservación de la biodiversidad, la agricultura, la silvicultura, las infraestructuras 
verdes urbanas y otros campos relacionados.

A continuación se presentan índices de vegetación relevantes para este trabajo:

\begin{itemize}
	\item \textit{Normalized Difference Vegetation Index} (NDVI) \citep{Xue2017}: se calcula como 
	   	relación normalizada entre las bandas roja e infrarroja cercana. Sus valores van en el rango
	   	entre 0 y 1. Tiene una reacción sensible a la vegetación verde, incluso en zonas con cobertura 
		vegetal escasa. El NDVI es sensible a los efectos del brillo y color del suelo,
		la atmósfera, las nubes y la sombra de las nubes, y la sombra del dosel foliar. Por ello su
		aplicación requiere la calibración de la teledetección. Se expresa como en la ecuación \ref{eq:ndvi}:
	
		\begin{equation}
			\label{eq:ndvi}
			NDVI = \left( \frac{p_{NIR} - p_{R}}{p_{NIR}} \right) + p_{R}
		\end{equation}

	\item \textit{Enhanced Vegetation Index} (EVI) \citep{Xue2017}: actúa como un parámetro que 
		corrige simultáneamente los efectos del suelo y de la atmósfera. Su formulación incluye los valores de NIR,
		 $R$ y $B$, previamente corregidos por la atmósfera. El término $L$ representa un parámetro de ajuste asociado
		  al suelo, cuyo valor se establece en 1. Además, se incorporan dos parámetros constantes con valores de 6 y 
		  7.5, respectivamente. La expresión matemática del índice se presenta en la ecuación \ref{eq:evi}:

		\begin{equation}
			\label{eq:evi}
			EVI = 2.5 \times \frac{(P_n - P_r)}{P_n + C_1 P_r - C_2 P_b + L}
		\end{equation}

	\item \textit{Atmospherically Resistant Vegetation Index} (ARVI) \citep{Xue2017}: se utiliza habitualmente para eliminar los efectos de los
		aerosoles atmosféricos. Se basa en el supuesto de que la atmósfera afecta significativamente a $R$ en comparación con el 
		NIR y puede reducir eficazmente la dependencia de este índice de vegetación de los efectos atmosféricos. En su fórmula 
		$RB$ es la diferencia entre $B$ y $R$, y está relacionada con la reflectancia influenciada por la dispersión molecular y la absorción gaseosa para 
		las correcciones por ozono, y representa los parámetros de climatización. La expresión correspondiente se muestra en la ecuación \ref{eq:arvi}:

		\begin{equation}
			\label{eq:arvi}
			ARVI = \frac{(NIR - RB)}{(NIR + RB)}
		\end{equation}
	
	\item \textit{Soil-Adjusted Vegetation Index} (SAVI) \citep{Xue2017}: se estableció para mejorar la 
		sensibilidad del NDVI al fondo del suelo, donde $L$ es el índice de condicionamiento del suelo.
		El rango de $L$ es de 0 a 1. Su fórmula se puede ver expresada en la ecuación \ref{eq:savi}:

		\begin{equation}
			\label{eq:savi}
			SAVI = \frac{(P_n - P_r)(1 + L)}{P_n + P_r + L}
		\end{equation}
\end{itemize}

	Los índices de vegetación son herramientas fundamentales en el estudio de la agricultura, ya que permiten
 	evaluar de manera objetiva y eficiente el estado de la cobertura vegetal mediante el análisis de imágenes
  	satelitales. Estos índices, como el NDVI, SAVI, ARVI y EVI, proporcionan información clave sobre la salud de
	los cultivos, la disponibilidad de agua, el estrés hídrico y la productividad agrícola. Su uso facilita 
	la toma de decisiones en la gestión de los cultivos y son esenciales para mejorar la eficiencia y 
	sostenibilidad en la agricultura moderna.

\section{Teledetección}
	\label{sec:seguimientofloracion}
	En la tabla \ref{tab:plataformas} se presentan plataformas de teledetección disponibles y requisitos sugeridos 
	(*marginal, **óptimo) para supervisar la fenología de la floración forestal\citep{Dixon2023}.
	
	
	\begin{table}[h]
		\centering
		\caption{Plataformas de teledetección disponibles.}
		\begin{tabular}{l c c c c}    
			\toprule
			\textbf{Sensor} & \textbf{Resolución} & \textbf{Frecuencia} & \textbf{Extensión} & \textbf{Coste} \\
			\midrule
			PlanetScope & 3m*  & Diariamente**  & Regional** & Moderado* \\		
			Sentinel-2	 & 10 - 20 m  & 5 días  & Continental**   & Bajo**  \\
			Landsat	 & 30 m  & 16 días  & Continental** & Bajo** \\
			\bottomrule
		\end{tabular}
		\label{tab:plataformas}
	\end{table}

\section{Dependencias y Herramientas}

\subsection{Python}
Es uno de los lenguajes de programación más utilizados en el análisis de datos, incluidas las 
imágenes satelitales, gracias a su amplia variedad de bibliotecas de procesamiento de imágenes y análisis
geoespacial. Para el preprocesamiento de imágenes satelitales, algunas de las bibliotecas más destacadas 
incluyen:

\begin{enumerate}
	\item Rasterio: biblioteca que permite leer y escribir archivos raster, que son fundamentales en el 
	análisis de imágenes satelitales. Rasterio facilita la manipulación de datos en formatos comunes como
	GeoTIFF.
	\item NumPy: utilizada para el manejo de matrices y operaciones matemáticas sobre las imágenes, especialmente
	útil cuando se trabaja con grandes volúmenes de datos raster.  
	\item GeoPandas: es una extensión de Pandas diseñada para el manejo de datos geoespaciales. Agrega soporte 
	para trabajar con geometrías como puntos, líneas y polígonos dentro de un DataFrame, facilita la
	manipulación de datos espaciales en Python.
	\item Plotly: es una biblioteca de visualización interactiva en Python que permite la creación de gráficos
	 avanzados, incluye mapas geoespaciales. Es una herramienta clave para la exploración y análisis de 
	 datos provenientes de imágenes satelitales, ya que facilita la representación de patrones y
	tendencias en mapas interactivos.
	\end{enumerate}

Ventajas:

\begin{itemize}
	\item Python es flexible y permite integrar múltiples bibliotecas para un flujo de trabajo personalizado.
	\item Tiene una gran comunidad y documentación, lo que facilita la solución de problemas y la optimización del código.
	\item Es gratuito y de código abierto, lo que lo hace accesible para cualquier usuario. 
\end{itemize}

\subsection{Google Earth Engine}

Es una plataforma de procesamiento de imágenes geoespaciales en la nube que permite acceder a grandes 
volúmenes de datos geoespaciales, incluidos los conjuntos de datos satelitales históricos. GEE es utilizado
para realizar análisis geoespaciales y monitoreo ambiental a gran escala.

Características Principales

\begin{itemize}
	\item Acceso a grandes bases de datos de satélites: ofrece acceso a varios conjuntos de datos satelitales,
	 como Landsat, MODIS, Sentinel-1 y Sentinel-2, entre otros.
	\item Preprocesamiento automatizado: facilita el preprocesamiento de imágenes satelitales mediante funciones predefinidas
	 para corrección atmosférica, resampling, y mosaicos de imágenes, lo que permite un análisis rápido y 
	 preciso.
	\item Análisis a gran escala: permite realizar análisis a nivel global o en áreas de interés específicas,
	 utiliza algoritmos potentes de análisis de imágenes, como la clasificación de imágenes y la detección 
	 de cambios.
\end{itemize}




 
	\chapter{Diseño e implementación} % Main chapter title

\label{Chapter3} % Change X to a consecutive number; for referencing this chapter elsewhere, use \ref{ChapterX}

Este capítulo presenta la problemática abordada y expone el enfoque adoptado para su resolución, junto con las estrategias implementadas. Luego, se describe 
el análisis exploratorio inicial de los datos y el proceso de descarga de imágenes satelitales. 
Finalmente, se presentan los modelos de aprendizaje automático utilizados, junto con el proceso 
de desarrollo de los scripts para su implementación.
\definecolor{mygreen}{rgb}{0,0.6,0}
\definecolor{mygray}{rgb}{0.5,0.5,0.5}
\definecolor{mymauve}{rgb}{0.58,0,0.82}

%%%%%%%%%%%%%%%%%%%%%%%%%%%%%%%%%%%%%%%%%%%%%%%%%%%%%%%%%%%%%%%%%%%%%%%%%%%%%
% parámetros para configurar el formato del código en los entornos lstlisting
%%%%%%%%%%%%%%%%%%%%%%%%%%%%%%%%%%%%%%%%%%%%%%%%%%%%%%%%%%%%%%%%%%%%%%%%%%%%%
\lstset{ %
  backgroundcolor=\color{white},   % choose the background color; you must add \usepackage{color} or \usepackage{xcolor}
  basicstyle=\footnotesize,        % the size of the fonts that are used for the code
  breakatwhitespace=false,         % sets if automatic breaks should only happen at whitespace
  breaklines=true,                 % sets automatic line breaking
  captionpos=b,                    % sets the caption-position to bottom
  commentstyle=\color{mygreen},    % comment style
  deletekeywords={...},            % if you want to delete keywords from the given language
  %escapeinside={\%*}{*)},          % if you want to add LaTeX within your code
  %extendedchars=true,              % lets you use non-ASCII characters; for 8-bits encodings only, does not work with UTF-8
  %frame=single,	                % adds a frame around the code
  keepspaces=true,                 % keeps spaces in text, useful for keeping indentation of code (possibly needs columns=flexible)
  keywordstyle=\color{blue},       % keyword style
  language=[ANSI]C,                % the language of the code
  %otherkeywords={*,...},           % if you want to add more keywords to the set
  numbers=left,                    % where to put the line-numbers; possible values are (none, left, right)
  numbersep=5pt,                   % how far the line-numbers are from the code
  numberstyle=\tiny\color{mygray}, % the style that is used for the line-numbers
  rulecolor=\color{black},         % if not set, the frame-color may be changed on line-breaks within not-black text (e.g. comments (green here))
  showspaces=false,                % show spaces everywhere adding particular underscores; it overrides 'showstringspaces'
  showstringspaces=false,          % underline spaces within strings only
  showtabs=false,                  % show tabs within strings adding particular underscores
  stepnumber=1,                    % the step between two line-numbers. If it's 1, each line will be numbered
  stringstyle=\color{mymauve},     % string literal style
  tabsize=2,	                   % sets default tabsize to 2 spaces
  title=\lstname,                  % show the filename of files included with \lstinputlisting; also try caption instead of title
  morecomment=[s]{/*}{*/}
}


%----------------------------------------------------------------------------------------
%	SECTION 1
%----------------------------------------------------------------------------------------
\section{Enfoque para resolver el problema}

El Laboratorio de Biotecnología recopiló datos sobre las fechas de floración de durazneros
y las ubicaciones de las parcelas durante un período de cinco años. A partir de esta información, 
se evaluó la viabilidad de descargar imágenes satelitales correspondientes a las parcelas en las 
fechas de floración. El objetivo fue calcular índices de vegetación y aumentar la precisión de las predicciones.

El proceso del trabajo se compone en: 

\begin{enumerate}
  \item Adquisición y preprocesamiento de datos
      \begin{itemize}
        \item Integración y limpieza de datos crudos: fechas de floración y ubicaciones de parcelas.
      \end{itemize}
  \item Obtención y procesamiento de imágenes satelitales
      \begin{itemize}
        \item Desarrollo de un script en Python para interactuar con la API de Google Earth Engine.
        \item Descarga de imágenes satelitales correspondientes a las parcelas en sus respectivas fechas de floración.
        \item Lectura y procesamiento de los archivos .TIFF para extraer índices de vegetación.
      \end{itemize}
  \item Construcción del \textit{dataset}
      \begin{itemize}
        \item Confección de un nuevo conjunto de datos enriquecido con índices de vegetación.
        \item Estructuración y limpieza del dataset final.
      \end{itemize}
  \item  Modelado y entrenamiento
      \begin{itemize}
        \item Preprocesamiento de datos.
        \item Entrenamiento de modelos.
      \end{itemize}
  \item Evaluación y ajuste del modelo
      \begin{itemize}
        \item Cálculo de métricas de desempeño del modelo.
        \item Validación de resultados. Retroalimentación para mejora y reentrenamiento de modelo.
      \end{itemize}
	\end{enumerate}

\section{Análisis exploratorio inicial}

Inicialmente, se procedió a la extracción de los registros de floración correspondientes 
a todas las parcelas en un lapso de 5 años. Una representación parcial de la 
estructura de dichos datos se observa en la tabla \ref{tab:firstdataset}.

	\begin{table}[h]
		\centering
		\caption{Estructura de datos de floración.}
		\begin{tabular}{l c c c c c}    
			\toprule
			\textbf{Número} & \textbf{ID} & \textbf{Dias-floracion-17} & \textbf{...} & \textbf{Dias-floracion-23} \\
			\midrule
			0 & clv2 & 67 & ... & 65 \\		
			1 & clv3 & NaN & ... & 31 \\
			2 & clv5 & 73 & ... & 68 \\
      ... & ... & ... & ... & ... \\
      184 & htv10 & 30 & ... & 73 \\
			\bottomrule
		\end{tabular}
		\label{tab:firstdataset}
	\end{table}

  En este conjunto de datos, cada observación está asociada a un identificador único (ID), que permite distinguir las parcelas entre sí.  
  A continuación, se listan aspectos a tener en cuenta:
\begin{itemize}
  \item Existen 185 IDs que corresponden a una parcela única.
  \item Cada polígono contiene entre uno y tres árboles de una misma variedad. En total, se registraron 185 variedades distintas de durazneros.
  \item La fecha (número de la celda) se toma cuando aproximadamente el 50\,\% de las flores del árbol están abiertas. 
 \end{itemize} 

En la tabla \ref{tab:nullsdataset} se presenta la insuficiencia de datos que presenta el dataset.

\begin{table}[h]
  \centering
  \caption{Cantidad de datos faltantes en el dataset original.}
  \begin{tabular}{l c c}    
    \toprule
     \textbf{Columna} & \textbf{Cantidad de Nulos} \\
    \midrule
    Dias-floracion-17 & 35 \\		
    Dias-floracion-18	 & 19  \\
    Dias-floracion-19	& 38  \\
    Dias-floracion-20	 & 47 \\
    Dias-floracion-21	 & 1 \\
    Dias-floracion-22	 & 2 \\
    Dias-floracion-23	 & 8 \\
    \bottomrule
  \end{tabular}
  \label{tab:nullsdataset}
\end{table}

La presencia de valores faltantes puede afectar de forma negativa el rendimiento
de los modelos de aprendizaje automático, ya que introduce sesgos o impide el
funcionamiento correcto de ciertos algoritmos. La identificación y el tratamiento 
adecuado de estos valores, mediante su imputación, eliminación o el análisis de su patrón 
de aparición, contribuyen a mejorar la calidad del conjunto de datos. Esto, a su vez, incrementa 
la robustez del modelo y favorece predicciones más precisas y confiables.

\section{Proceso de descarga de imágenes}
El proceso completo de descarga de imágenes satelitales para cada parcela en las distintas fechas de 
floración puede desglosarse en dos etapas principales. La primera, abarca el análisis y las consideraciones 
fundamentales para la selección de la fuente de adquisición de imágenes. La segunda, se centra en la 
problemática detectada respecto a la proximidad entre parcelas y la estrategia adoptada para su resolución.

\subsection{Consideraciones}
En la sección \ref{sec:seguimientofloracion} se analizan las distintas plataformas de teledetección 
disponibles. A continuación, se detallan los criterios considerados para 
seleccionar la más adecuada:

\begin{itemize}
  \item Intervalo de revisión: se refiere a la periodicidad con la que el satélite captura
    nuevas imágenes. Dado que intervalos largos pueden generar variaciones significativas 
    en la vegetación, se priorizó un satélite con mayor frecuencia de captura.
  \item Disponibilidad temporal de los datos: se evaluó el rango de fechas en que los datos
    están disponibles. Como el análisis requiere información desde 2017, se seleccionó el 
    conjunto de datos que cubre dicho período.
  \item Cantidad de bandas espectrales: las bandas espectrales representan distintos rangos 
    del espectro electromagnético y permiten analizar diversas características de la superficie
    terrestre. Se optó por la plataforma que ofrece el mayor número de bandas, ya que son 
    fundamentales para el cálculo de los índices de vegetación.
  \item Accesibilidad: se consideró si el acceso a los datos es gratuito o pago. En este caso, 
    se priorizó el uso de fuentes de acceso libre.
  \item Resolución espacial: hace referencia al tamaño del píxel en cada banda espectral, lo 
    que influye en el nivel de detalle de la imagen. Aunque un menor tamaño de píxel proporciona
     mayor precisión, este criterio no fue determinante en la selección.
\end{itemize}

Tras analizar las distintas plataformas, se optó por Sentinel-2. Esta opción resultó más adecuada 
para el trabajo, ya que reúne las condiciones necesarias en cuanto a resolución temporal, acceso a 
datos históricos, riqueza espectral y disponibilidad gratuita.

\subsection{Elección de parcelas}

En la figura \ref{fig:parcelasSP} se observa el total de parcelas de árboles de durazno ubicadas 
en INTA San Pedro, identificadas a través de Google Earth Engine. 

Luego, se observó que las parcelas están demasiado próximas entre sí. Esto dificultaba la correcta
extracción de los índices de vegetación, ya que la resolución espacial del satélite superaba el tamaño
individual de cada parcela. Para abordar esta limitación, se llevó a cabo una subselección de 
parcelas que presentaran una distancia mínima adecuada entre sí. La figura \ref{fig:parcelasfinalSP} 
ilustra el conjunto final de parcelas seleccionadas tras este proceso.

\begin{figure}[h]
	\centering
	\includegraphics[width=0.8\textwidth]{./Figures/parcelas_san_pedro.PNG}
	\caption{Parcelas de durazneros de INTA San Pedro \protect\footnotemark.}
	\label{fig:parcelasSP}
\end{figure}

\begin{figure}[h]
	\centering
	\includegraphics[width=0.8\textwidth]{./Figures/recorte_lotes_completo_sentinel.PNG}
	\caption{Parcelas seleccionadas de durazneros de INTA San Pedro \protect\footnotemark.}
	\label{fig:parcelasfinalSP}
\end{figure}


\footnotetext{Imagenes de elaboración propia a partir del proveedor Sentinel-2.}
	\include{Chapters/Chapter4} 
	\include{Chapters/Chapter5} 
\end{verbatim}

Los apéndices también deben escribirse en archivos .tex separados, que se deben ubicar dentro de la carpeta \emph{Appendices}. Los apéndices vienen comentados por defecto con el caracter \code{\%} y para incluirlos simplemente se debe eliminar dicho caracter.

Finalmente, se encuentra el código para incluir la bibliografía en el documento final.  Este código tampoco debe modificarse. La metodología para trabajar las referencias bibliográficas se desarrolla en la sección \ref{sec:biblio}.
%----------------------------------------------------------------------------------------

\section{Bibliografía}
\label{sec:biblio}

Las opciones de formato de la bibliografía se controlan a través del paquete de latex \option{biblatex} que se incluye en la memoria en el archivo memoria.tex.  Estas opciones determinan cómo se generan las citas bibliográficas en el cuerpo del documento y cómo se genera la bibliografía al final de la memoria.

En el preámbulo se puede encontrar el código que incluye el paquete biblatex, que no requiere ninguna modificación del usuario de la plantilla, y que contiene las siguientes opciones:

\begin{lstlisting}
\usepackage[backend=bibtex,
	natbib=true, 
	style=numeric, 
	sorting=none]
{biblatex}
\end{lstlisting}

En el archivo \file{reference.bib} se encuentran las referencias bibliográficas que se pueden citar en el documento.  Para incorporar una nueva cita al documento lo primero es agregarla en este archivo con todos los campos necesario.  Todas las entradas bibliográficas comienzan con $@$ y una palabra que define el formato de la entrada.  Para cada formato existen campos obligatorios que deben completarse. No importa el orden en que las entradas estén definidas en el archivo .bib.  Tampoco es importante el orden en que estén definidos los campos de una entrada bibliográfica. A continuación se muestran algunos ejemplos:

\begin{lstlisting}
@ARTICLE{ARTICLE:1,
    AUTHOR="John Doe",
    TITLE="Title",
    JOURNAL="Journal",
    YEAR="2017",
}
\end{lstlisting}


\begin{lstlisting}
@BOOK{BOOK:1,
    AUTHOR="John Doe",
    TITLE="The Book without Title",
    PUBLISHER="Dummy Publisher",
    YEAR="2100",
}
\end{lstlisting}


\begin{lstlisting}
@INBOOK{BOOK:2,
    AUTHOR="John Doe",
    TITLE="The Book without Title",
    PUBLISHER="Dummy Publisher",
    YEAR="2100",
    PAGES="100-200",
}
\end{lstlisting}


\begin{lstlisting}
@MISC{WEBSITE:1,
    HOWPUBLISHED = "\url{http://example.com}",
    AUTHOR = "Intel",
    TITLE = "Example Website",
    MONTH = "12",
    YEAR = "1988",
    URLDATE = {2012-11-26}
}
\end{lstlisting}

Se debe notar que los nombres \emph{ARTICLE:1}, \emph{BOOK:1}, \emph{BOOK:2} y \emph{WEBSITE:1} son nombres de fantasía que le sirve al autor del documento para identificar la entrada. En este sentido, se podrían reemplazar por cualquier otro nombre.  Tampoco es necesario poner : seguido de un número, en los ejemplos sólo se incluye como un posible estilo para identificar las entradas.

La entradas se citan en el documento con el comando: 

\begin{verbatim}
\citep{nombre_de_la_entrada}
\end{verbatim}

Y cuando se usan, se muestran así: \citep{ARTICLE:1}, \citep{BOOK:1}, \citep{BOOK:2}, \citep{WEBSITE:1}.  Notar cómo se conforma la sección Bibliografía al final del documento.

Finalmente y como se mencionó en la subsección \ref{subsec:configurando}, para actualizar las referencias bibliográficas tanto en la sección bibliografía como las citas en el cuerpo del documento, se deben ejecutar las herramientas de compilación PDFLaTeX, BibTeX, PDFLaTeX, PDFLaTeX, en ese orden.  Este procedimiento debería resolver cualquier mensaje "Citation xxxxx on page x undefined".

	\chapter{Introducción específica} % Main chapter title

\label{Chapter2}

%----------------------------------------------------------------------------------------
%	SECTION 1
%----------------------------------------------------------------------------------------
En esta sección se aborda el estudio de redes neuronales profundas aplicadas al análisis y predicción en 
agricultura, con un enfoque en la floración del duraznero. Se presenta una comparación entre diferentes 
arquitecturas de \textit{deep learning}, se destacan sus ventajas y limitaciones en este contexto. Además, se 
exploran los índices de vegetación más relevantes, utilizados para monitorear el estado de la cobertura 
vegetal y mejorar la precisión de los modelos predictivos.

\section{Aprendizaje profundo}

El uso de arquitecturas de aprendizaje profundo ha demostrado ser efectivo permitiendo 
una gestión agrícola más precisa y eficiente. A continuación, se describen algunas de las principales
arquitecturas aplicadas en este ámbito:


\begin{enumerate}
	\item Redes Neuronales Convolucionales (CNN): son ampliamente utilizadas para analizar datos 
	visuales, como imágenes satelitales o fotografías de cultivos. En el contexto de la floración del duraznero,
	pueden identificar patrones y características en imágenes que indican el inicio del proceso de
	floración. Por ejemplo, un estudio implementó dos arquitecturas de CNN para detectar la severidad de 
	lesiones causadas por enfermedades en hojas de durazno, y obtuvo una precisión del 87\,\% \citep{Rodriguez2023}. 
	\item Redes Generativas Antagónicas (GAN): esta arquitectura de aprendizaje profundo está compuesta 
	por dos modelos neuronales que compiten entre sí: un generador, que crea datos sintéticos, y un 
	discriminador, que evalúa la autenticidad de los datos. Esta dinámica permite generar datos 
	realistas similares a los de entrenamiento. En el ámbito agrícola, se han utilizado para generar imágenes 
	sintéticas de cultivos, lo cual facilitó la ampliación de conjuntos de datos para entrenar otros modelos de aprendizaje 
	automático \citep{Goodfellow2014}.
	\item Redes Neuronales de Base Radial (RBF): han sido aplicadas en estudios agrícolas para modelar relaciones
	 no lineales entre variables climáticas y fenológicas. Poseen gran capacidad para aproximar funciones complejas.
	 Por ejemplo, un estudio empleó RBF para estimar la evapotranspiración de referencia, y demostró su capacidad
	para modelar relaciones no lineales entre variables climáticas y fenómenos agrícolas \citep{Cerv2012}.
\end{enumerate}
	
En la tabla \ref{tab:redes_dl} se presenta un cuadro comparativo de las distintas arquitecturas menciondas.

\begin{table}[h]
    \centering
    \caption{Comparación de arquitecturas de Deep Learning}
    \renewcommand{\arraystretch}{1} % Aumenta el espacio entre filas
    \begin{tabular}{|>{\raggedright}m{2cm}|m{5cm}|m{5cm}|} % Usa celdas de tamaño fijo
        \hline
        \textbf{\large Modelo} & \textbf{\large Ventaja} & \textbf{\large Desventaja} \\
        \hline
        \large CNN & \large Excelente para extracción de características espaciales, especialmente en imágenes. 
		& \large Requiere gran cantidad de datos etiquetados y es computacionalmente costoso. \\
        \hline
        \large GAN & \large Genera datos sintéticos realistas, útil en aumento de datos y generación de imágenes. 
		& \large Difícil de entrenar y puede sufrir de colapso de modo, lo que afecta la calidad de la generación. \\
        \hline
        \large RBF & \large Buena aproximación de funciones no lineales, útil en problemas de clasificación y regresión.
		 & \large No escala bien con grandes volúmenes de datos y su entrenamiento puede ser ineficiente. \\
        \hline
        \large RNN & \large Captura dependencias temporales en secuencias, aplicable en procesamiento de lenguaje natural 
		y series temporales. & \large Problemas con gradientes desaparecidos en largas secuencias, requiere técnicas como 
		LSTM o GRU para mejorar rendimiento.\\
        \hline
    \end{tabular}
    \label{tab:redes_dl}
\end{table}


	Al seleccionar la arquitectura adecuada para un modelo de deep learning en tareas como la predicción de
	floración de plantas, es crucial considerar varios factores, entre ellos la naturaleza de los datos, el tipo
	de problema a resolver y los recursos disponibles. Las redes neuronales convolucionales (CNN) son 
  	ideales para tareas relacionadas con el procesamiento de imágenes o datos espaciales. En cambio, 
  	las redes neuronales recurrentes (RNN) y sus variantes como LSTM, son más adecuadas para modelar 
  	secuencias temporales, como los datos fenológicos de las plantas. Por otro lado, las redes generativas
   	antagónicas (GAN) pueden ser útiles cuando se requiere generar datos sintéticos, especialmente 
   	cuando hay escasez de datos etiquetados. La comparación entre modelos permite identificar cuál se adapta
	mejor a los objetivos del trabajo, con el fin de maximizar la precisión y eficiencia del sistema, y 
	minimizando los errores o el sobreajuste. La elección adecuada y la experimentación con diferentes
	enfoques garantizan un análisis robusto y confiable, fundamental para obtener resultados
	aplicables en la práctica agrícola.

\section{Procesamiento de Imágenes}
	El procesamiento de imágenes satelitales se ha convertido en una herramienta esencial para diversos
	campos, incluyendo la agricultura, la gestión de recursos naturales, la monitorización ambiental y 
 	la planificación urbana. Actualmente, existen varios métodos y técnicas avanzadas utilizados para
  	analizar y extraer información útil de estas imágenes. 
	

\subsection{Índices de vegetación}

La composición espectral del flujo radiante que emana de la superficie terrestre 
proporciona información sobre las propiedades físicas del suelo, el agua y la vegetación 
en entornos terrestres. Las técnicas, modelos e índices de teledetección están diseñados 
para convertir esta información espectral en una forma fácilmente interpretable \citep{Bannari1995}.
La información obtenida por teledetección sobre el crecimiento, el vigor y la dinámica de
la vegetación terrestre puede ser de gran provecho para el monitoreo del medio ambiente, 
la conservación de la biodiversidad, la agricultura, la silvicultura, las infraestructuras 
verdes urbanas y otros campos relacionados.

A continuación se presentan índices de vegetación relevantes para este trabajo:

\begin{itemize}
	\item \textit{Normalized Difference Vegetation Index} (NDVI) \citep{Xue2017}: se calcula como 
	   	relación normalizada entre las bandas roja e infrarroja cercana. Sus valores van en el rango
	   	entre 0 y 1. Tiene una reacción sensible a la vegetación verde, incluso en zonas con cobertura 
		vegetal escasa. El NDVI es sensible a los efectos del brillo y color del suelo,
		la atmósfera, las nubes y la sombra de las nubes, y la sombra del dosel foliar. Por ello su
		aplicación requiere la calibración de la teledetección. Se expresa como en la ecuación \ref{eq:ndvi}:
	
		\begin{equation}
			\label{eq:ndvi}
			NDVI = \left( \frac{p_{NIR} - p_{R}}{p_{NIR}} \right) + p_{R}
		\end{equation}

	\item \textit{Enhanced Vegetation Index} (EVI) \citep{Xue2017}: actúa como un parámetro que 
		corrige simultáneamente los efectos del suelo y de la atmósfera. Su formulación incluye los valores de NIR,
		 $R$ y $B$, previamente corregidos por la atmósfera. El término $L$ representa un parámetro de ajuste asociado
		  al suelo, cuyo valor se establece en 1. Además, se incorporan dos parámetros constantes con valores de 6 y 
		  7.5, respectivamente. La expresión matemática del índice se presenta en la ecuación \ref{eq:evi}:

		\begin{equation}
			\label{eq:evi}
			EVI = 2.5 \times \frac{(P_n - P_r)}{P_n + C_1 P_r - C_2 P_b + L}
		\end{equation}

	\item \textit{Atmospherically Resistant Vegetation Index} (ARVI) \citep{Xue2017}: se utiliza habitualmente para eliminar los efectos de los
		aerosoles atmosféricos. Se basa en el supuesto de que la atmósfera afecta significativamente a $R$ en comparación con el 
		NIR y puede reducir eficazmente la dependencia de este índice de vegetación de los efectos atmosféricos. En su fórmula 
		$RB$ es la diferencia entre $B$ y $R$, y está relacionada con la reflectancia influenciada por la dispersión molecular y la absorción gaseosa para 
		las correcciones por ozono, y representa los parámetros de climatización. La expresión correspondiente se muestra en la ecuación \ref{eq:arvi}:

		\begin{equation}
			\label{eq:arvi}
			ARVI = \frac{(NIR - RB)}{(NIR + RB)}
		\end{equation}
	
	\item \textit{Soil-Adjusted Vegetation Index} (SAVI) \citep{Xue2017}: se estableció para mejorar la 
		sensibilidad del NDVI al fondo del suelo, donde $L$ es el índice de condicionamiento del suelo.
		El rango de $L$ es de 0 a 1. Su fórmula se puede ver expresada en la ecuación \ref{eq:savi}:

		\begin{equation}
			\label{eq:savi}
			SAVI = \frac{(P_n - P_r)(1 + L)}{P_n + P_r + L}
		\end{equation}
\end{itemize}

	Los índices de vegetación son herramientas fundamentales en el estudio de la agricultura, ya que permiten
 	evaluar de manera objetiva y eficiente el estado de la cobertura vegetal mediante el análisis de imágenes
  	satelitales. Estos índices, como el NDVI, SAVI, ARVI y EVI, proporcionan información clave sobre la salud de
	los cultivos, la disponibilidad de agua, el estrés hídrico y la productividad agrícola. Su uso facilita 
	la toma de decisiones en la gestión de los cultivos y son esenciales para mejorar la eficiencia y 
	sostenibilidad en la agricultura moderna.

\section{Teledetección}
	\label{sec:seguimientofloracion}
	En la tabla \ref{tab:plataformas} se presentan plataformas de teledetección disponibles y requisitos sugeridos 
	(*marginal, **óptimo) para supervisar la fenología de la floración forestal\citep{Dixon2023}.
	
	
	\begin{table}[h]
		\centering
		\caption{Plataformas de teledetección disponibles.}
		\begin{tabular}{l c c c c}    
			\toprule
			\textbf{Sensor} & \textbf{Resolución} & \textbf{Frecuencia} & \textbf{Extensión} & \textbf{Coste} \\
			\midrule
			PlanetScope & 3m*  & Diariamente**  & Regional** & Moderado* \\		
			Sentinel-2	 & 10 - 20 m  & 5 días  & Continental**   & Bajo**  \\
			Landsat	 & 30 m  & 16 días  & Continental** & Bajo** \\
			\bottomrule
		\end{tabular}
		\label{tab:plataformas}
	\end{table}

\section{Dependencias y Herramientas}

\subsection{Python}
Es uno de los lenguajes de programación más utilizados en el análisis de datos, incluidas las 
imágenes satelitales, gracias a su amplia variedad de bibliotecas de procesamiento de imágenes y análisis
geoespacial. Para el preprocesamiento de imágenes satelitales, algunas de las bibliotecas más destacadas 
incluyen:

\begin{enumerate}
	\item Rasterio: biblioteca que permite leer y escribir archivos raster, que son fundamentales en el 
	análisis de imágenes satelitales. Rasterio facilita la manipulación de datos en formatos comunes como
	GeoTIFF.
	\item NumPy: utilizada para el manejo de matrices y operaciones matemáticas sobre las imágenes, especialmente
	útil cuando se trabaja con grandes volúmenes de datos raster.  
	\item GeoPandas: es una extensión de Pandas diseñada para el manejo de datos geoespaciales. Agrega soporte 
	para trabajar con geometrías como puntos, líneas y polígonos dentro de un DataFrame, facilita la
	manipulación de datos espaciales en Python.
	\item Plotly: es una biblioteca de visualización interactiva en Python que permite la creación de gráficos
	 avanzados, incluye mapas geoespaciales. Es una herramienta clave para la exploración y análisis de 
	 datos provenientes de imágenes satelitales, ya que facilita la representación de patrones y
	tendencias en mapas interactivos.
	\end{enumerate}

Ventajas:

\begin{itemize}
	\item Python es flexible y permite integrar múltiples bibliotecas para un flujo de trabajo personalizado.
	\item Tiene una gran comunidad y documentación, lo que facilita la solución de problemas y la optimización del código.
	\item Es gratuito y de código abierto, lo que lo hace accesible para cualquier usuario. 
\end{itemize}

\subsection{Google Earth Engine}

Es una plataforma de procesamiento de imágenes geoespaciales en la nube que permite acceder a grandes 
volúmenes de datos geoespaciales, incluidos los conjuntos de datos satelitales históricos. GEE es utilizado
para realizar análisis geoespaciales y monitoreo ambiental a gran escala.

Características Principales

\begin{itemize}
	\item Acceso a grandes bases de datos de satélites: ofrece acceso a varios conjuntos de datos satelitales,
	 como Landsat, MODIS, Sentinel-1 y Sentinel-2, entre otros.
	\item Preprocesamiento automatizado: facilita el preprocesamiento de imágenes satelitales mediante funciones predefinidas
	 para corrección atmosférica, resampling, y mosaicos de imágenes, lo que permite un análisis rápido y 
	 preciso.
	\item Análisis a gran escala: permite realizar análisis a nivel global o en áreas de interés específicas,
	 utiliza algoritmos potentes de análisis de imágenes, como la clasificación de imágenes y la detección 
	 de cambios.
\end{itemize}




 
	\chapter{Diseño e implementación} % Main chapter title

\label{Chapter3} % Change X to a consecutive number; for referencing this chapter elsewhere, use \ref{ChapterX}

Este capítulo presenta la problemática abordada y expone el enfoque adoptado para su resolución, junto con las estrategias implementadas. Luego, se describe 
el análisis exploratorio inicial de los datos y el proceso de descarga de imágenes satelitales. 
Finalmente, se presentan los modelos de aprendizaje automático utilizados, junto con el proceso 
de desarrollo de los scripts para su implementación.
\definecolor{mygreen}{rgb}{0,0.6,0}
\definecolor{mygray}{rgb}{0.5,0.5,0.5}
\definecolor{mymauve}{rgb}{0.58,0,0.82}

%%%%%%%%%%%%%%%%%%%%%%%%%%%%%%%%%%%%%%%%%%%%%%%%%%%%%%%%%%%%%%%%%%%%%%%%%%%%%
% parámetros para configurar el formato del código en los entornos lstlisting
%%%%%%%%%%%%%%%%%%%%%%%%%%%%%%%%%%%%%%%%%%%%%%%%%%%%%%%%%%%%%%%%%%%%%%%%%%%%%
\lstset{ %
  backgroundcolor=\color{white},   % choose the background color; you must add \usepackage{color} or \usepackage{xcolor}
  basicstyle=\footnotesize,        % the size of the fonts that are used for the code
  breakatwhitespace=false,         % sets if automatic breaks should only happen at whitespace
  breaklines=true,                 % sets automatic line breaking
  captionpos=b,                    % sets the caption-position to bottom
  commentstyle=\color{mygreen},    % comment style
  deletekeywords={...},            % if you want to delete keywords from the given language
  %escapeinside={\%*}{*)},          % if you want to add LaTeX within your code
  %extendedchars=true,              % lets you use non-ASCII characters; for 8-bits encodings only, does not work with UTF-8
  %frame=single,	                % adds a frame around the code
  keepspaces=true,                 % keeps spaces in text, useful for keeping indentation of code (possibly needs columns=flexible)
  keywordstyle=\color{blue},       % keyword style
  language=[ANSI]C,                % the language of the code
  %otherkeywords={*,...},           % if you want to add more keywords to the set
  numbers=left,                    % where to put the line-numbers; possible values are (none, left, right)
  numbersep=5pt,                   % how far the line-numbers are from the code
  numberstyle=\tiny\color{mygray}, % the style that is used for the line-numbers
  rulecolor=\color{black},         % if not set, the frame-color may be changed on line-breaks within not-black text (e.g. comments (green here))
  showspaces=false,                % show spaces everywhere adding particular underscores; it overrides 'showstringspaces'
  showstringspaces=false,          % underline spaces within strings only
  showtabs=false,                  % show tabs within strings adding particular underscores
  stepnumber=1,                    % the step between two line-numbers. If it's 1, each line will be numbered
  stringstyle=\color{mymauve},     % string literal style
  tabsize=2,	                   % sets default tabsize to 2 spaces
  title=\lstname,                  % show the filename of files included with \lstinputlisting; also try caption instead of title
  morecomment=[s]{/*}{*/}
}


%----------------------------------------------------------------------------------------
%	SECTION 1
%----------------------------------------------------------------------------------------
\section{Enfoque para resolver el problema}

El Laboratorio de Biotecnología recopiló datos sobre las fechas de floración de durazneros
y las ubicaciones de las parcelas durante un período de cinco años. A partir de esta información, 
se evaluó la viabilidad de descargar imágenes satelitales correspondientes a las parcelas en las 
fechas de floración. El objetivo fue calcular índices de vegetación y aumentar la precisión de las predicciones.

El proceso del trabajo se compone en: 

\begin{enumerate}
  \item Adquisición y preprocesamiento de datos
      \begin{itemize}
        \item Integración y limpieza de datos crudos: fechas de floración y ubicaciones de parcelas.
      \end{itemize}
  \item Obtención y procesamiento de imágenes satelitales
      \begin{itemize}
        \item Desarrollo de un script en Python para interactuar con la API de Google Earth Engine.
        \item Descarga de imágenes satelitales correspondientes a las parcelas en sus respectivas fechas de floración.
        \item Lectura y procesamiento de los archivos .TIFF para extraer índices de vegetación.
      \end{itemize}
  \item Construcción del \textit{dataset}
      \begin{itemize}
        \item Confección de un nuevo conjunto de datos enriquecido con índices de vegetación.
        \item Estructuración y limpieza del dataset final.
      \end{itemize}
  \item  Modelado y entrenamiento
      \begin{itemize}
        \item Preprocesamiento de datos.
        \item Entrenamiento de modelos.
      \end{itemize}
  \item Evaluación y ajuste del modelo
      \begin{itemize}
        \item Cálculo de métricas de desempeño del modelo.
        \item Validación de resultados. Retroalimentación para mejora y reentrenamiento de modelo.
      \end{itemize}
	\end{enumerate}

\section{Análisis exploratorio inicial}

Inicialmente, se procedió a la extracción de los registros de floración correspondientes 
a todas las parcelas en un lapso de 5 años. Una representación parcial de la 
estructura de dichos datos se observa en la tabla \ref{tab:firstdataset}.

	\begin{table}[h]
		\centering
		\caption{Estructura de datos de floración.}
		\begin{tabular}{l c c c c c}    
			\toprule
			\textbf{Número} & \textbf{ID} & \textbf{Dias-floracion-17} & \textbf{...} & \textbf{Dias-floracion-23} \\
			\midrule
			0 & clv2 & 67 & ... & 65 \\		
			1 & clv3 & NaN & ... & 31 \\
			2 & clv5 & 73 & ... & 68 \\
      ... & ... & ... & ... & ... \\
      184 & htv10 & 30 & ... & 73 \\
			\bottomrule
		\end{tabular}
		\label{tab:firstdataset}
	\end{table}

  En este conjunto de datos, cada observación está asociada a un identificador único (ID), que permite distinguir las parcelas entre sí.  
  A continuación, se listan aspectos a tener en cuenta:
\begin{itemize}
  \item Existen 185 IDs que corresponden a una parcela única.
  \item Cada polígono contiene entre uno y tres árboles de una misma variedad. En total, se registraron 185 variedades distintas de durazneros.
  \item La fecha (número de la celda) se toma cuando aproximadamente el 50\,\% de las flores del árbol están abiertas. 
 \end{itemize} 

En la tabla \ref{tab:nullsdataset} se presenta la insuficiencia de datos que presenta el dataset.

\begin{table}[h]
  \centering
  \caption{Cantidad de datos faltantes en el dataset original.}
  \begin{tabular}{l c c}    
    \toprule
     \textbf{Columna} & \textbf{Cantidad de Nulos} \\
    \midrule
    Dias-floracion-17 & 35 \\		
    Dias-floracion-18	 & 19  \\
    Dias-floracion-19	& 38  \\
    Dias-floracion-20	 & 47 \\
    Dias-floracion-21	 & 1 \\
    Dias-floracion-22	 & 2 \\
    Dias-floracion-23	 & 8 \\
    \bottomrule
  \end{tabular}
  \label{tab:nullsdataset}
\end{table}

La presencia de valores faltantes puede afectar de forma negativa el rendimiento
de los modelos de aprendizaje automático, ya que introduce sesgos o impide el
funcionamiento correcto de ciertos algoritmos. La identificación y el tratamiento 
adecuado de estos valores, mediante su imputación, eliminación o el análisis de su patrón 
de aparición, contribuyen a mejorar la calidad del conjunto de datos. Esto, a su vez, incrementa 
la robustez del modelo y favorece predicciones más precisas y confiables.

\section{Proceso de descarga de imágenes}
El proceso completo de descarga de imágenes satelitales para cada parcela en las distintas fechas de 
floración puede desglosarse en dos etapas principales. La primera, abarca el análisis y las consideraciones 
fundamentales para la selección de la fuente de adquisición de imágenes. La segunda, se centra en la 
problemática detectada respecto a la proximidad entre parcelas y la estrategia adoptada para su resolución.

\subsection{Consideraciones}
En la sección \ref{sec:seguimientofloracion} se analizan las distintas plataformas de teledetección 
disponibles. A continuación, se detallan los criterios considerados para 
seleccionar la más adecuada:

\begin{itemize}
  \item Intervalo de revisión: se refiere a la periodicidad con la que el satélite captura
    nuevas imágenes. Dado que intervalos largos pueden generar variaciones significativas 
    en la vegetación, se priorizó un satélite con mayor frecuencia de captura.
  \item Disponibilidad temporal de los datos: se evaluó el rango de fechas en que los datos
    están disponibles. Como el análisis requiere información desde 2017, se seleccionó el 
    conjunto de datos que cubre dicho período.
  \item Cantidad de bandas espectrales: las bandas espectrales representan distintos rangos 
    del espectro electromagnético y permiten analizar diversas características de la superficie
    terrestre. Se optó por la plataforma que ofrece el mayor número de bandas, ya que son 
    fundamentales para el cálculo de los índices de vegetación.
  \item Accesibilidad: se consideró si el acceso a los datos es gratuito o pago. En este caso, 
    se priorizó el uso de fuentes de acceso libre.
  \item Resolución espacial: hace referencia al tamaño del píxel en cada banda espectral, lo 
    que influye en el nivel de detalle de la imagen. Aunque un menor tamaño de píxel proporciona
     mayor precisión, este criterio no fue determinante en la selección.
\end{itemize}

Tras analizar las distintas plataformas, se optó por Sentinel-2. Esta opción resultó más adecuada 
para el trabajo, ya que reúne las condiciones necesarias en cuanto a resolución temporal, acceso a 
datos históricos, riqueza espectral y disponibilidad gratuita.

\subsection{Elección de parcelas}

En la figura \ref{fig:parcelasSP} se observa el total de parcelas de árboles de durazno ubicadas 
en INTA San Pedro, identificadas a través de Google Earth Engine. 

Luego, se observó que las parcelas están demasiado próximas entre sí. Esto dificultaba la correcta
extracción de los índices de vegetación, ya que la resolución espacial del satélite superaba el tamaño
individual de cada parcela. Para abordar esta limitación, se llevó a cabo una subselección de 
parcelas que presentaran una distancia mínima adecuada entre sí. La figura \ref{fig:parcelasfinalSP} 
ilustra el conjunto final de parcelas seleccionadas tras este proceso.

\begin{figure}[h]
	\centering
	\includegraphics[width=0.8\textwidth]{./Figures/parcelas_san_pedro.PNG}
	\caption{Parcelas de durazneros de INTA San Pedro \protect\footnotemark.}
	\label{fig:parcelasSP}
\end{figure}

\begin{figure}[h]
	\centering
	\includegraphics[width=0.8\textwidth]{./Figures/recorte_lotes_completo_sentinel.PNG}
	\caption{Parcelas seleccionadas de durazneros de INTA San Pedro \protect\footnotemark.}
	\label{fig:parcelasfinalSP}
\end{figure}


\footnotetext{Imagenes de elaboración propia a partir del proveedor Sentinel-2.}
	\include{Chapters/Chapter4} 
	\include{Chapters/Chapter5} 
\end{verbatim}

Los apéndices también deben escribirse en archivos .tex separados, que se deben ubicar dentro de la carpeta \emph{Appendices}. Los apéndices vienen comentados por defecto con el caracter \code{\%} y para incluirlos simplemente se debe eliminar dicho caracter.

Finalmente, se encuentra el código para incluir la bibliografía en el documento final.  Este código tampoco debe modificarse. La metodología para trabajar las referencias bibliográficas se desarrolla en la sección \ref{sec:biblio}.
%----------------------------------------------------------------------------------------

\section{Bibliografía}
\label{sec:biblio}

Las opciones de formato de la bibliografía se controlan a través del paquete de latex \option{biblatex} que se incluye en la memoria en el archivo memoria.tex.  Estas opciones determinan cómo se generan las citas bibliográficas en el cuerpo del documento y cómo se genera la bibliografía al final de la memoria.

En el preámbulo se puede encontrar el código que incluye el paquete biblatex, que no requiere ninguna modificación del usuario de la plantilla, y que contiene las siguientes opciones:

\begin{lstlisting}
\usepackage[backend=bibtex,
	natbib=true, 
	style=numeric, 
	sorting=none]
{biblatex}
\end{lstlisting}

En el archivo \file{reference.bib} se encuentran las referencias bibliográficas que se pueden citar en el documento.  Para incorporar una nueva cita al documento lo primero es agregarla en este archivo con todos los campos necesario.  Todas las entradas bibliográficas comienzan con $@$ y una palabra que define el formato de la entrada.  Para cada formato existen campos obligatorios que deben completarse. No importa el orden en que las entradas estén definidas en el archivo .bib.  Tampoco es importante el orden en que estén definidos los campos de una entrada bibliográfica. A continuación se muestran algunos ejemplos:

\begin{lstlisting}
@ARTICLE{ARTICLE:1,
    AUTHOR="John Doe",
    TITLE="Title",
    JOURNAL="Journal",
    YEAR="2017",
}
\end{lstlisting}


\begin{lstlisting}
@BOOK{BOOK:1,
    AUTHOR="John Doe",
    TITLE="The Book without Title",
    PUBLISHER="Dummy Publisher",
    YEAR="2100",
}
\end{lstlisting}


\begin{lstlisting}
@INBOOK{BOOK:2,
    AUTHOR="John Doe",
    TITLE="The Book without Title",
    PUBLISHER="Dummy Publisher",
    YEAR="2100",
    PAGES="100-200",
}
\end{lstlisting}


\begin{lstlisting}
@MISC{WEBSITE:1,
    HOWPUBLISHED = "\url{http://example.com}",
    AUTHOR = "Intel",
    TITLE = "Example Website",
    MONTH = "12",
    YEAR = "1988",
    URLDATE = {2012-11-26}
}
\end{lstlisting}

Se debe notar que los nombres \emph{ARTICLE:1}, \emph{BOOK:1}, \emph{BOOK:2} y \emph{WEBSITE:1} son nombres de fantasía que le sirve al autor del documento para identificar la entrada. En este sentido, se podrían reemplazar por cualquier otro nombre.  Tampoco es necesario poner : seguido de un número, en los ejemplos sólo se incluye como un posible estilo para identificar las entradas.

La entradas se citan en el documento con el comando: 

\begin{verbatim}
\citep{nombre_de_la_entrada}
\end{verbatim}

Y cuando se usan, se muestran así: \citep{ARTICLE:1}, \citep{BOOK:1}, \citep{BOOK:2}, \citep{WEBSITE:1}.  Notar cómo se conforma la sección Bibliografía al final del documento.

Finalmente y como se mencionó en la subsección \ref{subsec:configurando}, para actualizar las referencias bibliográficas tanto en la sección bibliografía como las citas en el cuerpo del documento, se deben ejecutar las herramientas de compilación PDFLaTeX, BibTeX, PDFLaTeX, PDFLaTeX, en ese orden.  Este procedimiento debería resolver cualquier mensaje "Citation xxxxx on page x undefined".
