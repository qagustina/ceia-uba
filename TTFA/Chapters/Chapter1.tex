% Chapter 1

\chapter{Introducción general} % Main chapter title

\label{Chapter1} % For referencing the chapter elsewhere, use \ref{Chapter1} 
\label{IntroGeneral}

En este capítulo se introduce a la problemática y se interpreta la importancia que implica el proyecto. 
Luego, se realiza un análisis del estado de arte sobre la gestión eficiente de salud de cultivos
y se puntualizan los objetivos y el alcance del trabajo.

%----------------------------------------------------------------------------------------

% Define some commands to keep the formatting separated from the content 
\newcommand{\keyword}[1]{\textbf{#1}}
\newcommand{\tabhead}[1]{\textbf{#1}}
\newcommand{\code}[1]{\texttt{#1}}
\newcommand{\file}[1]{\texttt{\bfseries#1}}
\newcommand{\option}[1]{\texttt{\itshape#1}}
\newcommand{\grados}{$^{\circ}$}

%----------------------------------------------------------------------------------------

%\section{Introducción}

%----------------------------------------------------------------------------------------
\section{Contexto}

La gestión eficiente de los montes frutales enfrenta varios desafíos significativos, entre los que se 
incluye la necesidad de monitoreo constante para asegurar la salud y productividad de los cultivos.
Los métodos tradicionales de inspección y evaluación, que dependen en gran medida de visitas de campo 
y observaciones manuales, son laboriosos, costosos y a menudo limitados en alcance y frecuencia. 
Estos métodos también pueden ser subjetivos, dependiendo de la experiencia y percepción del observador,
lo que puede llevar a inconsistencias en la evaluación. Además, factores como el cambio climático y
la variabilidad en las condiciones meteorológicas añaden capas adicionales de complejidad a la gestión
agronómica, ya que afectan directamente la salud y el rendimiento de los cultivos.

En este contexto, las imágenes satelitales emergen como una herramienta poderosa para abordar estos
problemas. Estas imágenes proporcionan datos de observación de la Tierra que son consistentes, 
repetitivos y de amplia cobertura, lo cual es esencial para un monitoreo efectivo. A través de 
imágenes satelitales, se pueden obtener índices de vegetación, como el NDVI (Índice de Vegetación 
de Diferencia Normalizada), que permiten evaluar la vegetación de manera precisa y objetiva.
Estos índices son cruciales para detectar cambios en la vegetación mucho antes de que sean
visibles a simple vista.  

Para este proyecto, se pretende automatizar estas tareas con una herramienta de fácil acceso 
a los usuarios, donde además de poder obtener mejores resultados, se facilitará el labor y 
tiempo que requieren las mismas.

\subsection{sec}

Si sos nuevo en \LaTeX{}, hay un muy buen libro electrónico - disponible gratuitamente en Internet como un archivo PDF - llamado, \enquote{A (not so short) Introduction to \LaTeX{}}. El título del libro es generalmente acortado a simplemente \emph{lshort}. Puede descargar la versión más reciente en inglés (ya que se actualiza de vez en cuando) desde aquí:
\url{http://www.ctan.org/tex-archive/info/lshort/english/lshort.pdf}

Se puede encontrar la versión en español en la lista en esta página: \url{http://www.ctan.org/tex-archive/info/lshort/}

\subsubsection{Una subsubsección}

Acá tiene un ejemplo de una ``subsubsección'' que es el cuarto nivel de ordenamiento del texto, después de capítulo, sección y subsección.  Como se puede ver, las subsubsecciones no van numeradas en el cuerpo del documento ni en el índice.  El formato está definido por la plantilla y no debe ser modificado.

\subsection{sec}

Si estás escribiendo un documento con mucho contenido matemático, entonces es posible que desees leer el documento de la AMS (American Mathematical Society) llamado, \enquote{A Short Math Guide for \LaTeX{}}. Se puede encontrar en línea en el siguiente link: \url{http://www.ams.org/tex/amslatex.html} en la sección \enquote{Additional Documentation} hacia la parte inferior de la página.


%----------------------------------------------------------------------------------------

\section{Estado del arte}

Si estás familiarizado con \LaTeX{}, entonces podés explorar la estructura de directorios de esta plantilla y proceder a personalizarla agregando tu información en el bloque \emph{INFORMACIÓN DE LA PORTADA} en el archivo \file{memoria.tex}.  

Se puede continuar luego modificando el resto de los archivos siguiendo los lineamientos que se describen en la sección \ref{sec:FillingFile} en la página \pageref{sec:FillingFile}.

Debés asegurarte de leer el capítulo \ref{Chapter2} acerca de las convenciones utilizadas para las Memoria de los Trabajos Finales de la \degreename.

Si sos nuevo en \LaTeX{}, se recomienda que continúes leyendo el documento ya que contiene información básica para aprovechar el potencial de esta herramienta.


%----------------------------------------------------------------------------------------

\section{Alcance y objetivos}

blablabla ...
\subsection{Propósito del proyecto}

El objetivo principal del proyecto consistió en desarrollar un algoritmo que permita descargar y 
analizar imágenes satelitales para vincularlas con características específicas de durazneros. 
Específicamente, determinar, a partir de los datos disponibles, el progreso de las etapas de 
floración y maduración de los frutos en el árbol. Este trabajo implicó la aplicación de distintas 
arquitecturas de modelos de aprendizaje automático y redes neuronales artificiales. Los indicadores
clave para medir el éxito del trabajo es el aumento de las métricas que indican el rendimiento 
de los modelos ejecutados. 

\subsection{Alcance del proyecto}

A continuación, se detallan las actividades incluidas en este trabajo:

\begin{itemize}
  \item La evaluación de las diferentes fuentes de datos para la realización de este trabajo.
  \begin{itemize}
    \item Evaluación de datos tabulares que corresponden mediciones a 
    campo de las etapas de floración y maduración de frutos, obtenidos durante 5 años.
    \item Evaluación de imágenes satelitales correspondientes a los lotes de durazneros.
  \end{itemize}
  \item La determinación del progreso de las etapas de floración y maduración de los frutos en el
  duraznero.
  \item El desarrollo de una herramienta de fácil acceso para el equipo del cliente.
  \item La elaboración de un informe que detalle el procedimiento realizado y resultados.
\end{itemize}

Los siguientes elementos quedan fuera del alcance:

\begin{itemize}
  \item El desarrollo de una interfaz para el sistema.
  \item El despligue del desarrollo en producción.
  \item La difusión de los datos, son confidenciales y propiedad del INTA.
\end{itemize}

%----------------------------------------------------------------------------------------

\section{Entorno de trabajo}

Ante de comenzar a editar la plantilla debemos tener un editor \LaTeX{} instalado en nuestra computadora.  En forma análoga a lo que sucede en lenguaje C, que se puede crear y editar código con casi cualquier editor, existen ciertos entornos de trabajo que nos pueden simplificar mucho la tarea.  En este sentido, se recomienda, sobre todo para los principiantes en \LaTeX{} la utilización de TexMaker, un programa gratuito y multi-plantaforma que está disponible tanto para windows como para sistemas GNU/linux.

La versión más reciente de TexMaker es la 4.5 y se puede descargar del siguiente link: \url{http://www.xm1math.net/texmaker/download.html}. Se puede consultar el manual de usuario en el siguiente link: \url{http://www.xm1math.net/texmaker/doc.html}.
 

\subsection{Paquetes adicionales}

Si bien durante el proceso de instalación de TexMaker, o cualquier otro editor que se haya elegido, se instalarán en el sistema los paquetes básicos necesarios para trabajar con \LaTeX{}, la plantilla de los trabajos de Especialización y Maestría requieren de paquete adicionales.

Se indican a continuación los comandos que se deben introducir en la consola de Ubuntu (ctrl + alt + t) para instalarlos:

\begin{lstlisting}[language=bash]
  $ sudo apt install texlive-lang-spanish texlive-science 
  $ sudo apt install texlive-bibtex-extra biber
  $ sudo apt install texlive texlive-fonts-recommended
  $ sudo apt install texlive-latex-extra
\end{lstlisting}


\subsection{Configurando TexMaker}
\label{subsec:configurando}



Una vez instalado el programa y los paquetes adicionales se debe abrir el archivo memoria.tex con el editor para ver una pantalla similar a la que se puede apreciar en la figura \ref{fig:texmaker}. 
Una vez instalado el programa y los paquetes adicionales se debe abrir el archivo memoria.tex con el editor para ver una pantalla similar a la que se puede apreciar en la figura \ref{fig:texmaker}. 
Una vez instalado el programa y los paquetes adicionales se debe abrir el archivo memoria.tex con el editor para ver una pantalla similar a la que se puede apreciar en la figura \ref{fig:texmaker}. 
Una vez instalado el programa y los paquetes adicionales se debe abrir el archivo memoria.tex con el editor para ver una pantalla similar a la que se puede apreciar en la figura \ref{fig:texmaker}. 

\vspace{1cm}

\begin{figure}[htbp]
	\centering
	\includegraphics[width=.5\textwidth]{./Figures/texmaker.png}
	\caption{Entorno de trabajo de texMaker.}
	\label{fig:texmaker}
\end{figure}

\vspace{1cm}

Notar que existe una vista llamada Estructura a la izquierda de la interfaz que nos permite abrir desde dentro del programa los archivos individuales de los capítulos.  A la derecha se encuentra una vista con el archivo propiamente dicho para su edición. Hacia la parte inferior se encuentra una vista del log con información de los resultados de la compilación.  En esta última vista pueden aparecen advertencias o \textit{warning}, que normalmente pueden ser ignorados, y los errores que se indican en color rojo y deben resolverse para que se genere el PDF de salida.

Recordar que el archivo que se debe compilar con PDFLaTeX es \file{memoria.tex}, si se tratara de compilar alguno de los capítulos saldría un error.  Para salvar la molestia de tener que cambiar de archivo para compilar cada vez que se realice una modificación en un capítulo, se puede definir el archivo \file{memoria.tex} como ``documento maestro'' yendo al menú opciones -> ``definir documento actual como documento maestro'', lo que permite compilar con PDFLaTeX memoria.tex directamente desde cualquier archivo que se esté modificando . Se muestra esta opción en la figura \ref{fig:docMaestro}.

\begin{figure}[ht]
	\centering
	\includegraphics[width=\textwidth]{./Figures/docMaestro.png}
	\caption{Definir memoria.tex como documento maestro.}
	\label{fig:docMaestro}
\end{figure}

En el menú herramientas se encuentran las opciones de compilación.  Para producir un archivo PDF a partir de un archivo .tex se debe ejecutar PDFLaTeX (el shortcut es F6). Para incorporar nueva bibliografía se debe utilizar la opción BibTeX del mismo menú herramientas (el shortcut es F11).

Notar que para actualizar las tablas de contenidos se debe ejecutar PDFLaTeX dos veces.  Esto se debe a que es necesario actualizar algunos archivos auxiliares antes de obtener el resultado final.  En forma similar, para actualizar las referencias bibliográficas se debe ejecutar primero PDFLaTeX, después BibTeX y finalmente PDFLaTeX dos veces por idénticos motivos.

\section{Personalizando la plantilla, el archivo \file{memoria.tex}}
\label{sec:FillingFile}

Para personalizar la plantilla se debe incorporar la información propia en los distintos archivos \file{.tex}. 

Primero abrir \file{memoria.tex} con TexMaker (o el editor de su preferencia). Se debe ubicar dentro del archivo el bloque de código titulado \emph{INFORMACIÓN DE LA PORTADA} donde se deben incorporar los primeros datos personales con los que se construirá automáticamente la portada.


%----------------------------------------------------------------------------------------

\section{El código del archivo \file{memoria.tex} explicado}

El archivo \file{memoria.tex} contiene la estructura del documento y es el archivo de mayor jerarquía de la memoria.  Podría ser equiparable a la función \emph{main()} de un programa en C, o mejor dicho al archivo fuente .c donde se encuentra definida la función main().

La estructura básica de cualquier documento de \LaTeX{} comienza con la definición de clase del documento, es seguida por un preámbulo donde se pueden agregar funcionalidades con el uso de \texttt{paquetes} (equiparables a bibliotecas de C), y finalmente, termina con el cuerpo del documento, donde irá el contenido de la memoria.

\lstset{%
  basicstyle=\small\ttfamily,
  language=[LaTeX]{TeX}
}

\begin{lstlisting}
\documentclass{article}  <- Definicion de clase
\usepackage{listings}	 <- Preambulo

\begin{document}	 <- Comienzo del contenido propio 
	Hello world!
\end{document}
\end{lstlisting}


El archivo \file{memoria.tex} se encuentra densamente comentado para explicar qué páginas, secciones y elementos de formato está creando el código \LaTeX{} en cada línea. El código está dividido en bloques con nombres en mayúsculas para que resulte evidente qué es lo que hace esa porción de código en particular. Inicialmente puede parecer que hay mucho código \LaTeX{}, pero es principalmente código para dar formato a la memoria por lo que no requiere intervención del usuario de la plantilla.  Sí se deben personalizar con su información los bloques indicados como:

\begin{itemize}
	\item Informacion de la memoria
	\item Resumen
	\item Agradecimientos
	\item Dedicatoria
\end{itemize}

El índice de contenidos, las listas de figura de tablas se generan en forma automática y no requieren intervención ni edición manual por parte del usuario de la plantilla. 

En la parte final del documento se encuentran los capítulos y los apéndices.  Por defecto se incluyen los 5 capítulos propuestos que se encuentran en la carpeta /Chapters. Cada capítulo se debe escribir en un archivo .tex separado y se debe poner en la carpeta \emph{Chapters} con el nombre \file{Chapter1}, \file{Chapter2}, etc\ldots El código para incluir capítulos desde archivos externos se muestra a continuación.

\begin{verbatim}
	% Chapter 1

\chapter{Introducción general} % Main chapter title

\label{Chapter1} % For referencing the chapter elsewhere, use \ref{Chapter1} 
\label{IntroGeneral}

En este capítulo se introduce la problemática y se interpreta la importancia que
implica el trabajo. Luego, se realiza un análisis del estado de arte sobre la ges-
tión eficiente de salud de cultivos y se puntualizan los objetivos y el alcance del
trabajo.

%----------------------------------------------------------------------------------------

% Define some commands to keep the formatting separated from the content 
\newcommand{\keyword}[1]{\textbf{#1}}
\newcommand{\tabhead}[1]{\textbf{#1}}
\newcommand{\code}[1]{\texttt{#1}}
\newcommand{\file}[1]{\texttt{\bfseries#1}}
\newcommand{\option}[1]{\texttt{\itshape#1}}
\newcommand{\grados}{$^{\circ}$}

%----------------------------------------------------------------------------------------

%\section{Introducción}

%----------------------------------------------------------------------------------------
\section{Contexto del trabajo}

En la EEA San Pedro de INTA se busca intensificar de manera sustentable la producción
de frutales, hortalizas y viveros. Actualmente, la inspección y evaluación
dependen de visitas de campo y observaciones manuales, procesos costosos, laboriosos
 y con alcance limitado. Además, el cambio climático y la variabilidad
meteorológica añaden capas adicionales de complejidad la gestión agronómica.

El Laboratorio de Biotecnología cuenta con datos históricos sobre las etapas de
floración y maduración de frutos en durazneros. En este contexto, las imágenes
satelitales se presentan como una herramienta poderosa, ya que ofrecen datos
consistentes y de amplia cobertura, permitiendo evaluar la vegetación de forma
precisa y objetiva mediante índices especializados.

La intención de este trabajo es automatizar la gestión de montes frutales mediante el 
uso de datos disponibles e índices obtenidos de imágenes satelitales. Para lograr este
objetivo, se hará uso de métodos de aprendizaje automático, aprendizaje profundo y 
técnicas de preprocesamiento de datos. Se propone una herramienta accesible para los 
usuarios, que optimiza los resultados y reduce el tiempo y esfuerzo requeridos.


\section{Estado del arte}

La recopilación de grandes cantidades de datos agrícolas ayuda a mejorar la toma
de decisiones para enriquecer la salud de los cultivos. Al mismo tiempo, el
aprendizaje profundo ha experimentado una gran popularidad en muchas áreas
de investigación y en diferentes modalidades de datos. Las imágenes por satélite
están disponibles en cantidades sin precedentes, lo que ha impulsado la investigación
en el ámbito de la teledetección. La naturaleza ávida de datos de los
modelos de aprendizaje profundo y este enorme volumen de datos resultan una
combinación perfecta.

\subsection{Seguimiento de floración mediante teledetección}
En la tabla \ref{tab:plataformas} se presentan plataformas de teledetección disponibles y requisitos sugeridos 
(*marginal, **óptimo) para supervisar la fenología de la floración forestal\citep{Dixon2023}.


\begin{table}[h]
	\centering
	\caption{Plataformas de teledetección disponibles.}
	\begin{tabular}{l c c c c}    
		\toprule
		\textbf{Sensor} & \textbf{Resolución} & \textbf{Frecuencia} & \textbf{Extensión} & \textbf{Coste} \\
		\midrule
		PlanetScope & 3m*  & Diariamente**  & Regional** & Moderado* \\		
		Sentinel-2	 & 10 - 20 m  & 5 días  & Continental**   & Bajo**  \\
		Landsat	 & 30 m  & 16 días  & Continental** & Bajo** \\
		\bottomrule
	\end{tabular}
	\label{tab:plataformas}
\end{table}



\subsection{Aprendizaje automático para pronóstico}

\begin{itemize}
  \item ARIMA \citep{Siami-Namini2018}: es un modelo de series temporales que combina autoregresión,
   diferenciación y media móvil para analizar y predecir datos no estacionarios. Ha demostrado 
   su superioridad en precisión y exactitud a la hora de predecir los próximos intervalos de 
   las series temporales.
   \item MLP \citep{Feng2020}: es un algoritmo de aprendizaje supervisado que aprende una función no lineal para
   problemas de clasficiación o regresión. Este modelo ha demostrado un buen rendimiento en 
   pronóstico.
  \item RNN \citep{Sebaa2020}: los datos de entrada de esta arquitectura son datos pasados
   y actuales, están diseñadas específicamente para tratar datos secuenciales, como secuencias de 
   palabras en problemas relacionados con la traducción automática, datos de audio en el 
   reconocimiento del habla o series temporales en problemas de pronóstico.
  \item LSTM \citep{Siami-Namini2018}: es un tipo de red neuronal Red Neuronal Recurrente (RNN) con la capacidad de
   de recordar los valores de etapas anteriores para predicción de la siguiente secuencia.
  \item SVR \citep{Makridakis2018}: es un proceso de regresión realizado por una máquina de vectores soporte que intenta
   identificar el hiperplano que maximiza el margen entre dos clases y minimiza el error total bajo 
   tolerancia. Se introduce una penalización de complejidad que equilibra el nivel de precisión en 
   pronósticos.
  \item Random Forest Regressor \citep{Nyemeche2023}: es un algoritmo que utiliza múltiples árboles de decisión para
  encontrar la salida del conjunto de datos de entrenamiento, ha demostrado dar buenos resultados en pronóstico.
  \end{itemize}

\subsection{Comparación de técnicas}

En la tabla \ref{tab:rmse} se muestra una comparación del RMSE (Error cuadrático medio) que se realizó
con datos de series temporales. Los valores de RMSE indican que los modelos basados en LSTM superan a los 
basados en ARIMA con un margen elevado \citep{Siami-Namini2018}.

\begin{table}[h]
	\centering
	\caption{Los RMSE de los modelos ARIMA y LSTM.}
	\begin{tabular}{l c c c}    
		\toprule
		\textbf{Metric} & \textbf{ARIMA} & \textbf{LSTM} & \textbf{\% Reducción en RMSE} \\
		\midrule
		RMSE Avg & 511.481  & 64.213  & -87.445 \\		
		\bottomrule
	\end{tabular}
	\label{tab:rmse}
\end{table}

\section{Alcance y Objetivos}
El objetivo principal del trabajo consistió en desarrollar un algoritmo que permita descargar y analizar 
imágenes satelitales para vincularlas con características específicas de durazneros. Específicamente, 
determinar, a partir de los datos disponibles, el progreso de las etapas de floración y maduración de 
los frutos en el árbol.

\subsection{Alcance del proyecto}
A continuación, se detallan las actividades incluidas en este trabajo:

\begin{itemize}
  \item Evaluación de las diferentes fuentes de datos disponibles:
  \begin{itemize}
    \item Datos tabulares, que corresponden a mediciones de campo
    de las etapas de floración y maduración de frutos, obtenidos durante 5
    años.
    \item Imágenes satelitales, correspondientes a los lotes de durazneros.
    \end{itemize}
  \item La determinación del progreso de las etapas de floración y maduración de
  los frutos en el duraznero.
  \item El desarrollo de una herramienta de fácil acceso para el equipo del cliente.
  \item La elaboración de un informe que detalle el procedimiento realizado y los resultados obtenidos.  
  \end{itemize}

  Los siguientes elementos quedan fuera del alcance:
  \begin{itemize}
    \item El desarrollo de una interfaz web para el sistema.
    \item El despligue del desarrollo en producción.
    \end{itemize}


	\chapter{Introducción específica} % Main chapter title

\label{Chapter2}

%----------------------------------------------------------------------------------------
%	SECTION 1
%----------------------------------------------------------------------------------------

En este capítulo se abordan los conceptos fundamentales que sustentan el desarrollo del trabajo. 
En primer lugar, se presenta una introducción al aprendizaje supervisado y se describen algoritmos 
como la regresión lineal y los modelos basados en árboles. Luego, se desarrolla el apartado 
de teledetección, donde se detallan las plataformas utilizadas, las colecciones de datos y las 
características de las bandas del satélite Sentinel-2. Finalmente, se incluye una sección dedicada 
al procesamiento de imágenes.

\section{Aprendizaje supervisado}
En el aprendizaje automático supervisado, los datos anotados conforman el conjunto de entrenamiento, mientras 
que los datos no anotados integran el conjunto de prueba. Cuando las anotaciones corresponden a valores 
discretos, se denominan etiquetas de clase, mientras que, si son valores numéricos continuos, se las conoce 
como valores objetivo continuos. Los problemas de aprendizaje supervisado se dividen en dos grandes categorías,
clasificación y regresión. En el caso de la regresión, el modelo se entrena a partir de un conjunto de datos 
etiquetados para aprender las relaciones existentes entre las variables y, posteriormente, predecir 
resultados de valor continuo sobre nuevos datos no observados \citep{mitchell2015}.

\subsection{Regresión lineal}

La regresión lineal \citep{shetty2022} es uno de los métodos más simples y ampliamente utilizados dentro de los 
modelos de regresión. Se trata de un enfoque lineal que busca modelar la relación existente entre
una variable dependiente (respuesta escalar) y una o más variables independientes (predictoras). 
Este método se emplea con frecuencia en tareas como la predicción de precios, la estimación de 
calificaciones o cualquier otro problema que implique valores continuos. En términos generales, la 
regresión lineal es un proceso estadístico utilizado con fines 
predictivos dentro del campo del aprendizaje automático. El modelo asume que la variable de salida 
$y$ y puede expresarse como una combinación lineal de las variables de entrada $x$. 


Por ejemplo, puede considerarse una ecuación lineal que combina un conjunto de variables de entrada $x$ y produce un valor de salida previsto 
$y$ correspondiente a esas entradas. En este contexto, tanto las variables de entrada como la salida son de naturaleza numérica.
Cada variable de entrada está asociada a un coeficiente, que actúa como un factor de escala y determina el grado 
de influencia de dicha variable sobre la respuesta. Además, se incorpora un término independiente, 
conocido como intersección o ordenada al origen, que representa el valor de salida cuando todas las variables de entrada son nulas.
El proceso de entrenamiento del modelo de regresión lineal consiste, en estimar los valores 
óptimos de estos coeficientes de manera que la ecuación resultante describa adecuadamente la relación 
entre los datos de entrada y los valores observados.

La figura \ref{fig:lr} muestra el ajuste de la línea de regresión lineal sobre los 
puntos de datos observados, lo que evidencia la correspondencia entre el modelo y el comportamiento de las variables.

\begin{figure}[h]
	\centering
	\includegraphics[width=0.6\textwidth]{./Figures/lr.png}
	\caption{Ilustración del ajuste de un modelo de regresión lineal sobre un conjunto de datos\protect\footnotemark.}
	\label{fig:lr}
\end{figure}
\footnotetext{Imagen adaptada de \citep{shetty2022}}

\subsection{Modelos basados en árboles}

Para las tareas de regresión se utilizan diversos algoritmos, entre ellos, los modelos basados en árboles que son algoritmos de aprendizaje supervisado que usan estructuras tipo árbol para dividir
datos y hacer predicciones. Los árboles simples siguen reglas de sí/no
hasta formar grupos homogéneos, ofrecen una interpretación sencilla y admiten distintos tipos de variables.
Se pueden categorizar en árboles de decisión individuales, métodos de ensamble que combinan múltiples árboles, 
y otras variantes basadas en \textit{boosting}. A continuación, se detallan los utilizados en este trabajo: 

\begin{itemize}
	\item \textit{Decision Tree Regressor} \citep{dtr}: es un modelo que predice un valor continuo divide recursivamente el conjunto
	de datos. En cada división, busca la característica que mejor reduce la varianza de la variable objetivo. 
	\item \textit{Random Forest Regressor} \citep{rfr}: introduce aleatoriedad al considerar solo un subconjunto de características 
	para cada división. Construye un gran número de árboles de decisión individuales, entrenando cada uno en una 
	submuestra del conjunto de datos seleccionada con reemplazo.
	\item \textit{LGBM Regressor} \citep{lgbm}: utiliza la técnica de crecimiento de árboles \textit{leaf-wise} (por hoja), enfocándose en hojas con la mayor pérdida a optimizar. Tiene alta 
	eficiencia y velocidad, especialmente en grandes conjuntos de datos.
	\item \textit{Gradient Boosting Regressor} \citep{gbr}: construye un modelo aditivo en una moda gradual \textit{stage-wise}, cada árbol intenta corregir 
	secuencialmente los errores residuales del conjunto de árboles predecesores para mejorar la predicción general.
	\item \textit{XGB Regressor} \citep{xgbr}: este modelo incluye métodos de regularización para controlar el sobreajuste, el uso de una aproximación de 
	segundo orden (Taylor) de la función de pérdida (Hessiano) para una optimización más precisa y el soporte para el procesamiento paralelo. 
	Es conocido por ofrecer resultados de vanguardia en tareas de regresión.
\end{itemize}

Las bibliotecas \texttt{XGBoost} \citep{xgboost} y \texttt{scikit-learn} \citep{scikitlearn} implementan los algoritmos mencionados.

\section{Teledetección}

La teledetección ha sido reconocida como una importante fuente de información durante las
últimas décadas en una amplia gama de aplicaciones de observación de la Tierra. La
periodicidad se considera el requisito más importante, ya que su cumplimiento permite 
abordar los problemas identificados en tiempo real. Durante la última década, estos
requisitos se han cumplido cada vez más gracias a una flota de satélites de reconocimiento
con capacidades avanzadas que permiten aplicaciones agrícolas rentables. Además, varias 
agencias espaciales y proveedores de productos satelitales han adoptado una política 
de acceso libre y sin restricciones a los datos; por ejemplo, la Agencia Espacial 
Europea (ESA) para el programa Copernicus, incluidos los satélites Sentinel-1 y Sentinel-2,
aplica una política de datos libre, completa y abierta. 

Una aplicación en la que la observación de la tierra puede ser una importante fuente de 
información es la agricultura a pequeña escala. La alta resolución espacial necesaria 
para detectar los campos agrícolas de los pequeños agricultores, la resolución radiométrica
para discriminar entre los tipos de plantas en este entorno heterogéneo y la 
resolución temporal necesaria para supervisar los acontecimientos y la evolución 
(por ejemplo, las prácticas agrícolas y el crecimiento de los cultivos) durante la
temporada de cultivo se han podido obtener últimamente gracias a los sensores satelitales.
La exploración de cultivos requiere mucho tiempo y mano de obra, y por lo tanto es 
costosa, la observación de la tierra presenta una alternativa viable \citep{Stratoulias2017}. En la figura 
\ref{fig:remotesensing} se presenta un esquema de proceso de teledetección.

\begin{figure}[h]
	\centering
	\includegraphics[width=0.6\textwidth]{./Figures/esquema_teledeteccion.png}
	\caption{Esquema de proceso de teledetección. (a) Fuente de iluminación. (b) 
	Sensor. (c) Antena en tierra. (d) Software. (e) Usuarios\protect\footnotemark.}
	\label{fig:remotesensing}
\end{figure}
\footnotetext{Imagen adaptada de \citep{pntign}}

\subsection{Plataformas de teledetección}
\label{subsec:plataformas}
En la tabla \ref{tab:plataformas} se presentan plataformas de teledetección disponibles y requisitos sugeridos 
(*marginal, **óptimo) para supervisar la fenología de la floración forestal \citep{Dixon2023}.
		
\begin{table}[h]
	\centering
	\caption{Plataformas de teledetección disponibles.}
		\begin{tabular}{l c c c c}    
			\toprule
			\textbf{Sensor} & \textbf{Resolución} & \textbf{Frecuencia} & \textbf{Extensión} & \textbf{Coste} \\
			\midrule
			PlanetScope & 3m*  & Diariamente**  & Regional** & Moderado* \\		
			Sentinel-2	 & 10 - 20 m  & 5 días  & Continental**   & Bajo**  \\
			Landsat	 & 30 m  & 16 días  & Continental** & Bajo** \\
			\bottomrule
		\end{tabular}
		\label{tab:plataformas}
\end{table}

\subsection{Colecciones de datos}

El programa Copernicus \citep{copernicus}, una iniciativa conjunta de la Comisión Europea y la Agencia Espacial Europea (ESA), se implementa a través 
de la constelación de satélites Sentinel, desarrollados por la ESA. Estos satélites ofrecen una amplia gama de datos, que incluye:
imágenes de radar (Sentinel-1A/1B), imágenes ópticas de alta resolución (Sentinel-2A/2B), datos de monitoreo ambiental y 
climático para océanos y tierras (Sentinel-3), y datos sobre la calidad del aire (Sentinel-5P).

\subsubsection{Características de bandas de Sentinel-2}
\label{subsec:bandas}
\begin{table}[!htpb]
  \centering
  \caption{Bandas espectrales de Sentinel-2 \citep{Sentinel2}.}
  \begin{tabular}{l c c c}    
    \toprule
     \textbf{Nombre} & \textbf{Longitud de onda ($\mu$m)} & \textbf{Resolución (m)}\\
    \midrule
	B1 - Aerosol & 0,43 - 0,45 & 60 \\
    B2 - Azul & 0,45 - 0,52 & 10  \\		
    B3 - Verde & 0,54 - 0,57  & 10  \\
    B4 - Rojo & 0,65 - 0,68  & 10  \\
    B5 - Borde rojo 1 & 0,69 - 0,71 & 20  \\
    B6 - Borde rojo 2 & 0,73 - 0,74 & 20  \\
	B7 - Borde rojo 3 & 0,77 - 0,79 & 20  \\
    B8 - NIR & 0,78 - 0,90  & 10  \\
	B8A - Borde rojo 4 & 0,85 - 0,87  & 20 \\
	B9 - Vapor de agua &  0,93 - 0,95 & 60 \\
	B10 - Cirro & 1,36 - 1,39 & 60 \\
    B11 - SWIR 1 & 1,56 - 1,65 & 20  \\
	B12 - SWIR 2 & 2,10 - 2,28  & 20  \\
    \bottomrule
  \end{tabular}
  \label{tab:tab-bandas}
\end{table}

\section{Procesamiento de Imágenes}
El procesamiento de imágenes satelitales se ha convertido en una herramienta esencial para diversos
campos, como la agricultura, la gestión de recursos naturales, la monitorización ambiental y 
la planificación urbana. Actualmente, existen varios métodos y técnicas avanzadas utilizados para
analizar y extraer información útil de estas imágenes.
	
\subsection{\textit{TIFF (Tagged Image File Format)}}
El formato TIFF es un estándar basado en etiquetas que se utiliza para almacenar e intercambiar 
imágenes ráster. La especificación GeoTIFF amplía este formato al incorporar un conjunto de 
etiquetas adicionales que permiten describir la información cartográfica asociada a las imágenes.
De esta manera, GeoTIFF facilita el uso de imágenes provenientes  sistemas de imágenes por 
satélite, fotografías aéreas escaneadas, mapas escaneados o como resultado de análisis 
geográficos. Su objetivo es establecer un vínculo directo entre la imagen ráster y un sistema de 
coordenadas \cite{Mahammad2003}.

\subsection{Índices de vegetación}

La composición espectral del flujo radiante que emana de la superficie terrestre 
proporciona información sobre las propiedades físicas del suelo, el agua y la vegetación 
en entornos terrestres. Las técnicas, modelos e índices de teledetección están diseñados 
para convertir esta información espectral en una forma fácilmente interpretable \citep{Bannari1995}. La construcción de un índice de 
vegetación implica la combinación de diferentes bandas sensibles para eliminar el impacto de los efectos
ambientales de fondo (por ejemplo, el suelo sin vegetación, las masas de agua).

A continuación, se presentan índices de vegetación relevantes para este trabajo:

\begin{itemize}
	\item \textit{Normalized Difference Vegetation Index} (NDVI) \citep{Xue2017}: se calcula como 
	   	relación normalizada entre las bandas roja e infrarroja cercana. Tiene una reacción sensible a la vegetación verde,
		 incluso en zonas con cobertura vegetal escasa, es sensible a los efectos del brillo y color del suelo,
		la atmósfera, las nubes y la sombra de las nubes, y la sombra del dosel foliar.
	\item \textit{Enhanced Vegetation Index} (EVI) \citep{Xue2017}: corrige los efectos del suelo y de la atmósfera. Su formulación 
		incluye los valores de NIR, R y B. El término L representa un parámetro de ajuste asociado
		  al suelo, cuyo valor se establece en 1. Además, se incorporan dos parámetros constantes con valores de 6 y 
		  7.5, respectivamente.
	\item \textit{Atmospherically Resistant Vegetation Index} (ARVI) \citep{Xue2017}: se utiliza para eliminar los efectos de los
		aerosoles atmosféricos. Se basa en el supuesto de que la atmósfera afecta significativamente a R en comparación con el 
		NIR y reduce la dependencia de este índice de vegetación de los efectos atmosféricos. 
	\item \textit{Renormalized Difference Vegetation Index} (RDVI) \citep{vescovo2012}: su objetivo es linealizar las relaciones entre 
		el índice, los parámetros biofísicos y reducir el efecto de saturación en coberturas altas. Utiliza las bandas infrarrojo cercano
		 y roja.
	\item \textit{Green Normalized Difference Vegetation Index} (GNDVI) \citep{gndvi}: es una variante del NDVI que sustituye 
		la banda roja por la banda verde, esta modificación permite una mayor sensibilidad al contenido de clorofila y a la 
		absorción del nitrógeno en el follaje.
	\item \textit{Structure Insensitive Pigment Index} (SIPI) \citep{vi}: se usa para monitorear la salud de las plantas en 
		regiones con alta variabilidad en la estructura del dosel o el índice de área foliar, 
		para la detección temprana de enfermedades de las plantas u otras causas de estrés.
	\item \textit{Normalized Difference Red Edge Index} (NDRE) \citep{ndre}: es un método para medir la cantidad de clorofila 
	en las plantas. Su aplicación resulta eficaz hacia la mitad o el final del ciclo de cultivo, cuando las plantas 
	alcanzan su madurez y se encuentran próximas a la cosecha.
	\item \textit{Leaf Chlorophyll Index} (LCI) \citep{nata2021}: es un índice de vegetación que involucra la banda lateral roja,
	 que utiliza las características de reflectancia espectral de la banda lateral roja
	 y la banda del infrarrojo cercano para mostrar las diferencias en el contenido de clorofila.  
\end{itemize}

 
	\chapter{Diseño e implementación} % Main chapter title

\label{Chapter3} % Change X to a consecutive number; for referencing this chapter elsewhere, use \ref{ChapterX}

En este capítulo se describe la problemática planteada, seguidamente se describe el enfoque 
utilizado para resolver el problema, detallando las estrategias adoptadas. Luego, se expone 
un análisis exploratorio inicial de los datos y el proceso de descarga de imágenes satelitales. 
Finalmente, se presentan los modelos de aprendizaje automático utilizados, junto con el proceso 
de desarrollo de los scripts para su implementación.
\definecolor{mygreen}{rgb}{0,0.6,0}
\definecolor{mygray}{rgb}{0.5,0.5,0.5}
\definecolor{mymauve}{rgb}{0.58,0,0.82}

%%%%%%%%%%%%%%%%%%%%%%%%%%%%%%%%%%%%%%%%%%%%%%%%%%%%%%%%%%%%%%%%%%%%%%%%%%%%%
% parámetros para configurar el formato del código en los entornos lstlisting
%%%%%%%%%%%%%%%%%%%%%%%%%%%%%%%%%%%%%%%%%%%%%%%%%%%%%%%%%%%%%%%%%%%%%%%%%%%%%
\lstset{ %
  backgroundcolor=\color{white},   % choose the background color; you must add \usepackage{color} or \usepackage{xcolor}
  basicstyle=\footnotesize,        % the size of the fonts that are used for the code
  breakatwhitespace=false,         % sets if automatic breaks should only happen at whitespace
  breaklines=true,                 % sets automatic line breaking
  captionpos=b,                    % sets the caption-position to bottom
  commentstyle=\color{mygreen},    % comment style
  deletekeywords={...},            % if you want to delete keywords from the given language
  %escapeinside={\%*}{*)},          % if you want to add LaTeX within your code
  %extendedchars=true,              % lets you use non-ASCII characters; for 8-bits encodings only, does not work with UTF-8
  %frame=single,	                % adds a frame around the code
  keepspaces=true,                 % keeps spaces in text, useful for keeping indentation of code (possibly needs columns=flexible)
  keywordstyle=\color{blue},       % keyword style
  language=[ANSI]C,                % the language of the code
  %otherkeywords={*,...},           % if you want to add more keywords to the set
  numbers=left,                    % where to put the line-numbers; possible values are (none, left, right)
  numbersep=5pt,                   % how far the line-numbers are from the code
  numberstyle=\tiny\color{mygray}, % the style that is used for the line-numbers
  rulecolor=\color{black},         % if not set, the frame-color may be changed on line-breaks within not-black text (e.g. comments (green here))
  showspaces=false,                % show spaces everywhere adding particular underscores; it overrides 'showstringspaces'
  showstringspaces=false,          % underline spaces within strings only
  showtabs=false,                  % show tabs within strings adding particular underscores
  stepnumber=1,                    % the step between two line-numbers. If it's 1, each line will be numbered
  stringstyle=\color{mymauve},     % string literal style
  tabsize=2,	                   % sets default tabsize to 2 spaces
  title=\lstname,                  % show the filename of files included with \lstinputlisting; also try caption instead of title
  morecomment=[s]{/*}{*/}
}


%----------------------------------------------------------------------------------------
%	SECTION 1
%----------------------------------------------------------------------------------------
\section{Enfoque para resolver el problema}

El Laboratorio de Biotecnología recopiló datos sobre las fechas de floración de durazneros
y las ubicaciones de las parcelas durante un período de cinco años. Con base en esta 
información, se evaluó la viabilidad de descargar imágenes satelitales correspondientes
a dichas parcelas en las fechas de floración, con el propósito de calcular índices de
vegetación y mejorar la precisión en las predicciones.

El proceso del trabajo se compone en: 

\begin{enumerate}
  \item Adquisición y preprocesamiento de datos
      \begin{itemize}
        \item Integración y limpieza de datos crudos: fechas de floración y ubicaciones de parcelas.
      \end{itemize}
  \item Obtención y procesamiento de imágenes satelitales
      \begin{itemize}
        \item Desarrollo de un script en Python para interactuar con la API de Google Earth Engine.
        \item Descarga de imágenes satelitales correspondientes a las parcelas en sus respectivas fechas de floración.
        \item Lectura y procesamiento de los archivos .TIFF para extraer índices de vegetación.
      \end{itemize}
  \item Construcción del dataset
      \begin{itemize}
        \item Confección de un nuevo conjunto de datos enriquecido con índices de vegetación.
        \item Estructuración y limpieza del dataset final.
      \end{itemize}
  \item  Modelado y entrenamiento
      \begin{itemize}
        \item Preprocesamiento de datos.
        \item Entrenamiento de modelos.
      \end{itemize}
  \item Evaluación y ajuste del modelo
      \begin{itemize}
        \item Cálculo de métricas de desempeño del modelo.
        \item Validación de resultados. Retroalimentación para mejora, y reentrenamiento de modelo.
      \end{itemize}
	\end{enumerate}

\section{Análisis exploratorio inicial}

Inicialmente, se procedió a la extracción de los registros de floración correspondientes 
a todas las parcelas en un lapso de 5 años. Una representación parcial de la 
estructura de dichos datos se observa en la tabla \ref{tab:firstdataset}.

	\begin{table}[h]
		\centering
		\caption{Estructura de datos de floración.}
		\begin{tabular}{l c c c c c}    
			\toprule
			\textbf{Número} & \textbf{ID} & \textbf{Dias-floracion-17} & \textbf{...} & \textbf{Dias-floracion-23} \\
			\midrule
			0 & clv2 & 67 & ... & 65 \\		
			1 & clv3 & NaN & ... & 31 \\
			2 & clv5 & 73 & ... & 68 \\
      ... & ... & ... & ... & ... \\
      184 & htv10 & 30 & ... & 73 \\
			\bottomrule
		\end{tabular}
		\label{tab:firstdataset}
	\end{table}

A continuación, se listan aspectos a tener en cuenta:
\begin{itemize}
  \item Existen 185 ids que corresponden a una parcela única.
  \item Cada polígono tiene de 1 a 3 árboles que corresponden a una misma variedad, es decir se tienen 185 variedades distintas
   de durazneros.
  \item La fecha (número de la celda) se toma cuando aproximadamente el \%50 de las flores del árbol están abiertas. 
 \end{itemize} 

En la tabla \ref{tab:nullsdataset} se presenta la insuficiencia de datos que presenta el dataset.

\begin{table}[h]
  \centering
  \caption{Cantidad de datos faltantes en el dataset original.}
  \begin{tabular}{l c c}    
    \toprule
     \textbf{Columna} & \textbf{Cantidad de Nulos} \\
    \midrule
    Dias-floracion-17 & 35 \\		
    Dias-floracion-18	 & 19  \\
    Dias-floracion-19	& 38  \\
    Dias-floracion-20	 & 47 \\
    Dias-floracion-21	 & 1 \\
    Dias-floracion-22	 & 2 \\
    Dias-floracion-23	 & 8 \\
    \bottomrule
  \end{tabular}
  \label{tab:nullsdataset}
\end{table}

La presencia de valores faltantes puede afectar negativamente el rendimiento de los modelos
de aprendizaje automático, introduciendo sesgos o impidiendo que ciertos algoritmos funcionen
correctamente. Identificar y gestionar estos valores ya sea imputándolos, eliminándolos o 
analizando su patrón de aparición permite mejorar la calidad del dataset, aumentar la 
robustez del modelo y garantizar que las predicciones sean más precisas y confiables.

\section{Proceso de descarga de imágenes}
El proceso completo de descarga de imágenes satelitales para cada parcela en las distintas fechas de 
floración puede desglosarse en dos etapas principales. La primera abarca el análisis y las consideraciones 
fundamentales para la selección de la fuente de adquisición de imágenes. La segunda se centra en la 
problemática detectada respecto a la proximidad entre parcelas y la estrategia adoptada para su resolución.

\subsection{Consideraciones}
En la sección \ref{sec:seguimientofloracion} se analizan las distintas plataformas de teledetección 
disponibles. A continuación, se detallan los criterios considerados para 
seleccionar la más adecuada:

\begin{itemize}
  \item Intervalo de revisión: Se refiere a la periodicidad con la que el satélite captura
    nuevas imágenes. Dado que intervalos largos pueden generar variaciones significativas 
    en la vegetación, se priorizó un satélite con mayor frecuencia de captura.
  \item Disponibilidad temporal de los datos: Se evaluó el rango de fechas en que los datos
    están disponibles. Como el análisis requiere información desde 2017, se seleccionó el 
    conjunto de datos que cubre dicho período.
  \item Cantidad de bandas espectrales: Las bandas espectrales representan distintos rangos 
    del espectro electromagnético y permiten analizar diversas características de la superficie
    terrestre. Se optó por la plataforma que ofrece el mayor número de bandas, ya que son 
    fundamentales para el cálculo de los índices de vegetación.
  \item Accesibilidad: Se consideró si el acceso a los datos es gratuito o pago. En este caso, 
    se priorizó el uso de fuentes de acceso libre.
  \item Resolución espacial: Hace referencia al tamaño del píxel en cada banda espectral, lo 
    que influye en el nivel de detalle de la imagen. Aunque un menor tamaño de píxel proporciona
     mayor precisión, este criterio no fue determinante en la selección.
\end{itemize}

Tras el análisis comparativo de las distintas plataformas se optó por Sentinel-2,
era la opción más adecuada para este trabajo, al reunir las condiciones 
necesarias en términos de resolución temporal, acceso a datos históricos, riqueza
espectral y disponibilidad gratuita.

\subsection{Elección de parcelas}

En la figura \ref{fig:parcelasSP} se observa el total de parcelas de árboles de durazno ubicadas 
en INTA San Pedro identificadas a través de Google Earth Engine. 

Luego, se determinó que la proximidad entre las parcelas era demasiado reducida para una correcta
extracción de los índices de vegetación, debido a que la resolución espacial del satélite 
resultaba mayor al área individual de cada parcela. Para abordar esta limitación, se llevó a cabo una 
subselección de parcelas que presentaran una distancia mínima adecuada entre sí. La figura \ref{fig:parcelasfinalSP} 
ilustra el conjunto final de parcelas seleccionadas tras este proceso.

\begin{figure}[h]
	\centering
	\includegraphics[width=0.8\textwidth]{./Figures/parcelas_san_pedro.PNG}
	\caption{Parcelas de durazneros de INTA San Pedro.}
	\label{fig:parcelasSP}
\end{figure}

\begin{figure}[h]
	\centering
	\includegraphics[width=0.8\textwidth]{./Figures/recorte_lotes_completo_sentinel.PNG}
	\caption{Parcelas seleccionadas de durazneros de INTA San Pedro.}
	\label{fig:parcelasfinalSP}
\end{figure}

	% Chapter Template

\chapter{Ensayos y resultados} % Main chapter title
En este capítulo se presentan los ensayos realizados y los resultados obtenidos a lo largo del desarrollo 
del estudio. En primer lugar, se describe la caracterización de los montes frutales, considerando las 
particularidades de las áreas de estudio y las variables que permiten representar su estado vegetativo
y fenológico. Posteriormente, se aborda la predicción de la floración del duraznero, donde se detalla
el conjunto de datos utilizado, el diseño experimental adoptado y las métricas empleadas. Además, 
se incluye la descripción de los esquemas de validación, tanto a nivel global como por lotes. 
Finalmente, se presentan los resultados obtenidos y su interpretación, destacando los comportamientos
observados y la relación entre las variables analizadas. 

\label{Chapter4} % Change X to a consecutive number; for referencing this chapter elsewhere, use \ref{ChapterX}
%----------------------------------------------------------------------------------------
%	SECTION 1
%----------------------------------------------------------------------------------------

\section{Caracterización de montes frutales}

Con el propósito de monitorear el estado de los cultivos en los meses previos a la floración, 
se llevaron a cabo diversas visualizaciones de los índices de vegetación, con el fin de analizar
y comprender en mayor profundidad su comportamiento temporal. Estos índices permiten identificar
variaciones en la actividad fotosintética y en la cobertura vegetal, proporcionando una 
herramienta clave para anticipar el desarrollo fenológico de los frutales.

Tal como se muestra en la figura \ref{fig:evolucion-indices}, el análisis previo a la floración 
evidencia patrones de incremento progresivo en los valores de los índices, reflejando la dinámica 
fisiológica de los cultivos durante la fase de crecimiento vegetativo.

Complementariamente, en las figuras \ref{fig:densidad-uno} 
y \ref{fig:densidad-dos} se presentan las distribuciones de los valores pico
de los índices de vegetación correspondientes al período previo a la floración de los durazneros,
discriminados por año. Esta comparación permite observar diferencias interanuales en la magnitud 
y variabilidad de los índices, posiblemente asociadas a factores ambientales o de manejo del cultivo.
Dicho análisis contribuye a una caracterización más precisa del comportamiento de la vegetación
en las etapas críticas del ciclo fenológico.
\clearpage
\vspace*{\fill}
\begin{figure}[!htbp]
	\centering
	\includegraphics[width=1.0\textwidth]{./Figures/indices_densidad_1.png}
	\caption{Distribución de los valores pico de índices de vegetación previos a la floración.}
	\label{fig:densidad-uno}
\end{figure}
\vspace*{\fill}
\clearpage
\begin{figure}[!htbp]
	\centering
	\includegraphics[width=1.0\textwidth]{./Figures/indices_densidad_2.png}
	\caption{Distribución de los valores pico de índices de vegetación previos a la floración.}
	\label{fig:densidad-dos}
\end{figure}
\clearpage
Se observa que los años 2017, 2018 y 2019 presentan distribuciones más concentradas y con
menores valores medios para la mayoría de los índices, lo que sugiere condiciones 
previas a la floración caracterizadas por un menor vigor vegetativo o menor cobertura
verde. En contraste, los años más recientes desde 2020 a 2023, muestran desplazamientos 
hacia valores más altos en la densidad de probabilidad, esto indica un mayor desarrollo 
vegetativo en los meses previos a la floración.

Análisis por índice:
\begin{itemize}
    \item EVI: presenta valores bajos, aunque
     con una ligera tendencia creciente en 2022-2023, indica una mejora en la 
     actividad fotosintética previa a la floración. Este índice es sensible al vigor 
     de la vegetación y sugiere una respuesta positiva en los últimos años.
     \item ARVI: las distribuciones 
     de los años 2021-2023 se desplazan hacia valores más altos respecto a 2017-2019, 
     lo que podría asociarse a un aumento en la cobertura verde y reducción del estrés
      vegetal.
    \item RVI y RDVI: ambos índices presentan un comportamiento similar, con picos 
    de densidad de 0,3 a 0,5 y una dispersión moderada. En los años más 
    recientes se observa una mayor concentración de valores medios-altos, esto refleja 
    una mejora gradual del vigor vegetal.
    \item LCI y NDRE : muestran una clara diferenciación entre años. Los valores de 
        2021-2023 se agrupan en rangos más altos 0,5 a 0,7 lo que sugiere una mayor
         concentración de clorofila y hojas más activas antes de la floración.
    \item GNDVI: mantiene una distribución 
     levemente bimodal en la mayoría de los años, con un segundo pico en torno de 0,6 a 0,7
      en los últimos años, lo cual puede reflejar una heterogeneidad en el desarrollo del 
     follaje del lote.
     \item SIPI: presenta un desplazamiento 
     progresivo hacia valores más altos desde 2020, lo que indica una mejora en
      la relación entre pigmentos fotosintéticos y estructurales, asociada a un estado
       vegetativo más activo.
\end{itemize}

\section{Predicción de floración de duraznero}
Esta sección aborda la evaluación de modelos predictivos orientados a estimar la fecha de floración 
del duraznero. Se incluyen la descripción del conjunto de datos, las métricas empleadas para 
evaluar el rendimiento de los modelos y los procedimientos de validación tanto a nivel global como por lotes.

\subsection{Descripción de conjunto datos}

Para iniciar con la predicción de floración, con el objetivo de garantizar una evaluación realista del 
comportamiento predictivo del modelo en escenarios futuros, la partición de los datos se realizó respetando
el orden temporal de las observaciones. Los registros correspondientes al período 2017-2022 se utilizaron 
para el entrenamiento del modelo, mientras que los datos del año 2023 se reservaron como conjunto de validación, 
la tabla \ref{tab:splitdata} muestra en detalle el total de cada conjunto. 
Esta estrategia evita la fuga de información entre conjuntos conocida como \textit{data leakage} y permite simular condiciones 
de predicción operativa, en las que se dispone de información histórica para estimar la floración de campañas
posteriores. 

Luego, se realizó un proceso de codificación sobre la variable ID, para transformar valores categóricos en números 
enteros únicos, asignando un número diferente a cada identificador distinto. Esta variable se mantuvo en el 
entrenamiento de los modelos con el objetivo de evaluar el comportamiento de los algoritmos tanto por parcela 
individual como de forma global, teniendo en cuenta su identificación específica.


\begin{table}[h]
		\centering
		\caption{\textit{Split} de datos para regresión.}
		\begin{tabular}{l c c c}    
			\toprule
			\textbf{Periodo} & \textbf{Total} & \textbf{Conjunto} & \textbf{Descripción} \\
			\midrule
			2017 - 2022 & 4988 & Entrenamiento & Ajuste del modelo para predicción \\		
			2023 & 1504 & Validación & Evaluación del desempeño \\
			\bottomrule
		\end{tabular}
		\label{tab:splitdata}
\end{table}

\subsection{Diseño experimental y métricas seleccionadas}

En el proceso de selección de modelos, se optó por seguir un enfoque incremental, avanzando desde algoritmos 
simples hacia otros de mayor complejidad. En una primera etapa, se implementó un modelo basado en árboles de
decisión, con el objetivo de establecer una línea base y comprender el comportamiento
general de los datos. Posteriormente, se incorporaron modelos de tipo ensemble, que combinan múltiples árboles de decisión para 
mejorar la capacidad predictiva y reducir el sobreajuste. Dentro de este grupo, se evaluaron Random Forest,
Gradient Boosting, LGBM y XGBoost, los cuales difieren principalmente en la forma en 
que agregan y ponderan los árboles individuales, así como en su eficiencia computacional.

Haciendo un breve repaso de las parcelas, es importante recordar que, este estudio se
centra en la especie \textit{Prunus persica} (duraznero) donde cada parcela corresponde a una
única variedad, entendida como familias de árboles. 

Además de evaluar el desempeño global de los modelos, con el objetivo de obtener una evaluación más precisa, se implementó una estrategia de
validación por lotes. En este esquema:
\begin{itemize}
    \item El entrenamiento se realizó excluyendo al azar determinadas variedades, y posteriormente se evaluó utilizando 
            todas las variedades, incluidas aquellas que no habían sido vistas durante el entrenamiento.
    \item Se analizó la dispersión del error del modelo XGBoost para las distintas variedades, con el propósito de 
          identificar posibles diferencias en su desempeño según la parcela.
\end{itemize}

Esta metodología permite, por un lado, evaluar la capacidad del modelo para mantener un rendimiento consistente de 
generalización frente a datos nuevos, como nuevas variedades o registros provenientes de una campaña agrícola diferente y, por
otro, examinar en detalle el patrón de error dentro de cada lote, aportando una visión más completa sobre su estabilidad y robustez.

Con respecto a la evaluación del rendimiento de los modelos de predicción y con la intención de seleccionar la más adecuada, se utilizaron métricas de error ampliamente
utilizadas en problemas de regresión. En particular, se calcularon el error cuadrático medio (MSE), el coeficiente de determinación R² y la 
raíz del error cuadrático medio (RMSE) los cuales permiten cuantificar el ajuste del modelo y la magnitud de los errores de predicción. A 
continuación, se muestran las ecuaciones para obtener cada uno: 

\begin{equation}
\text{MSE} = \frac{1}{n} \sum_{i=1}^{n} (y_i - \hat{y}_i)^2
\label{eq:mse}
\end{equation}

\begin{equation}
\text{RMSE} = \sqrt{\frac{1}{n} \sum_{i=1}^{n} (y_i - \hat{y}_i)^2}
\label{eq:rmse}
\end{equation}

\begin{equation}
R^2 = 1 - \frac{\sum_{i=1}^{n}(y_i - \hat{y}_i)^2}{\sum_{i=1}^{n}(y_i - \bar{y})^2}
\label{eq:r2}
\end{equation}

En estas expresiones, $n$ representa el número de observaciones analizadas, que en este estudio corresponde a 4988 muestras para el conjunto 
de entrenamiento y 1504 muestras para el conjunto de validación. Los valores $y_i$ hacen referencia a las fechas reales de floración registradas 
en cada parcela, mientras que $\hat{y}_i$ corresponde a las fechas estimadas por los modelos de regresión. Finalmente, $\bar{y}$ representa la
media de las fechas reales de floración consideradas en el conjunto utilizado para la evaluación.


\subsection{Validación de pruebas}
\subsubsection{Esquema de validación global}
Luego del entrenamiento de los modelos, siguiendo el flujo descripto en la sección anterior, se obtuvieron las 
métricas de validación que se pueden ver en la tabla \ref{tab:metrics-general}.

\begin{table}[h]
		\centering
		\caption{Métricas de \textit{performance} de algoritmos basados en árboles.}
		\begin{tabular}{l c c c}    
			\toprule
			\textbf{Modelo} & \textbf{MSE} & \textbf{RMSE} & \textbf{R²} \\
			\midrule
			Desicion Tree& 13,73 & 3,70 & 0,32\\		
			Random Forest& 12,06 & 3,47 & 0,47\\
            LGBM & 12,78& 3,58 & 0,41\\
            Gradient Boosting & 13,38 & 3,66 & 0,35\\
            XGBoost & 12,18 & 3,49 & 0,46\\
			\bottomrule
		\end{tabular}
		\label{tab:metrics-general}
\end{table}
Los modelos basados en ensambles, Random Forest y XGBoost mostraron el mejor desempeño, 
alcanzando los menores errores de RMSE y los mayores valores de R², lo que sugiere una adecuada capacidad para
capturar los patrones asociados a la floración. 

Por otro lado, el resto de los modelos presentaron un desempeño ligeramente inferior. En particular, el modelo 
Decision Tree mostró una menor capacidad de generalización, evidenciando una mayor variabilidad en las predicciones.
Los modelos Gradient Boosting y LGBM alcanzaron resultados intermedios, con errores moderados, lo que indica que 
si bien lograron capturar parcialmente la relación entre las variables predictoras y la floración, su ajuste 
fue menos estable en comparación con los métodos de ensamble más robustos.

La figura \ref{fig:errores-modelos} ilustra la distribución de los errores para los cinco modelos de regresión bajo
estudio. 

\begin{figure}[!htbp]
	\centering
	\includegraphics[width=0.9\textwidth]{./Figures/errores_modelos.png}
	\caption{\textit{Boxplot} de errores por modelo.}
	\label{fig:errores-modelos}
\end{figure}

Se observa que la mediana (línea dentro de la caja) del error para los cinco modelos es bastante cercana a cero, lo
que sugiere que en promedio, ninguno de los modelos presenta un sesgo significativo (subestimación o 
sobreestimación sistemática). 

El modelo Gradient Boosting parece tener una mediana ligeramente más positiva, en comparación con 
DecisionTree. En términos de variabilidad y consistencia, los modelos RandomForest y LGBM
presentan una dispersión intercuartílica (tamaño de la caja) similar y relativamente estrecha, 
indicando una consistencia aceptable en sus errores. Por otro lado, todos los modelos muestran la
presencia \textit{outliers}, especialmente DecisionTree y XGBoost en el extremo 
inferior, lo que indica que existen algunas predicciones con errores grandes en la muestra de datos.
\clearpage
\subsubsection{Esquema de validación por lotes}
Para esta instancia de validación del modelo, se optó por excluir un conjunto de parcelas específicas
del proceso de entrenamiento, con el objetivo de evaluar la capacidad de generalización y la
robustez del modelo frente a datos no vistos. Se dejaron fuera las parcelas identificadas 
como FTemTJ, cgd214, cgd287, clv17, htb37, FlavGom y cgd246.

Los resultados obtenidos a partir de esta validación independiente se presentan en la tabla
\ref{tab:metrics-idexcluded}, donde se detallan las métricas de evaluación correspondientes.

\begin{table}[h]
		\centering
		\caption{Métricas de error con variedades excluidas.}
		\begin{tabular}{l c c}
			\toprule
			\textbf{Modelo} & \textbf{MSE} & \textbf{RMSE} \\
			\midrule
			Desicion Tree& 16,02 & 4,0 \\		
			Random Forest& 14,02 & 3,74 \\
            LGBM & 12,98 & 3,59\\
            Gradient Boosting & 13,7 & 3,71 \\
            XGBoost & 13,6 & 3,70 \\
			\bottomrule
		\end{tabular}
		\label{tab:metrics-idexcluded}
\end{table}

Al comparar los resultados globales con los obtenidos tras excluir las cinco parcelas,
se observa que los errores aumentan ligeramente en todos los modelos, lo que 
indica una menor capacidad de generalización al enfrentarse a datos no vistos durante el 
entrenamiento. El modelo LGBM mostró el error más bajo, sugiriendo una mayor robustez ante la
omisión de variedades de durazneros específicas en comparación con los otros modelos 
de ensemble. En contraste, Decision Tree presenta el mayor incremento en el error,
evidenciando su mayor sensibilidad a los cambios en los datos de entrenamiento.


En la continuación del análisis centrado en las variedades, se seleccionó el modelo 
XGBoost debido a su alto desempeño en las etapas de validación previas. Este modelo 
permitió realizar un estudio detallado sobre la magnitud del error asociada a 
cada variedad, como se observa en la figura \ref{fig:error-bestmodel}.
\clearpage
\begin{figure}[!htbp]
	\centering
	\includegraphics[width=0.9\textwidth]{./Figures/error_best_model.png}
	\caption{\textit{Parity plot} del modelo XGBoost.}
	\label{fig:error-bestmodel}
\end{figure}

Observando la distribución de los puntos:
\begin{itemize}
    \item Tendencia general: los puntos se agrupan en general cerca de la línea de predicción 
    perfecta, lo que sugiere que el modelo  tiene una buena capacidad predictiva en la mayoría 
    de las parcelas. La tendencia general de las predicciones sigue la tendencia de los datos 
    reales. 
    \item Dispersión: en la parte inferior izquierda (floración temprana, días julianos 30-40), el modelo 
    parece tener algunos de sus errores más grandes. Por ejemplo, un caso donde la fecha real 
    es cercana a 35 y la predicción es cercana a 65, lo que representa un gran error de 
    subestimación de la fecha de floración real. En la parte superior derecha (floración tardía,
    días julianos 80-90), la precisión parece ser relativamente alta, con muchos puntos muy 
    cercanos a la línea.
\end{itemize}

\section{Resultados}

El código implementado demostró un funcionamiento estable sin inconvenientes, permitió la descarga de
imágenes satelitales correspondientes a cinco años históricos. Asimismo, logró procesar archivos TIFF 
con una resolución espectral de 14 bandas. Por lo tanto, se cumplieron de manera apropiada los 
principales requerimientos funcionales establecidos al inicio del proyecto.

En cuanto al monitoreo de cultivos, se logró ver que entre ocho y cuatro semanas antes de la floración,
varios índices mantienen valores relativamente altos y estable. Algunos índices como ARVI, LCI, NDRE,
RVI, RDVI, GNDVI muestran fluctuaciones más marcadas, y tienden a moverse en conjunto durante toda
la etapa observada antes de la floración y alcanzan su pico. En la fecha de floración, los valores 
de la mayoría de los índices tienden a disminuir levemente o estabilizarse.

En lo que respecta a la predicción de la fecha de floración, en una primera etapa no fue posible
obtener resultados satisfactorios con los datos en su formato original, tal como se 
detalla en la tabla \ref{tab:datasetsatelite}. 
A partir de esta observación, se llevó a cabo una nueva instancia de entrenamiento con 
las variables derivadas, desarrolladas a partir del procesamiento y transformación de los datos 
originales, según se describe en la sección \ref{subsubsec:featureengineering}. Estas variables
generadas permitieron mejorar la representación de la información relevante y, en consecuencia, 
optimizar la capacidad predictiva de los modelos.

Se concluye que los modelos de ensemble son los que presentan mejores resultados, y las 
métricas RMSE y R² son buenas opciones para evaluar 
su rendimiento. Los modelos XGBoost y RandomForest presentan
resultados muy similares en cuanto a error y bondad de ajuste. A partir de un análisis más 
detallado de estos modelos, se observaron los siguientes puntos a 
destacar: 

\begin{itemize}
    \item Las pendientes de los índices SIPI y ARVI son las que aportan mayor información al proceso
        de predicción, reflejan una fuerte influencia sobre el desempeño general del modelo. Este 
        resultado resulta relevante útil para comprender qué variables explican en mayor medida la 
        variabilidad de la respuesta y favorecen la obtención de predicciones más precisas.
    \item Si bien los modelos evaluados presentan valores similares en cuanto al error y la bondad 
        de ajuste, difieren en la importancia que asignan a las variables predictoras. En particular,
        el modelo Random Forest ubica la variable ID dentro de su conjunto de las tres más importantes,
        lo cual resulta problemático, dado que dicha variable funciona únicamente como un identificador
        y no representa un valor biológico asociado al fenómeno estudiado. Este comportamiento
        sugiere que el modelo podría estar capturando patrones falsos relacionados con la 
        identificación de las parcelas, en lugar de basarse en las características reales del 
        cultivo. Considerando este aspecto, y en busca de un balance entre las métricas, 
        se concluye que el modelo XGBoost ofrece el 
        mejor desempeño general entre los modelos entrenados.
\end{itemize}
 
	% Chapter Template

\chapter{Conclusiones} % Main chapter title

\label{Chapter5} % Change X to a consecutive number; for referencing this chapter elsewhere, use \ref{ChapterX}
En este capítulo se presentan las conclusiones, los aportes más relevantes del trabajo y las posibles líneas de mejora a futuro.
%----------------------------------------------------------------------------------------

%----------------------------------------------------------------------------------------
%	SECTION 1
%----------------------------------------------------------------------------------------

\section{Resultados obtenidos}
A continuación, se resumen las principales actividades realizadas, junto con la forma en que se abordaron 
los problemas que surgieron durante el desarrollo y los logros obtenidos. 

La planificación se cumplió en cuanto a las actividades, aunque el tiempo de dedicación fue mayor al 
previsto inicialmente.

Respecto a la gestión de riesgos, se presentaron distintas situaciones:

\begin{itemize}
    \item Imágenes aéreas sin datos de bandas. Este es un problema que se presentó y no estaba
    contemplado. Se optó por excluir de la descarga de imágenes aquellas que no tengan
    las bandas requeridas para el cálculo de los índices.
    \item Suficiencia de datos. Este aspecto representó uno de los principales riesgos del proyecto.
     Si bien se logró realizar con éxito la descarga de imágenes satelitales correspondientes a 
     años históricos, se detectaron limitaciones específicas en el año 2017. En particular, 
     durante los meses previos a la floración del duraznero, la colección de datos seleccionada
     presentó amplios intervalos de fechas con imágenes incompletas o carentes de información 
     en varias bandas espectrales.
    \item Bajos recursos computacionales. Este aspecto se identificaba inicialmente como un posible
     riesgo, sin embargo, no representó una limitación en la práctica. Tanto la ejecución del 
     script para la obtención de imágenes satelitales como el entrenamiento de los modelos pudieron
      realizarse satisfactoriamente en una computadora personal.
\end{itemize}

Respecto a los requerimientos:

\begin{itemize}
    \item Se cumplieron de manera apropiada los funcionales, de manejo de datos, herramientas de
     codigo y documentación. 
    \item Caracterización de montes frutales. Esta etapa del trabajo pudo desarrollarse sin 
    inconvenientes. Si bien se había identificado como posible riesgo la incapacidad del 
    algoritmo para reconocer correctamente las etapas de floración, el problema no se presentó.  
    \item La predicción de la fecha de floración no formaba parte de los requerimientos iniciales 
    del proyecto, sin embargo, se realizó de manera satisfactoria como un aporte adicional. 
    Durante las primeras pruebas se evidenció que los datos satelitales disponibles no eran 
    suficientes para obtener resultados precisos, y esto se abordó mediante la ingeniería de
    características. En la etapa de evaluación de los modelos, la métrica de bondad de ajuste  
    arrojó un valor inferior al 50\%, lo que indica una capacidad moderada para explicar la 
    variabilidad de los datos. El equipo del cliente manifestó que, en este estudio, la métrica
    RMSE tiene mayor relevancia al momento de valorar el desempeño de los modelos, dado que 
    refleja de manera más directa el error promedio en días de predicción. En la sección \ref{sec:nextsteps}
    se presentan propuestas de mejora.
\end{itemize}

Entre los contenidos incorporados a lo largo de la especialización, algunos resultaron fundamentales
para la ejecución y el enfoque de este trabajo, especialmente los siguientes:

\begin{itemize}
    \item Análisis exploratorio de datos. Influyó directamente en la toma de decisiones respecto a las
     metodologías y herramientas aplicadas, además de aportar criterios sólidos para la interpretación
     y validación de los resultados.
    \item Aprendizaje de máquina. El aprendizaje sobre preparación y split  de datos, así como sobre métricas
     de evaluación, resultó de gran utilidad para el desarrollo del trabajo. De los algoritmos abordados
      durante la especialización, se implementaron modelos basados en árboles de decisión.

\end{itemize}
%----------------------------------------------------------------------------------------
%	SECTION 2
%----------------------------------------------------------------------------------------
\section{Próximos pasos}
\label{sec:nextsteps}
En una primera etapa de este estudio se descargaron imágenes satelitales correspondientes al
periodo de floración de los montes frutales, con el propósito de analizar el comportamiento
de los índices de vegetación durante ese momento fenológico. Sin embargo, no fueron 
utilizadas en el desarrollo de este trabajo, ya que el enfoque se centró en el análisis
previo a la floración. Estos datos representan un recurso valioso para futuros estudios,
ya que podrían emplearse para mejorar los modelos actuales.

En cuanto a líneas de trabajo futuras, sería de interés profundizar en la evaluación
temporal de los modelos mediante técnicas de validación más avanzadas. Entre ellas,
se propone la aplicación de \textit{rolling window}, que permitiría analizar el
rendimiento del modelo a lo largo del tiempo, considerando distintos periodos
de entrenamiento y prueba. Asi como también la implementación de un esquema de
validación cruzada para series temporales, contribuiría a obtener una estimación
más robusta del desempeño predictivo.

 
\end{verbatim}

Los apéndices también deben escribirse en archivos .tex separados, que se deben ubicar dentro de la carpeta \emph{Appendices}. Los apéndices vienen comentados por defecto con el caracter \code{\%} y para incluirlos simplemente se debe eliminar dicho caracter.

Finalmente, se encuentra el código para incluir la bibliografía en el documento final.  Este código tampoco debe modificarse. La metodología para trabajar las referencias bibliográficas se desarrolla en la sección \ref{sec:biblio}.
%----------------------------------------------------------------------------------------

\section{Bibliografía}
\label{sec:biblio}

Las opciones de formato de la bibliografía se controlan a través del paquete de latex \option{biblatex} que se incluye en la memoria en el archivo memoria.tex.  Estas opciones determinan cómo se generan las citas bibliográficas en el cuerpo del documento y cómo se genera la bibliografía al final de la memoria.

En el preámbulo se puede encontrar el código que incluye el paquete biblatex, que no requiere ninguna modificación del usuario de la plantilla, y que contiene las siguientes opciones:

\begin{lstlisting}
\usepackage[backend=bibtex,
	natbib=true, 
	style=numeric, 
	sorting=none]
{biblatex}
\end{lstlisting}

En el archivo \file{reference.bib} se encuentran las referencias bibliográficas que se pueden citar en el documento.  Para incorporar una nueva cita al documento lo primero es agregarla en este archivo con todos los campos necesario.  Todas las entradas bibliográficas comienzan con $@$ y una palabra que define el formato de la entrada.  Para cada formato existen campos obligatorios que deben completarse. No importa el orden en que las entradas estén definidas en el archivo .bib.  Tampoco es importante el orden en que estén definidos los campos de una entrada bibliográfica. A continuación se muestran algunos ejemplos:

\begin{lstlisting}
@ARTICLE{ARTICLE:1,
    AUTHOR="John Doe",
    TITLE="Title",
    JOURNAL="Journal",
    YEAR="2017",
}
\end{lstlisting}


\begin{lstlisting}
@BOOK{BOOK:1,
    AUTHOR="John Doe",
    TITLE="The Book without Title",
    PUBLISHER="Dummy Publisher",
    YEAR="2100",
}
\end{lstlisting}


\begin{lstlisting}
@INBOOK{BOOK:2,
    AUTHOR="John Doe",
    TITLE="The Book without Title",
    PUBLISHER="Dummy Publisher",
    YEAR="2100",
    PAGES="100-200",
}
\end{lstlisting}


\begin{lstlisting}
@MISC{WEBSITE:1,
    HOWPUBLISHED = "\url{http://example.com}",
    AUTHOR = "Intel",
    TITLE = "Example Website",
    MONTH = "12",
    YEAR = "1988",
    URLDATE = {2012-11-26}
}
\end{lstlisting}

Se debe notar que los nombres \emph{ARTICLE:1}, \emph{BOOK:1}, \emph{BOOK:2} y \emph{WEBSITE:1} son nombres de fantasía que le sirve al autor del documento para identificar la entrada. En este sentido, se podrían reemplazar por cualquier otro nombre.  Tampoco es necesario poner : seguido de un número, en los ejemplos sólo se incluye como un posible estilo para identificar las entradas.

La entradas se citan en el documento con el comando: 

\begin{verbatim}
\citep{nombre_de_la_entrada}
\end{verbatim}

Y cuando se usan, se muestran así: \citep{ARTICLE:1}, \citep{BOOK:1}, \citep{BOOK:2}, \citep{WEBSITE:1}.  Notar cómo se conforma la sección Bibliografía al final del documento.

Finalmente y como se mencionó en la subsección \ref{subsec:configurando}, para actualizar las referencias bibliográficas tanto en la sección bibliografía como las citas en el cuerpo del documento, se deben ejecutar las herramientas de compilación PDFLaTeX, BibTeX, PDFLaTeX, PDFLaTeX, en ese orden.  Este procedimiento debería resolver cualquier mensaje "Citation xxxxx on page x undefined".
