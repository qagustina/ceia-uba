\documentclass{article}
\usepackage{amsmath}
\usepackage{amssymb}
\usepackage{graphicx}

\title{\bf Probabilidad y Estadística para Inteligencia Artificial (CEIA-PEIA)}
\author{Agustina Quiros}     

\begin{document}
\maketitle

{\bfseries 1. Inferencia Bayesiana} 

Probabilidades de obtener cada color de pelota en las urnas:

\textit{B = pelota blanca, N = pelota negra} 

\begin{itemize}
    \item Urna 1: 2 blancas, 2 negras. Total = 4.
    \begin{align*}
        P(B) &= \left(\frac{2}{4}\right) \cdot 100 = 50\% \\
        P(N) &= \left(\frac{2}{4}\right) \cdot 100 = 50\%
    \end{align*}
    \item Urna 2: 4 blancas, 2 negras. Total = 6.
    \begin{align*}
        P(B) &= \left(\frac{4}{6}\right) \cdot 100 = 66.67\% \\
        P(N) &= \left(\frac{2}{6}\right) \cdot 100 = 33.33\%
    \end{align*}
    \item Urna 3: 1 blanca, 4 negras. Total = 5.
    \begin{align*}
        P(B) &= \left(\frac{1}{5}\right) \cdot 100 = 20\% \\
        P(N) &= \left(\frac{4}{5}\right) \cdot 100 = 80\%
    \end{align*}
\end{itemize}

{\bfseries Probabilidades de pelotas en cada urna}
\begin{itemize}
    \item Urna 1
    \begin{align*}
        \text{Pelotas blancas} &= 0.5 \\
        \text{Pelotas negras} &= 0.5
    \end{align*}
    \item Urna 2
    \begin{align*}
        \text{Pelotas blancas} &= 0.6 \\
        \text{Pelotas negras} &= 0.4
    \end{align*}
    \item Urna 3
    \begin{align*}
        \text{Pelotas blancas} &= 0.2 \\
        \text{Pelotas negras} &= 0.8
    \end{align*}
\end{itemize}

{\bfseries Probabilidad de seleccionar una urna al azar}
    \[ P(U1) = P(U2) = P(U3) = \frac{1}{3} \]

\begin{itemize}
    \item $B$ = Probabilidad de sacar una pelota blanca.
    \item $U$ = Probabilidad de sacar la urna II.
\end{itemize}

\[
P(B/U) = \frac{\frac{1}{3} \cdot 0.6}{\left(\frac{1}{3} \cdot 0.5\right) + \left(\frac{1}{3} \cdot 0.6\right) + \left(\frac{1}{3} \cdot 0.2\right)} = 0.46
\]
\\
\textit{\large La probabilidad de que la urna utilizada sea la urna II, dado que la pelota obtenida es de color blanco es: 0.46.}
\\

{\bfseries 2. Densidades Marginales} 

Función de densidad conjunta
\[ f_{X,Y}(x,y) = \frac{6(2 - \frac{2}{4}x - y)}{16} \ = \frac{3}{8} \left(2 - \frac{1}{2}x - y\right) \]

Los límites de integración para $y$ van de 0 a $ 2 - \frac{2}{4}x$ y para $x$ 0 a 4.

\textit{Gráfico del triángulo delimitado por el eje $x$, el eje $y$ y la recta $y=2 - \frac{2}{4}x$}

\begin{figure}[h] 
    \centering
    \includegraphics[width=0.5\textwidth]{../grafico_punto2.JPG}
    \caption{$y=2 - \frac{2}{4}x$}
\end{figure}

{\bfseries Densidad marginal de X}

\[ f_X(x) = \int_{0}^{2 - \frac{1}{2} x} \frac{3}{8} \left(2 - \frac{1}{2} x - y\right) \, dy \]

Resolución de la integral:
\[ f_X(x) = \frac{3}{8} \int_{0}^{2 - \frac{1}{2} x} \left(2 - \frac{1}{2} x - y\right) \, dy \]

\[ f_X(x) = \frac{3}{8} \left[ \int_{0}^{2 - \frac{1}{2} x} 2 \, dy - \int_{0}^{2 - \frac{1}{2} x} \frac{1}{2} x \, dy - \int_{0}^{2 - \frac{1}{2} x} y \, dy \right] \]

Integración de cada término
\[ \int_{0}^{2 - \frac{1}{2} x} 2 \, dy = 2y \Bigg|_{0}^{2 - \frac{1}{2} x} = 2 \left(2 - \frac{1}{2} x\right) \]

\[ \int_{0}^{2 - \frac{1}{2} x} \frac{1}{2} x \, dy = \frac{1}{2} x y \Bigg|_{0}^{2 - \frac{1}{2} x} = \frac{1}{2} x \left(2 - \frac{1}{2} x\right) \]

\[ \int_{0}^{2 - \frac{1}{2} x} y \, dy = \frac{y^2}{2} \Bigg|_{0}^{2 - \frac{1}{2} x} = \frac{1}{2} \left(2 - \frac{1}{2} x\right)^2 \]

Sustitución y simplificación
\[ f_X(x) = \frac{3}{8} \left[ 2 \left(2 - \frac{1}{2} x\right) - \frac{1}{2} x \left(2 - \frac{1}{2} x\right) - \frac{1}{2} \left(2 - \frac{1}{2} x\right)^2 \right] \]

\[ f_X(x) = \frac{3}{8} \left[ 4 - x - x + \frac{1}{4} x^2 - 2 + x - \frac{1}{8} x^2 \right] \]

\[ f_X(x) = \frac{3}{8} \left[ 2 - \frac{1}{8} x^2 \right] \]


{\bfseries 3. Máxima Verosimilitud} 
\[
\begin{array}{c|cccc}
\text{número} & 0 & 1 & 2 & 3 \\
\hline
\text{cantidad de ocurrencias} & 4 & 3 & 2 & 1 \\
\end{array}
\]

Función de verosimilitud para los datos observados:

\[ 
L(\theta) = \left( \frac{1\theta}{3} \right)^4 \cdot \left( \frac{2\theta}{3} \right)^3 \cdot \left( \frac{1-1\theta}{3} \right)^2 \cdot \left( \frac{2(1-1\theta)}{3} \right)^1 
\]

Logaritmo natural para simplificar pasando de productoria a sumatoria y de exponentes a coeficientes:

\[
l(\theta) = 4\ln\left(\frac{\theta}{3}\right) + 3\ln\left(\frac{2\theta}{3}\right) + 2\ln\left(1 - \frac{\theta}{3}\right) + \ln\left(\frac{2(1-\theta)}{3}\right)
\]

Propiedad del logaritmo de una fracción:

\[
4\ln\left(\frac{\theta}{3}\right) = 4\left(\ln(\theta) - \ln(3)\right) = 4\ln(\theta) - 4\ln(3)
\]

\[
3\ln\left(\frac{2\theta}{3}\right) = 3\left(\ln(2) + \ln(\theta) - \ln(3)\right) = 3\ln(2) + 3\ln(\theta) - 3\ln(3)
\]

\[
2\ln\left(1 - \frac{\theta}{3}\right)
\]

\[
\ln\left(\frac{2(1-\theta)}{3}\right) = \ln\left(2(1-\theta)\right) - \ln(3) = \ln(2) + \ln(1-\theta) - \ln(3)
\]

\[
l(\theta) = 4\ln(\theta) - 4\ln(3) + 3\ln(2) + 3\ln(\theta) - 3\ln(3) + 2\ln\left(1 - \frac{\theta}{3}\right) + \ln(2) + \ln(1-\theta) - \ln(3)
\]
\[
l(\theta) = 7\ln(\theta) + 4\ln(2) + 2\ln\left(1 - \frac{\theta}{3}\right) + \ln(1-\theta) - 8\ln(3)
\]

Derivación y resolución:
\[
\frac{\delta l(\theta)}{\delta \theta} = \frac{7}{\theta} + 2  \frac{1}{\frac{1-\theta}{3}} \cdot \left( -\frac{1}{3} \right) + \frac{1}{1-\theta} \cdot (-1) = \frac{7}{\theta} - \frac{3}{1 - \theta}
\]

\[
\frac{7}{\theta} - \frac{3}{1 - \theta} = 0
\]

\[
7 - 7\theta - 3\theta = 0
\]

\[
7 - 10\theta = 0
\]

\[
10\theta = 7
\]

\[
\theta = \frac{7}{10} 
\] 
 
\textit{\large Estimador de máxima verosimilitud : $\theta = \frac{7}{10}$ }

\end{document}