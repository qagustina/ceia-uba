\documentclass{article}
\usepackage{amsmath}
\usepackage{amssymb}
\title{\bf Análisis Matemático para Inteligencia Artificial (CEIA-AMIA)}
\author{Agustina Quiros}     

\begin{document}
\maketitle

{\bfseries Guía 2. Ejercicio 5.} 

El objetivo es que la relación recursiva de la secuencia de Fibonacci se pueda expresar de la forma matricial:
\[
\mathbf{x}^{(k)} = A \mathbf{x}^{(k-1)}
\]

Tenemos que:
\[
\mathbf{x}^{(0)} = \begin{bmatrix}
1 \\
0
\end{bmatrix}
\quad
\mathbf{x}^{(k)} = \begin{bmatrix}
f_{k+1} \\
f_k
\end{bmatrix}
\quad
\mathbf{x}^{(k-1)} = \begin{bmatrix}
f_k \\
f_{k-1}
\end{bmatrix}
\]

La matriz \( A \) es:
\[
A = \begin{bmatrix}
a & b \\
c & d
\end{bmatrix}
\]

Multiplicamos la matriz \( A \) por el vector \(\mathbf{x}^{(k-1)}\):
\[
A \cdot \mathbf{x}^{(k-1)}
=
\begin{bmatrix}
a & b \\
c & d
\end{bmatrix}
\cdot
\begin{bmatrix}
f_k \\
f_{k-1}
\end{bmatrix}
=
\begin{bmatrix}
a f_k + b f_{k-1} \\
c f_k + d f_{k-1}
\end{bmatrix}
\]

Queremos igualar la siguiente expresión:
\[
\begin{bmatrix}
f_k + f_{k-1} \\
f_k
\end{bmatrix}
\]
Igualando:
\[
a f_k + b f_{k-1} = f_k + f_{k-1}
\]
\[
c f_k + d f_{k-1} = f_k
\]


1.  Coeficiente de \( f_k \) : \( a = 1 \) y \( f_{k-1} \): \( b = 1 \)

2.  Coeficiente de \( f_k \) : \( c = 1 \) y \( f_{k-1} \) : \( d = 0 \)

Entonces, la matriz \( A \) queda:
\[
A = \begin{bmatrix}
1 & 1 \\
1 & 0
\end{bmatrix}
\]

Multiplicamos \( A \) por el vector \(\mathbf{x}^{(k-1)}\):
\[
A \cdot \begin{bmatrix}
f_k \\
f_{k-1}
\end{bmatrix}
=
\begin{bmatrix}
1 & 1 \\
1 & 0
\end{bmatrix}
\cdot
\begin{bmatrix}
f_k \\
f_{k-1}
\end{bmatrix}
=
\begin{bmatrix}
1 \cdot f_k + 1 \cdot f_{k-1} \\
1 \cdot f_k + 0 \cdot f_{k-1}
\end{bmatrix}
=
\begin{bmatrix}
f_k + f_{k-1} \\
f_k
\end{bmatrix}
\]

Y esto es igual a:
\[
\begin{bmatrix}
f_{k+1} \\
f_k
\end{bmatrix}
\]

La relación de las variables de forma matricial es:
\[
\mathbf{x}^{(k)} = A^k \cdot \mathbf{x}^{(0)}
\]
donde:
\[
A = \begin{bmatrix}
1 & 1 \\
1 & 0
\end{bmatrix}
\]
y
\[
\mathbf{x}^{(0)} = \begin{bmatrix}
1 \\
0
\end{bmatrix}
\]

Esta matriz \( A \) permite expresar la relación recursiva de la secuencia de Fibonacci en forma matricial.


\section*{Autovalores}

Sea \( A \) la matriz
\[
A = \begin{bmatrix}
1 & 1 \\
1 & 0
\end{bmatrix}
\]

La función característica \( p(\lambda) \) se define como
\[
p(\lambda) = \det (A - \lambda I)
\]

\[
A - \lambda I = \begin{bmatrix}
1 & 1 \\
1 & 0
\end{bmatrix} - \lambda \begin{bmatrix}
1 & 0 \\
0 & 1
\end{bmatrix} = \begin{bmatrix}
1 - \lambda & 1 \\
1 & -\lambda
\end{bmatrix}
\]

Determinante
\[
\det \begin{bmatrix}
1 - \lambda & 1 \\
1 & -\lambda
\end{bmatrix} = (1 - \lambda)(-\lambda) - (1 \cdot 1) = -\lambda^2 - \lambda - 1 
\]
Resolviendo:
\[
\lambda^2 - \lambda - 1 = 0
\]
Utilizando la fórmula resolvente
\[
\lambda = \frac{-b \pm \sqrt{b^2 - 4ac}}{2a}
\]
donde \( a = -1 \), \( b = -1 \), y \( c = -1 \)

Las soluciones son
\[
\lambda_1 = \frac{1 + \sqrt{5}}{2}
\]
\[
\lambda_2 = \frac{1 - \sqrt{5}}{2}
\]

\begin{enumerate}
    \item Multiplicidad algebraica de \( \lambda_1 \) = 1.
    \item Multiplicidad algebraica de \( \lambda_2 \) = 1.
\end{enumerate}

\section*{Autovectores}

\[
(A - \lambda I)x = 0
\]

\[
A - \lambda_1 I = \begin{bmatrix}
1 - \frac{1 + \sqrt{5}}{2} & 1 \\
1 & -\frac{1 + \sqrt{5}}{2}
\end{bmatrix}
\]
Simplificando
\[
\begin{bmatrix}
\frac{1 - \sqrt{5}}{2} & 1 \\
1 & -\frac{1 + \sqrt{5}}{2}
\end{bmatrix}
\]

\[
\begin{bmatrix}
\frac{1 - \sqrt{5}}{2} & 1 \\
1 & -\frac{1 + \sqrt{5}}{2}
\end{bmatrix}
\begin{bmatrix}
v_1 \\
v_2
\end{bmatrix}
=
\begin{bmatrix}
0 \\
0
\end{bmatrix}
\]

Sistema de ecuaciones:
\[
\left\{
\begin{array}{l}
\frac{1 - \sqrt{5}}{2} v_1 + v_2 = 0 \\
v_1 - \frac{1 + \sqrt{5}}{2} v_2 = 0
\end{array}
\right.
\]

Tomando \( v_1 = 1 \)

- Autovector de \( \lambda_1 \):
\[
 \lambda_1 =
\begin{bmatrix}
1 \\
-\frac{1 + \sqrt{5}}{2}
\end{bmatrix}
\]

Para \( \lambda_2 = \frac{1 - \sqrt{5}}{2} \):
\[
A - \lambda_2 I = \begin{bmatrix}
1 - \frac{1 - \sqrt{5}}{2} & 1 \\
1 & -\frac{1 - \sqrt{5}}{2}
\end{bmatrix}
\]
Simplificando
\[
A - \lambda_2 I = \begin{bmatrix}
\frac{1 + \sqrt{5}}{2} & 1 \\
1 & -\frac{1 - \sqrt{5}}{2}
\end{bmatrix}
\]

Luego
\[
\begin{bmatrix}
\frac{1 + \sqrt{5}}{2} & 1 \\
1 & -\frac{1 - \sqrt{5}}{2}
\end{bmatrix}
\begin{bmatrix}
v_1 \\
v_2
\end{bmatrix}
=
\begin{bmatrix}
0 \\
0
\end{bmatrix}
\]

Sistema de ecuaciones:
\[
\left\{
\begin{array}{l}
\frac{1 + \sqrt{5}}{2} v_1 + v_2 = 0 \\
v_1 - \frac{1 + \sqrt{5}}{2} v_2 = 0
\end{array}
\right.
\]

- Autovector de \( \lambda_2 \):
\[
\lambda_2 =
\begin{bmatrix}
1 \\
-\frac{1 + \sqrt{5}}{2}
\end{bmatrix}
\]

\begin{enumerate}
    \item Multiplicidad geometrica de \( \lambda_1 \) = 1.
    \item Multiplicidad geometrica de \( \lambda_2 \) = 1.
\end{enumerate}

{\bfseries Dado que para todos los autovalores la multiplicidad algebraica coincide con la geométrica, la matriz $A$ es diagonizable.} 


\section*{Diagonalización de la Matriz $A$}

Dada la matriz
\[
A = \begin{bmatrix}
1 & 1 \\
1 & 0
\end{bmatrix},
\]

La matriz de autovectores \(S\) se forma colocando los autovectores como columa:
\[
S = \begin{bmatrix}
1 & 1 \\
\frac{\sqrt{5} - 1}{2} & -\frac{1 + \sqrt{5}}{2}
\end{bmatrix},
\]


La matriz \(D\) se forma colocando los autovalores en la diagonal:

\[
D = \begin{bmatrix}
\frac{1 + \sqrt{5}}{2} & 0 \\
0 & \frac{1 - \sqrt{5}}{2}
\end{bmatrix}.
\]

La inversa de \(S\):

El determinante de \( S \) se calcula como:
\[
\det(S) = \begin{vmatrix}
1 & 1 \\
\frac{\sqrt{5} - 1}{2} & -\frac{1 + \sqrt{5}}{2}
\end{vmatrix}
\]

\[
\det(S) = 1 \cdot \left(-\frac{1 + \sqrt{5}}{2}\right) - 1 \cdot \left(\frac{\sqrt{5} - 1}{2}\right)
\]

\[
\det(S) = -\frac{1 + \sqrt{5}}{2} - \frac{\sqrt{5} - 1}{2}
\]

\[
\det(S) = -\frac{1 + \sqrt{5} + \sqrt{5} - 1}{2}
\]

\[
\det(S) = -\frac{2\sqrt{5}}{2} = -\sqrt{5}
\]


La matriz adjunta \(\text{adj}(S)\) 

\[
\text{adj}(S) = \begin{bmatrix}
-\frac{1 + \sqrt{5}}{2} & -1 \\
-\frac{\sqrt{5} - 1}{2} & 1
\end{bmatrix}.
\]

Entonces, la diagonalización de la matriz \(A\) se expresa como:
\[
A = S D S^{-1}.
\]

Donde:
\[
S = \begin{bmatrix}
1 & 1 \\
\frac{\sqrt{5} - 1}{2} & -\frac{1 + \sqrt{5}}{2}
\end{bmatrix},
\]
\[
D = \begin{bmatrix}
\frac{1 + \sqrt{5}}{2} & 0 \\
0 & \frac{1 - \sqrt{5}}{2}
\end{bmatrix},
\]
\[
S^{-1} = \frac{1}{\sqrt{5}} \begin{bmatrix}
\frac{1 + \sqrt{5}}{2} & 1 \\
\frac{\sqrt{5} - 1}{2} & -1
\end{bmatrix}.
\]


\subsection*{Fórmula explícita para el \( k \)-ésimo elemento de la serie de Fibonacci}

Dada la matriz $A$ y utilizando la diagonalización obtenida anteriormente, podemos hacer:


{\bfseries Potencias de la Matriz Diagonal} 

Elevamos \( A \) a la potencia \( k \):
\[
A^k = (S D S^{-1})^k = S D^k S^{-1}.
\]
La matriz diagonal \( D \) tiene la forma:
\[
D = \begin{bmatrix}
\lambda_1 & 0 \\
0 & \lambda_2
\end{bmatrix},
\]
donde \(\lambda_1\) y \(\lambda_2\) son los autovalores de \( A \). La matriz \( D^k \) se calcula como:
\[
D^k = \begin{bmatrix}
\lambda_1^k & 0 \\
0 & \lambda_2^k
\end{bmatrix}.
\]

{\bfseries Expresión para \( x(k) \)} 

El vector \( x(k) \) se expresa como:
\[
x(k) = A^k x(0),
\]
donde \( x(0) \) es el vector inicial dado por:
\[
x(0) = \begin{bmatrix}
f_1 \\
f_0
\end{bmatrix} = \begin{bmatrix}
1 \\
0
\end{bmatrix}.
\]

{\bfseries Fórmula Explícita} 

Sustituimos la expresión de \( A^k \) en \( x(k) \):
\[
x(k) = S D^k S^{-1} x(0).
\]

El resultado de esta operación nos proporciona los términos \( f_{k+1} \) y \( f_k \). De esta manera, obtenemos una fórmula explícita para \( f_k \). La fórmula de Binet, que resulta de este proceso, es:
\[
f_k = \frac{(1 + \sqrt{5})^k - (1 - \sqrt{5})^k}{2^k \sqrt{5}}.
\]


\end{document}
