\documentclass{article}
\usepackage{amsmath}
\usepackage{amssymb}
\title{\bf Análisis Matemático para Inteligencia Artificial (CEIA-AMIA)}
\author{Agustina Quiros}     

\begin{document}
\maketitle

{\bfseries Guía 1. Ejercicio 7.} 

Mostrar que 
\[
\langle A, B \rangle = \text{Tr}(AB^H)
\]
define un producto interno en $(C)^{n \times m}$. A este p.i. se lo conoce como producto
interno de Frobenius. (La operación $A^H$ representa A transpuesta y conjugada). \\

{\bfseries Espacios con Producto Interno: Definición.} 
Sea \( V \) un espacio vectorial sobre \( K \), donde \( K = \mathbb{R} \) o \( \mathbb{C} \). Un producto interno sobre \( V \) es una función \(\Phi : V \times V \to \mathbb{R} \) (o \(\mathbb{C}\)) que satisface:

1. Para cada \(\alpha \in \mathbb{R}\) (o \(\mathbb{C}\)), \( u, v, w \in V \):
\[
\Phi(u + v, w) = \Phi(u, w) + \Phi(v, w)
\]
\[
\Phi(\alpha \cdot u, v) = \alpha \cdot \Phi(u, v)
\]

2. 
\[
\Phi(u, v) = \overline{\Phi(v, u)}
\]

3. 

\[
\Phi(v, v) \geq 0,\Phi(v, v) = 0 \quad \text{sii} \quad v = 0
\]


\textit{Demostración}

\textbf{1. a.}

\[
    \langle A + B, C \rangle = \langle A, C \rangle + \langle B, C \rangle
\]

\[
\begin{array}{rcl}
\langle A + B, C \rangle & = & \text{Tr}[(A + B)C^H] \\
                         &   & \quad (\text{1}) \\
                         & = & \text{Tr}(AC^H + BC^H) \\
                         &   & \quad (\text{2}) \\
                         & = & \text{Tr}(AC^H) + \text{Tr}(BC^H) \\
                         &   & \quad (\text{3}) \\
                         & = & \langle A, C \rangle + \langle B, C \rangle
\end{array}
\]
\begin{enumerate}
    \item Por definición del producto interno de Frobenius y propiedad de la traza de una matriz.
    \item Por la propiedad distributiva de las matrices.
    \item Por propiedad de la traza de una matriz que establece que la traza de una suma es igual a la suma de las trazas.
\end{enumerate}


\textbf{1. b.}
\[
\langle \alpha A, B \rangle = \alpha \langle A, B \rangle
\]

\[
\begin{array}{rcl}
\langle \alpha A, B \rangle & = & \text{Tr}[(\alpha A) B^H] \\
                            &   & \quad (\text{1}) \\
                            & = & \text{Tr}[\alpha (A B^H)] \\
                            &   & \quad (\text{2}) \\
                            & = & \alpha \text{Tr}(A B^H) \\
                            &   & \quad (\text{3}) \\
                            & = &  \alpha \langle A, B \rangle
\end{array}
\]

\begin{enumerate}
    \item Por definición del producto interno de Frobenius.
    \item Por propiedad lineal de la traza de una matriz.
    \item Por la propiedad de la traza de una matriz respecto a un escalar. 
\end{enumerate}

\textbf{2.}
\[
\langle A, B \rangle = \overline{\langle B, A \rangle}
\]

\[
\begin{array}{rcl}
\overline{\langle B, A \rangle} & = & \overline{\text{Tr}\langle  BA^H \rangle} \\
                         &   & \quad (\text{1}) \\
                         & = & \text{Tr}((AB^H)^H) \\
                         &   & \quad (\text{2}) \\
                         & = & \text{Tr}(AB^H) \\
                         &   & \quad (\text{3}) \\
                         & = & \langle A, B \rangle
\end{array}
\]

\begin{enumerate}
    \item Por propiedad de la traza de una matriz y propiedad de matriz transpuesta conjugada.
    \item Por propiedad de la matriz conjugada traspuesta de inversión de la multiplicación.
    \item Por propiedad involutiva de la matriz adjunta.
\end{enumerate}

Ejemplo. Sean \( A, B \) matrices complejas de tamaño \( nxn \):

\[
A = \begin{pmatrix}
1 + i & 2 - i \\
-i & 3
\end{pmatrix}
B = \begin{pmatrix}
    2  & 1 + 2i \\
    i & -1
    \end{pmatrix}
\]  
    
\[
AB^H = 
\begin{pmatrix}
1 + i & 2 - i \\
-i & 3
\end{pmatrix}
\begin{pmatrix}
2 & -i \\
1-2i & -1
\end{pmatrix}
\]
        
\[
AB^H = \begin{pmatrix}
(1+i)\cdot 2 + (2-i)\cdot (1-2i) & (1+i)\cdot(-i) + (2-i)\cdot(-1) \\
(-i)\cdot 2 + 3\cdot (1-2i) & (-i)\cdot (-i) + 3\cdot (-1)
\end{pmatrix}
\]

\[
Tr(AB^H) = (1+i)\cdot 2 + (2-i)\cdot (1-2i) + (-i)\cdot (-i) + 3\cdot (-1)
\]

Luego,
\[
BA^H = 
\begin{pmatrix}
    2  & 1 + 2i \\
    i & -1
    \end{pmatrix}
\begin{pmatrix}
    1-i  & i \\
    2+i & 3
    \end{pmatrix}    
\]

\[
BA^H = 
\begin{pmatrix}
    2 \cdot (1 - i) + (1 + 2i) \cdot (2 + i) & 2 \cdot i + (1 + 2i) \cdot 3 \\
    i \cdot (1 - i) + (-1) \cdot (2 + i) & i \cdot i + (-1) \cdot 3
\end{pmatrix}
\]

\[
Tr(BA^H) = 2\cdot (1-i) + (1+2i)\cdot (2+i) + i\cdot i + (-1) \cdot (-3)
\]

Queda demostrado que: 
\[
    Tr(AB^H) = \overline{Tr(BA^H)}
\]

\textbf{3.}

\[
\langle A, A \rangle \geq 0
\]

\[
\langle A, A \rangle  =  \text{Tr}(A A^H) \geq 0 
\]

Dado que \(A A^H\) es una matriz hermitiana y semidefinida positiva (todos sus valores propios son no negativos), la traza de una matriz hermitiana semidefinida positiva es siempre no negativa.

Por lo tanto:

\[
\text{Tr}(A A^H) \geq 0
\]

Además, \(\text{Tr}(A A^H) = 0\) si y solo si \(A A^H = 0\), lo cual sucede si y solo si \(A = 0\).

Por lo tanto:

\[
\langle A, A \rangle = 0 \quad \text{si y solo si} \quad A = 0
\]

Ejemplo. Sea \( A \) una matriz compleja de tamaño \( nxn \):

\[
A = \begin{pmatrix}
1 + i & 2 \\
3i & -1
\end{pmatrix}
\]

La matriz conjugada transpuesta \( A^H \) es:

\[
A^H = \begin{pmatrix}
1 - i & -3i \\
2 & -1
\end{pmatrix}
\]

\[
A A^H = \begin{pmatrix}
1 + i & 2 \\
3i & -1
\end{pmatrix}
\begin{pmatrix}
1 - i & -3i \\
2 & -1
\end{pmatrix}
\]

\[
A A^H = \begin{pmatrix}
(1+i)(1-i) + 2 \cdot 2 & (1+i)(-3i) + 2 \cdot (-1) \\
3i(1-i) + (-1)(2) & 3i(-3i) + (-1)(-1)
\end{pmatrix}
\]

\[
\text{Tr}(A A^H) = (1+i)(1-i) + 2 \cdot 2 + 3i \cdot (-3i) + (-1) \cdot (-1) \geq 0
\]


Queda demostrado que: \[\text{Tr}(A A^H) \geq 0.\]

\end{document}
